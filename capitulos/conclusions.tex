\chapter{Conclusións e posibles ampliacións}
Ao longo deste proxecto deseñamos e implementamos un sistema que permite levar
partidas de rol tipo play-by-post e ofrecerlle ao director de xogo ferramentas
para manter a consistencia do mundo. Este sistema está composto por dous
compoñentes diferenciados e independentes, o motor de xogo e o servidor web.
Grazas a isto, é posible para calquera usuario montar o seu propio sistema de
xogo nun servidor, pero ademais disto é posible tamén para outros
desenvolvedores deseñar novas interfaces de xogo que se adapten ás súas
necesidades, empregando o mesmo motor de xogo.
\par
No proxecto optouse por seguir unha metodoloxía áxil de tipo Scrum. Esta
decisión demostrou ser un acerto, xa que  permitiu obter versións executables do
software dende etapas moi temperás do desenvolvemento, ademais de permitir unha
reacción rápida ante a aparición de problemas ou o cambio de requisitos grazas
á flexibilidade que proporcionan as metodoloxías áxiles. Tamén foi determinante
á hora de facilitar o uso de tecnoloxías nas que o desenvolvedor conta con
experiencia escasa e coas que resultaría difícil facer estimacións precisas de
tempo.
\par
Como xa se resaltou ao longo do presente documento, o desenvolvemento da
linguaxe COE foi un piar central no desenvolvemento deste proxecto. A súa
definición e implementación levou a maior parte da carga de traballo, e a súa
finalización foi condición necesaria para comezar co resto de tarefas. A maioría
dos requisitos do proxecto están de feito relacionados con este compoñente.
Entre algunhas das súas características atopamos:
\begin{itemize}
  \item É unha linguaxe orientada a obxectos, que permite definir herdanza de
  clases entre outras características.
  \item Emprega distintos tipos de datos, tanto tipos primitivos como tipos
  asociados ás partidas de rol, principalmente obxectos e escenarios.
  \item Permite realizar operacións básicas, operacións propias do xogo de rol e
  definir métodos asociados ás clases.
\end{itemize}
\par
O uso de tecnoloxías como ANTLR ou Spring foi determinante por dúas cuestións: a
axilidade que proporcionan para definir a linguaxe e deseñar a web
respectivamente foi crucial para cumprir cos prazos asignados. Por outra banda,
o emprego destas tecnoloxías permitiu facer ampliacións tanto na linguaxe como
na web con facilidade, ofrecendo a posibilidade de facer melloras en pouco tempo
en base ás probas con usuarios reais.
\par
O sistema permite numerosas ampliacións. A máis evidente é o desenvolvemento de
novas interfaces, como unha app móbil ou un apartado web deseñado para móbiles e
tabletas. Mais en realidade as posibilidades son moi amplas grazas á
versatilidade dos servizos REST, puidendo comunicarse incluso dende redes
sociais como Telegram para axilizar o acesso por parte dos usuarios.
\par
Outra posible ampliación que se valorou implementar durante o proxecto pero foi
descartada por falta de tempo é o engadido dun servidor de chat XMPP no
DemiurgoWeb, que facilite a comunicación inmediata entre o director de xogo e os
xogadores, o cal podería enriquecer a experiencia de xogo notablemente.
\par
Ademais disto, valorouse implementar un xestor de temas en DemiurgoWeb que
permitise cambiar a estética da web segundo as preferencias do usuario: cores,
imaxes, tipografías\ldots Esta opción quedou relegada a ampliación futura, mais
non entra dentro do alcance do proxecto actual.
\par
Como se pode ver nestes exemplos, o feito de separar o motor de xogo da
interface gráfica ofrece numerosas posibilidades de mellora; xa que sempre é
posible engadir campos extra para o seu uso na interface. Un exemplo desta
aplicación é o uso que se fai en DemiurgoWeb do campo \textbf{str v\_imgurl}:
cando un obxecto posúe este campo e ten asignado a el unha \textbf{URL}, o
DemiurgoWeb interprétao como un enlace á imaxe propia dese obxecto, mostrándoa
por pantalla.
