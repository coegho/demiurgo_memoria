\chapter{Conclusións e posibles ampliacións}
Ao longo deste proxecto deseñamos e implementamos un sistema que permite levar
partidas de rol tipo play-by-post e ofrecerlle ao director de xogo ferramentas
para manter a consistencia do mundo. Este sistema está composto por dous
compoñentes diferenciados e independentes, o motor de xogo e o servidor web.
Grazas a isto, é posible para calquera usuario montar o seu propio sistema de
xogo nun servidor, pero ademais disto é posible tamén para outros
desenvolvedores deseñar novas interfaces de xogo que se adapten ás súas
necesidades, empregando o mesmo motor de xogo.
\par
O sistema permite numerosas ampliacións. A máis evidente é o desenvolvemento de
novas interfaces, como unha app móbil ou un apartado web deseñado para móbiles e
tables. Mais en realidade as posibilidades son moi amplas grazas á versatilidade
dos servizos REST, puidendo comunicarse incluso dende redes sociais como
Telegram para axilizar o acesso por parte dos usuarios.
\par
Outra posible ampliación que se valorou implementar durante o proxecto pero foi
descartada por falta de tempo é o engadido dun servidor de chat XMPP no
DemiurgoWeb, que facilite a comunicación inmediata entre o director de xogo e os
xogadores, o cal podería enriquecer a experiencia de xogo notablemente.
\par
Como se pode ver nestes exemplos, o feito de separar o motor de xogo da
interface gráfica ofrece posibilidades ilimitadas de mellora; xa que sempre é
posible engadir campos extra para o seu uso na interface, tal e como fai
DemiurgoWeb para engadir as imaxes dos obxectos.
