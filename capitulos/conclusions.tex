\chapter{Conclusións e posibles ampliacións}
O presente proxecto tiña por obxecto o deseño e implementación dun software
orientado ao xogo de rol en liña. O xogo de rol é unha actividade que consiste
na elaboración dunha narrativa común entre os participantes, onde por unha banda
os xogadores interpretan aos seus personaxes, e por outra o director de xogo
organiza e dirixe a partida.

Nunha partida de rol típica, o director define un mundo de xogo poboado por
obxectos que interactúan entre si, entre eles os propios personaxes  dos
xogadores. Os xogadores poden describir as súas accións empregando a linguaxe
natural, pero quedará en mans do director de xogo interpretalas e elaborar unha
narrativa común a todos. Esta modalidade de xogo é coñecida co nome de
\textit{play-by-post}. O resultado do presente proxecto é unha plataforma
software que permite que os xogadores poidan enviar as súas decisións, pero
ofrecendo ao director ferramentas para manter unha equivalencia dixital do mundo, que pode manipular mediante o uso dunha linguaxe
de \textit{script}. O produto resultante, por tanto, é unha
ferramenta de xestión para crear e manter partidas de rol en liña, nas que
xogadores e director de xogo poidan comunicarse e xerar unha dinámica de xogo de
forma sinxela e intuitiva.

A linguaxe de script empregada para definir o mundo de xogo e interactuar con el
é un compoñente singular deste proxecto. Recibe o nome de \textbf{linguaxe COE},
acrónimo de \textit{Código de Obxectos e Escenarios}, os dous elementos
fundamentais no sistema. Esta linguaxe, orientada ás necesidades do xogo de rol, contén todas as funcionalidades e operacións
necesarias para definir o sistema de xogo, ofrecendo unha considerable
flexibilidade á hora de implementar as distintas mecánicas e permitindo que o
software se adapte ás normas empregadas na partida.
Ademais disto, tamén ofrece características típicas de outras linguaxes, tales
como operacións aritméticas básicas, tipos de datos primitivos, e
funcionalidades como a herdanza de clases propia das linguaxes orientadas a
obxectos. No proxecto inclúese un manual no apéndice \ref{ch:manual} para
coñecer e aprender a usar esta linguaxe.

Sendo un elemento fundamental no proxecto, a linguaxe COE e o seu correspondente
intérprete conlevaron a maior carga de traballo e foron a
principal prioridade no mesmo, dende a análise de requisitos ata o plan de
probas.

Optouse por seguir unha metodoloxía áxil de tipo Scrum. Esta decisión demostrou
ser un acerto dadas as características do proxecto: tendo en conta que a
plataforma software aborda un dominio con escasas aportacións tecnolóxicas,
deuse por feito que sería frecuente a aparición de novos requisitos e a modificación de requisitos existentes, tendo como exemplo máis claro os progresivos cambios na linguaxe COE
para adaptarse correctamente a novos casos de uso procedentes da comunidade de
xogo de rol en liña.
A maiores, o autor deste proxecto conta cunha experiencia limitada tanto en
xestión de proxectos grandes como no uso das tecnoloxías aquí empregadas. Por
tanto, decidiuse seguir a metodoloxía de Scrum e descartar outras opcións menos
tolerantes ao cambio.

A principal característica da arquitectura do sistema é a división do mesmo en
dous compoñentes diferenciados. O primeiro destes compoñentes é o motor de xogo,
que recibe internamente o nome de \textbf{Demiurgo}. Este elemento é o
núcleo fundamental do sistema, encargándose tanto de manter o estado do mundo de
xogo como de interpretar a linguaxe COE. Por outra banda, o outro compoñente é o
servidor web, co nome de \textbf{DemiurgoWeb}, que ofrece unha interface
interactiva aos usuarios para comunicarse co sistema.

No desenvolvemento deste software empregáronse numerosas tecnoloxías, pero
destaca especialmente o uso de \textit{ANTLR} como ferramenta para elaborar un
intérprete de código a partir da definición dunha gramática.
ANTLR usouse neste proxecto para crear o intérprete da linguaxe COE sen ser
preciso implementar analizadores léxico e sintáctico, compoñentes que
ANTLR nos proporciona de maneira automática a partires dunha especificación
formal.
Isto ofreceunos unha axilidade no desenvolvemento que resultou crucial no proxecto, xa que permitiu elaborar unha
linguaxe relativamente complexa dentro duns prazos asumibles para o mesmo.

No caso de DemiurgoWeb, cabe destacar o uso do framework \textit{Spring}, que
proporciona numerosas ferramentas útiles para deseñar unha web de forma rápida e
sinxela. \textit{Spring} baséase no patrón modelo-vista-controlador, por
tanto diferencia claramente os compoñentes estritamente visuais da lóxica
interna e a comunicación co motor de xogo. A maiores, a web conta con diferentes
plugins baseados en \textit{Javascript} ou en \textit{JQuery} que proporcionan
ao usuario un entorno interactivo.


\section{Traballo futuro}
O proxecto cumpre con todos os requisitos esenciais definidos ao inicio do
mesmo, ofrecendo deste xeito un programa completo e plenamente funcional. A
maiores, algúns requisitos opcionais pospuxéronse pola súa menor prioridade,
mais xa hai formuladas liñas de traballo para implementalos no futuro.

Neste senso, a plataforma software aquí presentada dá pé a varias ampliacións,
sendo de mención obrigada o desenvolvemento de novas interfaces adicionais a maiores do DemiurgoWeb; isto é,
\textit{apps} orientadas ao mercado de móbiles e tabletas, novos deseños
temáticos no propio DemiurgoWeb ou outras aplicacións web. Ademais:

\begin{itemize}

\item Cabe destacar nesta liña a implementación xa realizada dun \textit{bot}
para o servizo de mensaxería instantánea \textit{Telegram} que permite interactuar co
sistema a través da súa rede social.
Esta aplicación, ideada na etapa final do proxecto como unha vía para chegar potencialmente a un público
maior, xa conta cunha liña de traballo definida para o seu desenvolvemento, e
dispón na actualidade dun prototipo inicial.

\item O desenvolvemento de novas formas de comunicarse co motor de xogo é unha
tarefa sinxela grazas ao uso de servizos REST no motor de xogo, que permiten
establecer a comunicación co mesmo mediante peticións web. O presente documento
conta cunha API detallada destes servizos no apéndice \ref{ch:api}.

\item O engadido dun chat XMPP ao servidor web foi outra opción valorada no
proxecto, que xa está planificada como unha futura ampliación. Supón un modo de
permitir a comunicación directa entre os xogadores e o director de xogo en
tempo real, enriquecendo deste xeito a experiencia de xogo.

\item Finalmente, no ámbito do motor de xogo tamén hai espazo para melloras. As
principais ampliacións formuladas para o futuro son: a
adaptación a bases de datos diferentes a MySQL, tales como
sistemas no-SQL ou o uso de ficheiros YAML; e a ampliación da linguaxe COE para
proporcionar novas funcionalidades, como un tipo de datos \textit{map}
semellante á clase \textit{Map} de Java, ou visibilidade para campos e métodos
que afecten á forma de visualizalos na interface gráfica, facendo que algúns
atributos dos obxectos non se mostren por pantalla.
\end{itemize}

\section{Comunidade e visibilidade do proxecto}
O proxecto está enfocado á comunidade dos xogadores de rol. Céntrase por tanto
nun sector do mercado moi específico, e convén coñecelo á hora de promocionar
o software. A comunidade de rol é relativamente pequena en comparación con
outras, pero está fortemente interrelacionada, e funciona con especial eficacia
o ``boca a boca''. Isto, combinado co feito de que na actualidade non existen
ferramentas que cumpran a función para a que esta plataforma software foi
desenvolvida, proporciona a posibilidade de difundir este produto a un número
considerable de usuarios.

Un elemento a destacar neste proxecto é o feito de usar a xogadores reais no
plan de probas. Na fase final do proxecto instalouse o sistema nun servidor VPS
para permitir o seu acceso a calquera usuario, e deixouse o papel de director de
xogo en mans dun usuario alleo ao proxecto pero interesado no mesmo.
Adicionalmente, contactouse con persoas interesadas nos xogos de rol para
participar na partida. Isto, ademais de proporcionarlle unha maior fiabilidade
ás probas de usabilidade, ofrécenos a vantaxe de dar a coñecer o proxecto a un
conxunto inicial de persoas pertencentes á comunidade do rol, que poderán
recomendalo aos seus amigos ou en foros de internet se quedan satisfeitos coa
experiencia de xogo.

Cabe destacar o aspecto da internacionalización do sistema. A pesar de atoparse
a interface web unicamente en inglés actualmente, o uso de Spring facilita
a creación de ficheiros de idioma adicionais para traducir DemiurgoWeb a
distintas linguas. A maiores, por fundamentarse o sistema no código e nas
narracións elaboradas polo propio director de xogo, é posible facer partidas de
rol nos idiomas que os xogadores prefiran, relegando a tradución dos elementos
da interface a unha cuestión secundaria.

En canto á comunidade de desenvolvemento, tanto o código fonte do Demiurgo como
do DemiurgoWeb están ubicados en repositorios públicos en \textit{GitHub},
ofrecendo a posibilidade a terceiros de realizar melloras en calquera dos dous
compoñentes, ou incluso iniciar un desenvolvemento propio. O código fonte está
escrito en inglés e conta con abundante documentación, facilitando a
colaboración a nivel internacional. Finalmente, o feito de contar cunha API do
motor de xogo fomenta o desenvolvemento de ferramentas de terceiros segundo as
súas necesidades.

Tamén forma parte da planificación a
promoción do produto a través de redes sociais tales como \textit{Facebook} ou
\textit{Twitter}, así como a súa presentación en foros adicados ao rol. É
importante para isto o feito de contar xa cunha primeira partida, para deste
xeito poder mostrar exemplos visuais aos usuarios potenciais. Para a
promoción do produto pódense resaltar características positivas á marxe do que é
estritamente entretemento, como o feito de que achegue a programación
informática aos usuarios por vía da linguaxe COE, ou os aspectos positivos dos
xogos de rol nas aulas.

En definitiva, este proxecto non só satisface os intereses dun pequeno colectivo
de usuarios, senón que pode estenderse a nivel internacional e obter unha
comunidade de usuarios considerable, aproveitando como oportunidade
a carencia de ferramentas semellantes, ademais das súas características
competitivas, como a facilidade para
desenvolver novas ferramentas a través da API.
