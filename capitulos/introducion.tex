\chapter{Introdución}
\hyphenation{In-te-re-se}

% Introdución: composta por Obxectivos Xerais, Relación da Documentación que
% conforma a Memoria, Descrición do Sistema, Información Adicional de Interese
% (métodos, técnicas ou arquitecturas utilizadas, xustificación da súa elección,
% etc.).
Un xogo de rol é un xogo de interpretación no que o conxunto de xogadores vai
desenvolvendo unha narrativa común, interpretando cada un deles a un personaxe
ficticio que inflúe na historia coas súas accións. Un dos participantes asume o
papel de mestre ou director de xogo, que é esencialmente o que asume a labor de
deseñar a historia e fiar os acontecementos, mentres que o resto de xogadores
toman decisións en base aos seus personaxes. Cando un xogador pretende facer
algunha acción con risco de fracasar, o éxito ou fracaso da súa acción
determínase mediante unha tirada de dados. Por tanto, podería dicirse que os
xogos de rol no seu concepto son un termo medio entre o teatro e os contacontos.
\par
A orixe dos xogos de rol é discutida, pero é habitual considerar que o xénero
queda plenamente definido con Dungeons and Dragons, publicado en 1974. Esta saga
leva os xogos de rol ao seu auxe comercial nos anos 80, onde xurden numerosos
xogos que seguen a súa estela. Non pasou moito tempo ata que este fenómeno se
extendeu aos ordenadores, especialmente na época de internet, o que propiciou
que se comezaran a organizar partidas de rol mediante correo electrónico e
posteriormente a través de foros, o que se pasou a denominar como rol asíncrono
debido a que os xogadores podían participar con horarios totalmente distintos.
\par
A finais dos 70 por outra banda comezou a desenvolverse software deseñado
especificamente para xogar ao rol, no que non era necesario un director de xogo
xa que o propio sistema era o que xestionaba toda a partida; un exemplo disto é
o software MUD (Multi-User Dungeon) e todos os seus descendentes, sendo o MUD1
(ano 1978) o primeiro mundo virtual existente. Isto tiña o inconveniente de que
o xogador só podía tomar decisións prefixadas no propio sistema e seguir
historias totalmente predeseñadas, sen apenas capacidade de improvisación real.
Cando este software foi emplazado en servidores en liña coa aparición de
Internet, o número de xogadores por partida aumentou pero o problema mantívose;
a diferencia dos foros de rol que aínda existen hoxe en día, nos que se mantén a
figura tradicional do director de xogo.
\par
A día de hoxe os xogos de rol están moi estendidos, especialmente entre a xente
nova, e na rede hai numerosos exemplos dos dous grandes grupos mencionados:
software de rol síncrono por un lado (coñecidos como MUD en referencia ao
software que iniciou este xénero), e foros e grupos de correo de rol asíncrono
polo outro. Non obstante, é no punto medio onde se atopa unha carencia: non hai
ningún software deseñado para partidas de rol asíncrono. O obxectivo deste
proxecto é precisamente dotar ao mercado dun software deseñado para facilitar o
traballo do director de xogo, que proporcione unha linguaxe de script para que o
director poida construír un mundo parcialmente automatizado, pero no que en
última instancia o xogador non estea restrinxido a un espectro limitado de
decisións.

\section{Obxectivos xerais}
O obxectivo deste traballo é o de deseñar e implementar unha ferramenta software
que facilite a dirección de partidas de rol asíncrono en liña, empregando para
isto unha linguaxe de script propia, é dicir:
\begin{itemize}
  \item Definir unha linguaxe formal que permita ao usuario crear un mundo virtual
  mediante a especificación de obxectos distintos cos seus atributos, entre eles os obxectos que representen
  aos propios personaxes dos xogadores, ademais de permitir definir interaccións preprogramadas
  entre os propios obxectos. Esta linguaxe debe estar plantexada para servir ao xogo de rol, polo
  que terá en conta elementos clave neste contexto: estancias (habitacións), visibilidade ou
  ocultación das accións e sucesos, cálculos para comprobar o resultado das accións, etc.
  \item Implementar dita linguaxe formal mediante un intérprete de script.
  \item Deseñar unha base de datos que dote de persistencia ao mundo virtual.
  \item Deseñar unha interface web para interactuar co sistema de tal modo que o
  director de xogo poida empregar a linguaxe de script directamente; os xogadores pola contra só precisan dunha caixa de
  texto para explicarlle ao director as accións que desexan realizar, quedando nas mans deste a
  interacción co mundo interno.
\end{itemize}

\section{Documentación}
O proxecto seguirá unha metodoloxía de desenvolvemento áxil, polo que a maior
parte da documentación obterase a partir de comentarios no propio código. A
maiores incluiranse toda a documentación adicional como diagramas de clases
ou diagramas entidade-relación. O documento estará dividido nos seguintes
capítulos (o capítulo 1 é esta propia introdución):
\begin{itemize}
  \item O capítulo 2 contén a planificación do proxecto, tanto os recursos
  necesarios como a propia planificación temporal.
  \item O capítulo 3 é o destinado á especificación de requisitos:
  definicións, casos de uso, e a lista de requisitos obtidos.
  \item O capítulo 4 referirase ao deseño do sistema, dende a interrelación dos
  distintos compoñentes á descrición en detalle do seu funcionamento.
\end{itemize}

\section{Descrición do sistema}
O proxecto consta dun software entregable formado por varios compoñentes
diferenciados pero interrelacionados entre si:
\begin{itemize}
  \item A linguaxe na que se fundamenta o sistema. O sistema precisa unha
  linguaxe orientada a obxectos, na que se poidan especificar clases cos seus
  atributos e métodos. Unha parte crucial deste traballo é a de describir
  formalmente esta linguaxe de forma que cumpra a súa función de definir o
  mundo virtual do xogo, tal e como se indican nos obxectivos deste
  anteproxecto. Debe realizarse unha análise das solucións abertas dispoñibles,
  e no caso de non atopar unha linguaxe que satisfaga todas as necesidades do
  sistema, será necesario implementar unha de cero co seu correspondente
  intérprete.
  \begin{itemize}
    \item Esta linguaxe deberá conter todo o necesario para un correcto
    tratamento da información:
    variables, operadores aritméticos e lóxicos, condicións lóxicas,
    modificación de atributos, etc.
    Todo isto debe ser debidamente especificado na etapa de deseño.
  \end{itemize}
  \item A interface web que tanto o director como os xogadores empregarán para
  interactuar co mundo.
  \begin{itemize}
    \item Os xogadores poderán en todo momento ver o estado do seu personaxe e
    ler o texto descritivo das accións sucedidas na súa presencia (redactado e
    filtrado segundo o criterio do director de xogo), ademais de escribir en
    linguaxe natural as actuacións que desexan realizar ({\it decisións} na
    terminoloxía do sistema) na seguinte iteración ou quenda. En ningún momento
    terán acceso á linguaxe de xogo propiamente dita, nin a datos do mundo
    virtual alleos ao seu personaxe.
    \item O director de xogo pola súa parte terá acceso a unha consola de comandos,
    que lle permitirá alterar o mundo virtual mediante o uso da linguaxe de
    xogo. Recibirá todas as peticións de actuación dos xogadores e escribirá o
    código en consecuencia que deberá ser executado; en tempo de execución o 
   director de xogo poderá instanciar novos obxectos, chamar aos métodos de
   obxectos xa instanciados ou modificar os seus atributos directamente.
  \end{itemize}
  \item A capa de software que mantén o estado do mundo e xestiona o resto de
  aspectos do sistema, tales como a xestión de usuarios, ademais de servir de
  ponte entre o resto de compoñentes.
  \item Outros compoñentes necesarios para o funcionamento do sistema: servidor
  web e base de datos relacional.
\end{itemize}
O software a entregar por tanto terá dous elementos diferenciados: a interface
web, sen lóxica interna e que se limitará a enviar e recibir mensaxes, e o
motor do xogo, que se conectará coa interface web para comunicarse, pero que
debe deseñarse tendo en mente a posibilidade de incorporar outros modos de
comunicarse con el no futuro, tales como software de escritorio para os
usuarios. Estas vías alternativas e o seu desenvolvemento, en todo caso, non
entran no alcance deste proxecto concreto.

\section{Información adicional de interese}
% métodos, técnicas ou arquitecturas utilizadas, xustificación da súa elección
\subsection{Tecnoloxías empregadas}

\subsubsection{Java SE}
Para o desenvolvemento do software deste Traballo de Fin de Grao escolleuse Java
como linguaxe para os distintos compoñentes.
Os principais argumentos a favor para realizar esta escolla foron os seguintes:
\begin{itemize}
  \item Java é unha linguaxe amplamente difundida, cunha extensa documentación
  na rede. Ademais disto, o autor deste TFG está familiarizado coa súa sintaxe.
  \item Dispón de numerosas librarías que permiten desenvolver os distintos
  compoñentes do proxecto. Dispón en particular de ferramentas para desenvolver
  analizadores léxico-sintácticos, e de librarías para montar sistemas web.
  \item É compatible coa maioría de entornos de desenvolvemento, nomeadamente
  Eclipse, entorno que se empregará neste proxecto.
\end{itemize}

\subsubsection{ANTLR}
Un dos pilares centrais deste Traballo de Fin de Grao é a linguaxe de script
empregada polo Director de Xogo para comunicarse co sistema. Debido á alta
especifidade deste proxecto, optouse por deseñar unha linguaxe de
cero no canto de adaptar unha linguaxe de propósito xeral.
\par
Escolleuse ANTLR como ferramenta para deseñar a linguaxe e xerar os
correspondentes analizadores léxico e sintáctico. ANTLR trátase dunha ferramenta
que a partir dunha gramática elabora un analizador sintáctico descendente;
grazas a isto podemos deseñar unha gramática que satisfaga os requisitos do
proxecto e deixar en mans de ANTLR o proceso de parsing.

\subsubsection{Servizos web: Grizzly + Jersey}
Para a comunicación entre o servidor de xogo e o servidor web valoráronse
distintas posibilidades e métodos. Unha primeira aproximación suxeriu o uso de
Java RMI para comunicar ambos compoñentes, aproveitando o feito de que ambos
corrían sobre JVM. Non obstante, optouse finalmente por descartar esta
opción e empregar servizos web de tipo REST en base ás seguintes cuestións:
\begin{itemize}
  \item Java RMI só funciona entre programas que corran sobre JVM. A pesar de
  non ser este un problema no alcance actual deste proxecto, limita a evolución
  do sistema no futuro; empregando servizos web ábrese a porta a comunicarse co
  servidor de xogo mediante outras aplicacións desenvolvidas noutras
  tecnoloxías.
  \item Java RMI é unha tecnoloxía menos flexible cós servizos
  web, de mantemento máis complexo e que pode xerar dificultades cando os dous
  servidores se atopan en distintas ubicacións físicas. Pola contra, os servizos
  web só requiren dunha conexión vía HTTP entre ambos servidores, polo que son
  máis fáciles de empregar.
  \item Con Java RMI é necesario compartir código entre ambos servidores a
  través de librarías comúns para poder compartir POJOs ({\it Plain Old Java
  Objects}). Mediante os servizos web, en cambio, a comunicación pode realizarse
  mediante obxectos JSON.
\end{itemize}
Para implementar os servizos web optouse por empregar as librarías Grizzly e
Jersey, por tratarse da opción máis sinxela para poñer en funcionamento o
servidor. Por un lado, Grizzly ofrece todo o necesario para poder montar un
servidor web sen necesidade de programas adicionais, empregando simplemente o
executable da nosa aplicación. Polo outro, Jersey encárgase da recepción,
manipulación e resposta das peticións web recibidas, axilizando a
implementación do sistema.

\subsubsection{Apache Maven}
Tanto no servidor de xogo como no servidor web, decidiuse empregar Maven para
xestionar os distintos paquetes e dependencias. As principais vantaxes disto
son:
\begin{itemize}
  \item Ofrece unha estrutura de directorios estandarizada que facilita o
  mantemento do código.
  \item Simplifica a xestión de dependencias, descargando de forma automática os
  paquetes necesarios.
  \item Simplifica a execución de tests no código.
\end{itemize}

\subsubsection{Spring Framework}
Spring é un Framework que facilita o desenvolvemento de aplicacións complexas.
Ten módulos útiles para todo tipo de tarefas, tales como montar un servidor web,
empregar o patrón modelo-vista-controlador ou executar tests de JUnit. A
principal característica de Spring é a inxección de dependencias, que simplifica
o control do ciclo de vida dos obxectos: crear os obxectos, chamar aos seus
métodos de inicialización, e referencialos entre eles.
\par
Neste proxecto empregaremos Spring no compoñente web, é dicir, Spring axudará a
crear un servidor web que se conecte co servidor de xogo. Deste modo, aforrarase
tempo que poderá ser empregado no deseño da web.

\subsubsection{MariaDB}
MariaDB é un xestor de bases de datos derivado de MySQL. Empregaremos bases de
datos MariaDB no noso proxecto para manter a persistencia do sistema de xogo.

\subsubsection{Git}
Git é unha ferramenta para desenvolvedores que xestiona os cambios realizados no
código, mantendo un control de versións e facilitando o desenvolvemento de
software.
\par
Neste proxecto empregaremos repositorios Git para os distintos compoñentes.
Usaremos Github para almacenar estes repositorios nun lugar publicamente
accesible.