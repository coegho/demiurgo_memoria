\chapter{Manual técnico: API de Demiurgo}
\renewcommand\theadfont{\bfseries}
\renewcommand\cellalign{ll}

Os servizos web tipo REST son o xeito que teñen o DemiurgoWeb e outras posibles
interfaces de comunicarse co Demiurgo. A comunicación realízase mediante
conexións HTTP mediante GET ou POST, dependendo do método, que envían e reciben
obxectos JSON.
\par
Neste manual especificaranse os distintos tipos de datos e os métodos
dispoñibles para que calquera desenvolvedor poida implementar novas interfaces.

\section{Tipos de datos}
Os tipos de datos definen a estrutura JSON dos datos que se envían e reciben a
través dos métodos. Para simplificar a lectura clasificaremos estes tipos en
dúas categorías: tipos básicos e tipos complexos.

\subsection{Tipos básicos}
Os tipos básicos son os obxectos JSON que fan referencia a elementos
fundamentais do sistema, tales como obxectos, escenarios ou usuarios. Na maioría
dos casos estes obxectos correspóndense en gran medida co modelo de datos
empregado internamente polo Demiurgo.

\subsubsection{Action}
Obxecto que se corresponde cunha acción determinada. Contén toda a información
relativa á acción, polo que non se envía nunca aos xogadores, só aos DX.
\begin{tabular} { | l | l | l | l | }
\hline
\thead{Campo} & \thead{Tipo} & \thead{Opcional} & \thead{Descrición} \\
\hline
id & Enteiro & non & Identificador único da acción. \\
\hline
narration & Texto & non & Narración textual da acción. \\
\hline
room & Texto & non & \makecell{Nome do escenario no que se \\ desenvolve a
acción.}
\\
\hline
witnesses & Lista: Witness & non & \makecell{Lista das testemuñas da
acción; \\ é dicir, os usuarios que teñen \\ acceso a esta acción.} \\
\hline
date & Texto & non & Data na que se creou a acción. \\
\hline
published & Booleano & non & \makecell{Estado da acción. Verdadeiro \\ no caso
de tratarse dunha \\ acción publicada; falso se aínda \\ non foi publicada. As
accións \\ só se publican cando se redacta \\ a súa correspondente narración.}
\\
\hline
\end{tabular}

\subsubsection{Class}
Correspóndese cunha clase do sistema de xogo, a partir da cal poden herdar
os obxectos.

\begin{tabular} { | l | l | l | l | }
\hline
\thead{Campo} & \thead{Tipo} & \thead{Opcional} & \thead{Descrición} \\
\hline
classname & Texto & non & Nome da clase. \\
\hline
parent & Class & si & Clase pai da que herda. \\
\hline
fields & Lista: Variable & non & Lista de campos da clase. \\
\hline
methods & Lista: Method & non & Lista de métodos. \\
\hline
code & Texto & non & Código COE da clase. \\
\hline
\end{tabular}

\subsubsection{Inventory}
Correspóndese cun inventario dun obxecto. Esta clase non fai referencia ao campo
de tipo \textit{inventario} do obxecto, senón ao inventario en si e os obxectos
que contén.

\begin{tabular} { | l | l | l | l | }
\hline
\thead{Campo} & \thead{Tipo} & \thead{Opcional} & \thead{Descrición} \\
\hline
name & Texto & non & \makecell{Nome do inventario, que se \\ corresponde co nome
do campo.}
\\
\hline
objects & Lista: Object & non & \makecell{Lista de obxectos que contén o \\
inventario.}
\\
\hline
\end{tabular}

\subsubsection{Method}
Método asociado a unha clase. Contén o nome e o nome dos argumentos.

\begin{tabular} { | l | l | l | l | }
\hline
\thead{Campo} & \thead{Tipo} & \thead{Opcional} & \thead{Descrición} \\
\hline
name & Texto & non & Nome do método. \\
\hline
args & Lista: Texto & non & \makecell{Lista de argumentos de entrada \\ do
método.}
\\
\hline
returnArg & Texto & non & Nome do argumento de saída. \\
\hline
\end{tabular}

\subsubsection{Object}
Contén a información dun obxecto do mundo de xogo.

\begin{tabular} { | l | l | l | l | }
\hline
\thead{Campo} & \thead{Tipo} & \thead{Opcional} & \thead{Descrición} \\
\hline
id & Enteiro & non & \makecell{Identificador único do \\ obxecto.} \\
\hline
classname & Texto & non & Nome da clase do obxecto. \\
\hline
locationId & Enteiro & non & Identificador da localización. \\
\hline
fields & Lista: Variable & non & Lista de campos do obxecto. \\
\hline
methods & Lista: Method & non & Lista de métodos da clase. \\
\hline
inventories & Lista: Inventory & non & \makecell{Lista de inventarios do \\
obxecto.} \\
\hline
\end{tabular}

\subsubsection{Room}
Correspóndese cun escenario do mundo de xogo. Contén as variables do escenario e
tamén a lista de obxectos que se atopan nel.

\begin{tabular} { | l | l | l | l | }
\hline
\thead{Campo} & \thead{Tipo} & \thead{Opcional} & \thead{Descrición} \\
\hline
id & Enteiro & non & \makecell{Identificador único da \\ localización.} \\
\hline
longPath & Texto & non & Nome completo do escenario. \\
\hline
variables & Lista: Variable & non & \makecell{Lista de variables de \\
escenario.}
\\
\hline
objects & Lista: Object & non & \makecell{Lista de obxectos contidos no \\
escenario.}
\\
\hline
users & Lista: User & non & \makecell{Lista de usuarios con \\ personaxes no
escenario.}
\\
\hline
prenarration & Texto & si & \makecell{Prenarración do escenario, \\ xerada
mediante os \\ comandos ``echo'' \\ invocados mediante o \\ código.}
\\
\hline
\end{tabular}

\subsubsection{Status Inventory}
Obxecto especial. Fai referencia a un inventario tal e como é percibido por un
usuario. Non contén todos os datos que contén un obxecto Inventory.

\begin{tabular} { | l | l | l | l | }
\hline
\thead{Campo} & \thead{Tipo} & \thead{Opcional} & \thead{Descrición} \\
\hline
name & Texto & non & Nome do inventario. \\
\hline
objects & Lista: Object & non &  Lista de obxectos no inventario.\\
\hline
\end{tabular}

\subsubsection{Status Object}
Obxecto especial. Fai referencia a un obxecto do mundo de xogo tal e como é
percibido por un usuario. Non contén todos os datos que contén un obxecto
Object.

\begin{tabular} { | l | l | l | l | }
\hline
\thead{Campo} & \thead{Tipo} & \thead{Opcional} & \thead{Descrición} \\
\hline
name & Texto & non & \makecell{Nome do obxecto, definido \\ polo campo
``v\_name''.
No \\ caso de non ter tal campo, \\ o nome será o nome da clase.}
\\
\hline
description & Texto & si & \makecell{Descrición do obxecto, definida \\ polo
campo ``v\_description''.}
\\
\hline
imgUrl & Texto & si & \makecell{Imaxe do obxecto, definida \\ polo campo
``v\_imgurl''.} \\
\hline
fields & Lista: Variable & non & \makecell{Lista de campos do obxecto \\
visibles para o usuario.}
\\
\hline
inventories & \makecell{Lista: Status \\ Inventory} & non & \makecell{Lista de
inventarios do obxecto \\
visibles para o usuario.} \\
\hline
\end{tabular}

\subsubsection{User}
Contén os datos asociados a un usuario.

\begin{tabular} { | l | l | l | l | }
\hline
\thead{Campo} & \thead{Tipo} & \thead{Opcional} & \thead{Descrición} \\
\hline
name & Texto & non & Nome do usuario. \\
obj & Object & si & Personaxe do usuario, en caso de telo. \\
decision & Texto & si & \makecell{Decisión do usuario, en caso de ter \\ escrito
unha.}
\\
role & Texto & non & \makecell{Rol do usuario. Dous roles \\ dispoñibles:
``gm'' e ``user''.}
\\
\hline
\end{tabular}

\subsubsection{Variable}
Este obxecto contén información sobre unha variable de escenario, un campo dun
obxecto ou un campo definido nunha clase. Contén o nome da variable, o seu valor
en forma de cadea de texto, e o seu tipo.

\begin{tabular} { | l | l | l | l | }
\hline
\thead{Campo} & \thead{Tipo} & \thead{Opcional} & \thead{Descrición} \\
\hline
name & Texto & non & Nome da variable ou campo. \\
\hline
value & Texto & non & Valor textual da variable ou campo. \\
\hline
type & Texto & non & \makecell{Tipo da variable ou campo. Os tipos \\
dispoñibles pódense atopar no manual \\ da linguaxe COE
(apéndice \ref{ch:manual}).}
\\
\hline
\end{tabular}


\subsubsection{Witness}
Obxecto auxiliar que serve para identificar ás testemuñas dunha acción. As
testemuñas son os usuarios involucrados na acción, é dicir, que teñen personaxes
no escenario no momento de suceder a acción.

\begin{tabular} { | l | l | l | l | }
\hline
\thead{Campo} & \thead{Tipo} & \thead{Opcional} & \thead{Descrición} \\
\hline
user & Texto & non & Nome do usuario. \\
\hline
decision & Texto & si & Decisión do usuario en caso de tela. \\
\hline
\end{tabular}


\subsection{Tipos complexos}
Os tipos complexos non se corresponden con elementos do modelo de datos; no seu
lugar, serven para agrupar diferentes tipos básicos. A maioría de métodos que
devolven un obxecto JSON como resposta devolven un tipo complexo.

\subsubsection{AllClasses Response}
Resposta ao método \textit{allclasses}. Contén unha árbore de clases
estruturadas segundo a herdanza.

\begin{tabular} { | l | l | l | l | }
\hline
\thead{Campo} & \thead{Tipo} & \thead{Opcional} & \thead{Descrición} \\
\hline
classes & ClassTree & non & \makecell{Clase raíz ``object" que contén ao
\\ resto.}
\\
\hline
\end{tabular}

\subsubsection{AllRoomPaths Response}
Resposta ao método \textit{roompaths}. Contén unha lista con todos os
nomes de escenarios.

\begin{tabular} { | l | l | l | l | }
\hline
\thead{Campo} & \thead{Tipo} & \thead{Opcional} & \thead{Descrición} \\
\hline
status & \makecell{Status \\ Response} & non & Resposta co estado da operación.
\\
\hline
paths & Lista: Texto & non & Lista de nomes de escenario. \\
\hline
\end{tabular}

\subsubsection{ChangePassword Request}
Datos para a petición de cambio de contrasinal mediante o método
\textit{changepasswd}.

\begin{tabular} { | l | l | l | l | }
\hline
\thead{Campo} & \thead{Tipo} & \thead{Opcional} & \thead{Descrición} \\
\hline
username & Texto & non & Nome do usuario. \\
\hline
oldPassword & Texto & non & Contrasinal anterior. \\
\hline
newPassword & Texto & non & Contrasinal novo. \\
\hline
\end{tabular}

\subsubsection{CheckRoom Response}
Resposta ao método \textit{checkroom}. Contén un obxecto de tipo \textit{Room}
e, no caso de haber algunha acción sen resolver, o obxecto \textit{Action}
correspondente.

\begin{tabular} { | l | l | l | l | }
\hline
\thead{Campo} & \thead{Tipo} & \thead{Opcional} & \thead{Descrición} \\
\hline
status & \makecell{Status \\ Response} & non & \makecell{Resposta co estado da
\\ operación.}
\\
\hline
room & Room & non & Escenario solicitado. \\
\hline
unpublishedAction & Action & si & \makecell{Acción sen resolver, no caso \\ de
que haxa unha.}
\\
\hline
\end{tabular}

\subsubsection{ClassTree}
Obxecto intermedio empregado para facer árbores de clases baseándose na súa
herdanza.

\begin{tabular} { | l | l | l | l | }
\hline
\thead{Campo} & \thead{Tipo} & \thead{Opcional} & \thead{Descrición} \\
\hline
className & Texto & non & \makecell{Nome da clase que se \\ corresponde con este
nodo.}
\\
\hline
children & Lista: ClassTree & si & \makecell{Lista de clases fillas.}
\\
\hline
\end{tabular}

\subsubsection{CreateClass Request}
Datos da petición de creación dunha clase do xogo.

\begin{tabular} { | l | l | l | l | }
\hline
\thead{Campo} & \thead{Tipo} & \thead{Opcional} & \thead{Descrición} \\
\hline
name & Texto & non & Nome da clase. \\
\hline
code & Texto & non & Código COE de definición da clase. \\
\hline
\end{tabular}

\subsubsection{CreateClass Response}
Resposta á petición de creación dunha clase do xogo.

\begin{tabular} { | l | l | l | l | }
\hline
\thead{Campo} & \thead{Tipo} & \thead{Opcional} & \thead{Descrición} \\
\hline
status & \makecell{Status \\ Response} & non & Resposta co estado da operación.
\\
\hline
createdClass & Class & non & Devolve a clase creada. \\
\hline
\end{tabular}

\subsubsection{CreateRoom Request}
Datos da petición de creación dun escenario. É un obxecto moi sinxelo que só
contén o nome do escenario.

\begin{tabular} { | l | l | l | l | }
\hline
\thead{Campo} & \thead{Tipo} & \thead{Opcional} & \thead{Descrición} \\
\hline
path & Texto & non & \makecell{Nome completo do escenario.} \\
\hline

\hline
\end{tabular}

\subsubsection{DeleteVariable Request}
Datos da petición de borrado dunha variable de escenario.

\begin{tabular} { | l | l | l | l | }
\hline
\thead{Campo} & \thead{Tipo} & \thead{Opcional} & \thead{Descrición} \\
\hline
path & Texto & non & \makecell{Nome do escenario que contén a variable.}
\\
\hline
varName & Texto & non & \makecell{Nome da variable.}
\\
\hline
\end{tabular}

\subsubsection{DeleteVariable Response}
Resposta á petición de borrado dunha variable de escenario.

\begin{tabular} { | l | l | l | l | }
\hline
\thead{Campo} & \thead{Tipo} & \thead{Opcional} & \thead{Descrición} \\
\hline
status & \makecell{Status \\ Response} & non & Resposta co estado da operación.
\\
\hline
\end{tabular}

\subsubsection{DestroyClass Request}
Datos da petición de destrución dunha clase do xogo.

\begin{tabular} { | l | l | l | l | }
\hline
\thead{Campo} & \thead{Tipo} & \thead{Opcional} & \thead{Descrición} \\
\hline
className & Texto & non & \makecell{Nome da clase a borrar.}
\\
\hline
\end{tabular}

\subsubsection{DestroyClass Response}
Resposta á petición de destrución dunha clase do xogo.

\begin{tabular} { | l | l | l | l | }
\hline
\thead{Campo} & \thead{Tipo} & \thead{Opcional} & \thead{Descrición} \\
\hline
status & \makecell{Status \\ Response} & non & Resposta co estado da operación.
\\
\hline
\end{tabular}

\subsubsection{DestroyObject Request}
Datos da petición de destrución dun obxecto do mundo.

\begin{tabular} { | l | l | l | l | }
\hline
\thead{Campo} & \thead{Tipo} & \thead{Opcional} & \thead{Descrición} \\
\hline
objId & Enteiro & non & \makecell{Identificador do obxecto a \\ borrar.}
\\
\hline
destroyContents & Booleano & non & \makecell{Se é verdadeiro, destruiranse \\
todos os obxectos contidos \\ nos inventarios do obxecto. \\ Se é falso, serán
desprazados \\ ao escenario no que se atopa \\  o obxecto en cuestión.}
\\
\hline
\end{tabular}

\subsubsection{DestroyObject Response}
Resposta á petición de destrución dun obxecto do mundo.

\begin{tabular} { | l | l | l | l | }
\hline
\thead{Campo} & \thead{Tipo} & \thead{Opcional} & \thead{Descrición} \\
\hline
status & \makecell{Status \\ Response} & non & Resposta co estado da operación.
\\
\hline
\end{tabular}

\subsubsection{DestroyRoom Request}
Datos da petición de destrución dun escenario.

\begin{tabular} { | l | l | l | l | }
\hline
\thead{Campo} & \thead{Tipo} & \thead{Opcional} & \thead{Descrición} \\
\hline
path & Texto & non & \makecell{Nome completo do escenario.}
\\
\hline
\end{tabular}

\subsubsection{DestroyRoom Response}
Resposta á petición de destrución dun escenario.

\begin{tabular} { | l | l | l | l | }
\hline
\thead{Campo} & \thead{Tipo} & \thead{Opcional} & \thead{Descrición} \\
\hline
status & \makecell{Status \\ Response} & non & Resposta co estado da operación.
\\
\hline
\end{tabular}

\subsubsection{ExecuteCode Request}
Datos da petición de execución de código.

\begin{tabular} { | l | l | l | l | }
\hline
\thead{Campo} & \thead{Tipo} & \thead{Opcional} & \thead{Descrición} \\
\hline
path & Texto & non & \makecell{Nome do escenario no que se \\ executará o
código.}
\\
\hline
code & Texto & non & \makecell{Código COE a executar.}
\\
\hline
createAction & Booleano & non & \makecell{Se é verdadeiro, crearase unha \\ nova
acción sen publicar e \\\ solicitarase unha narración ao \\ DX. Se é falso,
non se crearán \\ accións.}
\\
\hline
\end{tabular}

\subsubsection{ExecuteCode Response}
Resposta á petición de execución de código.

\begin{tabular} { | l | l | l | l | }
\hline
\thead{Campo} & \thead{Tipo} & \thead{Opcional} & \thead{Descrición} \\
\hline
status & \makecell{Status \\ Response} & non & Resposta co estado da operación.
\\
\hline
action & Action & si & Acción creada no caso de existir unha. \\
\hline
\end{tabular}

\subsubsection{GetPendingRooms Response}
Resposta á petición de escenarios activos.

\begin{tabular} { | l | l | l | l | }
\hline
\thead{Campo} & \thead{Tipo} & \thead{Opcional} & \thead{Descrición} \\
\hline
status & \makecell{Status \\ Response} & non & \makecell{Resposta co estado da
\\ operación.}
\\
\hline
pendingRooms & \makecell{Lista: Pending \\ Room} & non & Lista de escenarios
activos.
\\
\hline
numUsers & Enteiro & non & \makecell{Número de usuarios en \\ xogo.} \\
\hline
noObjUsers & Lista: User & non & \makecell{Lista de usuarios sen un \\ personaxe
asignado.}
\\
\hline
\end{tabular}

\subsubsection{Login Request}
Datos da petición de \textit{login}.

\begin{tabular} { | l | l | l | l | }
\hline
\thead{Campo} & \thead{Tipo} & \thead{Opcional} & \thead{Descrición} \\
\hline
name & Texto & non & \makecell{Nome de usuario.}
\\
\hline
password & Texto & non & \makecell{Contrasinal do usuario.}
\\
\hline
world & Texto & non & Nome do mundo. \\
\hline
\end{tabular}

\subsubsection{Me Response}
Resposta á petición de datos do usuario.

\begin{tabular} { | l | l | l | l | }
\hline
\thead{Campo} & \thead{Tipo} & \thead{Opcional} & \thead{Descrición} \\
\hline
status & \makecell{Status \\ Response} & non & Resposta co estado da operación.
\\
\hline
user & User & non & Datos do usuario. \\
\hline
obj & \makecell{Status \\ Object} & si & Personaxe do usuario. \\
\hline
world & Texto & non & Mundo ao que pertence o usuario. \\
\hline
\end{tabular}

\subsubsection{NarrateAction Request}
Datos da petición de narración dunha acción.

\begin{tabular} { | l | l | l | l | }
\hline
\thead{Campo} & \thead{Tipo} & \thead{Opcional} & \thead{Descrición} \\
\hline
id & Enteiro & non & \makecell{Identificador da acción.}
\\
\hline
narration & Texto & non & \makecell{Narración da acción.}
\\
\hline
\end{tabular}

\subsubsection{NarrateAction Response}
Resposta á petición de narración dunha acción.

\begin{tabular} { | l | l | l | l | }
\hline
\thead{Campo} & \thead{Tipo} & \thead{Opcional} & \thead{Descrición} \\
\hline
status & \makecell{Status \\ Response} & non & Resposta co estado da operación.
\\
\hline
action & Action & non & Acción modificada. \\
\hline
\end{tabular}

\subsubsection{PastActions Response}
Resposta á petición de accións pasadas do xogador.

\begin{tabular} { | l | l | l | l | }
\hline
\thead{Campo} & \thead{Tipo} & \thead{Opcional} & \thead{Descrición} \\
\hline
status & \makecell{Status \\ Response} & non & \makecell{Resposta co estado da
\\ operación.}
\\
\hline
totalActions & Enteiro & non & \makecell{Total de accións visibles \\ para o
xogador.}
\\
\hline
actions & Lista: Texto & non & \makecell{Lista co texto das accións \\
solicitadas.}
\\
\hline
\end{tabular}

\subsubsection{Pending Room}
Obxecto que contén os datos relativos a un escenario activo.

\begin{tabular} { | l | l | l | l | }
\hline
\thead{Campo} & \thead{Tipo} & \thead{Opcional} & \thead{Descrición} \\
\hline
path & Texto & non & Nome completo do escenario. \\
\hline
numObjects & Enteiro & non & \makecell{Total de obxectos no \\ escenario.} \\
\hline
numUsers & Enteiro & non & \makecell{Total de xogadores no \\ escenario.} \\
\hline
decidingUsers & Lista: Texto & non & \makecell{Nome dos xogadores que \\
mandaron decisións.}
\\
\hline
\end{tabular}

\subsubsection{RegisterUser Request}
Datos da petición de rexistro dun usuario.

\begin{tabular} { | l | l | l | l | }
\hline
\thead{Campo} & \thead{Tipo} & \thead{Opcional} & \thead{Descrición} \\
\hline
world & Texto & non & \makecell{Nome do mundo onde se rexistra o \\ usuario.}
\\
\hline
username & Texto & non & \makecell{Nome do usuario.}
\\
\hline
password & Texto & non & Contrasinal do usuario. \\
\hline
mail & Texto & non & \makecell{Enderezo de correo electrónico do \\ usuario.} \\
\hline
\end{tabular}

\subsubsection{RegisterUser Response}
Resposta á petición de rexistro dun usuario.

\begin{tabular} { | l | l | l | l | }
\hline
\thead{Campo} & \thead{Tipo} & \thead{Opcional} & \thead{Descrición} \\
\hline
status & \makecell{Status \\ Response} & non & Resposta co estado da operación.
\\
\hline
active & Booleano & non & \makecell{Se é verdadeiro, a conta está activa e \\ o
xogador pode comezar a xogar \\ inmediatamente.}
\\
\hline
\end{tabular}

\subsubsection{Response Status}
Obxecto auxiliar que acompaña ás respostas indicando se a operación se executou
correctamente ou houbo algún problema.

\begin{tabular} { | l | l | l | l | }
\hline
\thead{Campo} & \thead{Tipo} & \thead{Opcional} & \thead{Descrición} \\
\hline
ok & Booleano & non & \makecell{Se é verdadeiro, a operación \\ executouse
correctamente. Se é \\ falso, houbo algún problema.}
\\
\hline
description & Texto & si & \makecell{No caso de haber algún problema, \\ devolve
unha descrición do mesmo.}
\\
\hline
\end{tabular}

\subsubsection{RoomHistory Response}
Resposta á petición de historial dun escenario.

\begin{tabular} { | l | l | l | l | }
\hline
\thead{Campo} & \thead{Tipo} & \thead{Opcional} & \thead{Descrición} \\
\hline
status & \makecell{Status \\ Response} & non & \makecell{Resposta co estado da
\\ operación.}
\\
\hline
actions & Lista: Action & non & Lista coas accións devoltas. \\
\hline
totalActions & Enteiro & non & Total de accións do escenario. \\
\hline
\end{tabular}

\subsubsection{SubmitDecision Request}
Datos da petición de envío dunha decisión.

\begin{tabular} { | l | l | l | l | }
\hline
\thead{Campo} & \thead{Tipo} & \thead{Opcional} & \thead{Descrición} \\
\hline
decision & Texto & non & \makecell{Texto da decisión do usuario.}
\\
\hline
\end{tabular}

\subsubsection{WorldList Response}
Resposta á petición da lista de mundos.

\begin{tabular} { | l | l | l | l | }
\hline
\thead{Campo} & \thead{Tipo} & \thead{Opcional} & \thead{Descrición} \\
\hline
worlds & Lista: Texto & non & \makecell{Lista cos nomes dos mundos \\
dispoñibles.}
\\
\hline
\end{tabular}


\section{Métodos dispoñibles}
Nesta sección especificaremos os métodos que ofrece a API. Por cada método
indicarase, ademais dunha breve descrición: se é unha petición HTTP por método
GET ou POST, o obxecto JSON que devolve, e os parámetros que require. Por cada
parámetro especificarase o nome no caso de ir por GET; nos métodos POST deberán
ir no corpo da petición.

\subsection{Métodos de autentificación}
Estes métodos chámanse sen estar identificado polo sistema. Empréganse por tanto
para xestionar a autentificación e rexistro de usuarios.

\subsubsection{login}
Método POST. Identifica o usuario co seu nome e contrasinal no mundo
especificado. Devolve unha cadea de texto coñecida como \textit{token}, que
deberá ser especificada nos distintos métodos que requiran autentificación. O
token inclúese dentro da cabeceira HTTP no campo \textit{Authorization} da
seguinte forma:
\begin{lstlisting}
Authorization: Bearer <token>
\end{lstlisting}

\begin{tabular} {| l | l | l | l |}
\hline
\thead{Parámetro} & \thead{Tipo} & \thead{Opcional} & \thead{Descrición} \\
\hline
(POST) & Login Request & non & Datos de usuario. \\
\hline
\end{tabular}

\subsubsection{register}
Método POST. Rexistra un novo usuario cos datos proporcionados. Devolve un
obxecto \textit{RegisterUser Response}.

\begin{tabular} {| l | l | l | l |}
\hline
\thead{Parámetro} & \thead{Tipo} & \thead{Opcional} & \thead{Descrición} \\
\hline
(POST) & RegisterUser Request & non & Datos de usuario. \\
\hline
\end{tabular}

\subsection{Métodos xenéricos}
Os métodos xenéricos poden ser empregado sen importar o rol do usuario. A súa
función está vinculada principalmente aos propios datos do usuario, e non a
elementos funcionais do xogo.

\subsubsection{changepasswd}
Método POST. Emprégase para cambiar o contrasinal do usuario. Devolve un obxecto
\textit{Response Status}.

\begin{tabular} {| l | l | l | l |}
\hline
\thead{Parámetro} & \thead{Tipo} & \thead{Opcional} & \thead{Descrición} \\
\hline
(POST) & \makecell{ChangePassword \\ Request} & non & \makecell{Contén o
contrasinal vello e \\ o novo.}
\\
\hline
\end{tabular}

\subsubsection{me}
Método GET. Útil para recibir información do usuario e, no caso de tratarse dun
xogador, obter un estado do personaxe. Devolve un obxecto \textit{Me
Response}.

O método \textit{me} non ten parámetros.


\subsection{Métodos para xogadores}
Os xogadores contan só con dous métodos: un para enviar novas decisións, e outro
para ler accións. Isto débese ao feito de que os usuarios nunca perciben
directamente o mundo de xogo, só acceden á información que o DX lles ofrece.

\subsubsection{pastactions}
Método GET. Devolve as narracións das últimas accións. É posible especificar un
rango de accións concreto. As narracións obtidas con este método estarán
filtradas no caso de conter fragmentos ocultos para o xogador. Devolve un
obxecto \textit{PastActions Response}.

\begin{tabular} {| l | l | l | l |}
\hline
\thead{Parámetro} & \thead{Tipo} & \thead{Opcional} & \thead{Descrición} \\
\hline
first & Texto & si & \makecell{Índice da primeira acción que se \\ desexa 
recibir. Debe ser convertible a \\ un número enteiro. Se o valor é \\ negativo,
calcularase a primeira acción \\ restando o valor dende a última acción \\
dispoñible. O valor por defecto é -10.}
\\
\hline
count & Texto & si & \makecell{Número de accións que se queren \\ recibir dende
a primeira. Debe ser \\ convertible  un número enteiro. Por \\ defecto este
valor é 10.}
\\
\hline
\end{tabular}

\subsubsection{submitdecision}
Método POST. Envía unha decisión redactada polo usuario. Non devolve ningún
obxecto.

\begin{tabular} {| l | l | l | l |}
\hline
\thead{Parámetro} & \thead{Tipo} & \thead{Opcional} & \thead{Descrición} \\
\hline
(POST) & \makecell{SubmitDecision \\ Request} & non & \makecell{Contén a
decisión do \\ usuario.}
\\
\hline
\end{tabular}


\subsection{Métodos para o DX}
Os métodos do DX son os que xestionan a partida, ofrecendo ao DX unha canle para
executar código e recibir información do mundo de xogo. Ademais disto, tamén
ofrece métodos útiles para realizar modificacións no mundo directamente.

\subsubsection{action}
Método GET. Emprégase para recibir a información dunha acción concreta. Devolve
un obxecto \textit{Action}.

\begin{tabular} {| l | l | l | l |}
\hline
\thead{Parámetro} & \thead{Tipo} & \thead{Opcional} & \thead{Descrición} \\
\hline
id & Enteiro & non & Identificador único da acción. \\
\hline
\end{tabular}

\subsubsection{allclasses}
Método GET. Emprégase para obter unha árbore de todas as clases no mundo de
xogo. As clases están organizadas segundo a herdanza, comezando pola clase que
fai de raíz: a clase \textit{Object}. As clases que herdan directamente de
\textit{Object} aparecen como fillas deste primeiro nodo, e así sucesivamente.
Devolve un obxecto \textit{AllClasses Response}.

O método \textit{allclasses} non ten parámetros.

\subsubsection{checkroom}
Método GET. Serve para comprobar o estado dun escenario. Ademais do escenario
en si, se existe algunha acción sen publicar, este método serve para chegar a
ela. Devolve un obxecto \textit{CheckRoom Response}.

\begin{tabular} {| l | l | l | l |}
\hline
\thead{Parámetro} & \thead{Tipo} & \thead{Opcional} & \thead{Descrición} \\
\hline
path & Texto & non & Nome completo do escenario. \\
\hline
\end{tabular}

\subsubsection{createclass}
Método POST. Método empregado para crear clases novas. Devolve un obxecto
\textit{CreateClass Response}.

\begin{tabular} {| l | l | l | l |}
\hline
\thead{Parámetro} & \thead{Tipo} & \thead{Opcional} & \thead{Descrición} \\
\hline
(POST) & \makecell{CreateClass \\ Request} & non & Nome e código da clase. \\
\hline
\end{tabular}

\subsubsection{createroom}
Método POST. Método empregado para crear novos escenarios. Devolve un obxecto
\textit{Room} correspondente ao escenario creado.

\begin{tabular} {| l | l | l | l |}
\hline
\thead{Parámetro} & \thead{Tipo} & \thead{Opcional} & \thead{Descrición} \\
\hline
(POST) & \makecell{CreateRoom \\ Request} & non & Especifica o escenario a
borrar.
\\
\hline
\end{tabular}

\subsubsection{delvar}
Método POST. Indica unha variable de escenario para que sexa borrada. Devolve un
obxecto \textit{DeleteVariable Response}.

\begin{tabular} {| l | l | l | l |}
\hline
\thead{Parámetro} & \thead{Tipo} & \thead{Opcional} & \thead{Descrición} \\
\hline
(POST) & \makecell{DeleteVariable \\ Request} & non & Especifica a variable a
borrar.
\\
\hline
\end{tabular}

\subsubsection{destroyclass}
Método POST. Destrúe unha clase do xogo. Devolve un obxecto \textit{DestroyClass
Response}.

\begin{tabular} {| l | l | l | l |}
\hline
\thead{Parámetro} & \thead{Tipo} & \thead{Opcional} & \thead{Descrición} \\
\hline
(POST) & \makecell{DestroyClass \\ Request} & non & Especifica a clase a borrar.
\\
\hline
\end{tabular}

\subsubsection{destroyobj}
Método POST. Destrúe un obxecto do mundo de xogo. Devolve un obxecto
\textit{DestroyObject Response}.

\begin{tabular} {| l | l | l | l |}
\hline
\thead{Parámetro} & \thead{Tipo} & \thead{Opcional} & \thead{Descrición} \\
\hline
(POST) & \makecell{DestroyObject \\ Request} & non & Especifica o obxecto a
borrar.
\\
\hline
\end{tabular}

\subsubsection{destroyroom}
Método POST. Destrúe un escenario. Devolve un obxecto \textit{DestroyRoom
Response}.

\begin{tabular} {| l | l | l | l |}
\hline
\thead{Parámetro} & \thead{Tipo} & \thead{Opcional} & \thead{Descrición} \\
\hline
(POST) & \makecell{DestroyRoom \\ Request} & non & \makecell{Especifica o
escenario a \\ borrar.}
\\
\hline
\end{tabular}

\subsubsection{executecode}
Método POST. Executa código COE nun escenario, podendo xerarse unha acción ou
non segundo se especifique. Devolve un obxecto \textit{ExecuteCode Response}.

\begin{tabular} {| l | l | l | l |}
\hline
\thead{Parámetro} & \thead{Tipo} & \thead{Opcional} & \thead{Descrición} \\
\hline
(POST) & \makecell{ExecuteCode \\ Request} & non & \makecell{Contén o código e
outros \\ datos adicionais.}
\\
\hline
\end{tabular}

\subsubsection{getclass}
Método GET. Método empregado para obter unha clase do xogo. Devolve un obxecto
\textit{Class}.

\begin{tabular} {| l | l | l | l |}
\hline
\thead{Parámetro} & \thead{Tipo} & \thead{Opcional} & \thead{Descrición} \\
\hline
classname & Texto & non & Nome da clase. \\
\hline
\end{tabular}

\subsubsection{history}
Método GET. Permite acceder ao historial dun escenario. Ten un funcionamento
semellante ao método \textit{pastactions}. Devolve un obxecto
\textit{RoomHistory Response}.

\begin{tabular} {| l | l | l | l |}
\hline
\thead{Parámetro} & \thead{Tipo} & \thead{Opcional} & \thead{Descrición} \\
\hline
path & Texto & non & Nome completo do escenario. \\
\hline
first & Texto & si & \makecell{Índice da primeira acción que se \\ desexa 
recibir. Debe ser convertible a \\ un número enteiro. Se o valor é \\ negativo,
calcularase a primeira acción \\ restando o valor dende a última acción \\
dispoñible. O valor por defecto é -10.}
\\
\hline
count & Texto & si & \makecell{Número de accións que se queren \\ recibir dende
a primeira. Debe ser \\ convertible  un número enteiro. Por \\ defecto este
valor é 10.} \\
\hline
\end{tabular}

\subsubsection{modifyclass}
Método POST. Serve para modificar unha clase xa existente. Devolve un obxecto
\textit{CreateClass Response}, e ten un funcionamento semellante ao método
\textit{createclass}.

\begin{tabular} {| l | l | l | l |}
\hline
\thead{Parámetro} & \thead{Tipo} & \thead{Opcional} & \thead{Descrición} \\
\hline
(POST) & \makecell{CreateClass \\ Request} & non & \makecell{Contén o novo
código COE da \\ clase.}
\\
\hline
\end{tabular}

\subsubsection{narrateaction}
Método POST. Engade a narración a unha acción concreta. No caso de xa contar
cunha narración, sobreescríbea. Devolve un obxecto \textit{NarrateAction
Response}.

\begin{tabular} {| l | l | l | l |}
\hline
\thead{Parámetro} & \thead{Tipo} & \thead{Opcional} & \thead{Descrición} \\
\hline
(POST) & \makecell{NarrateAction \\ Request} & non & \makecell{Contén a
narración da acción.}
\\
\hline
\end{tabular}

\subsubsection{pendingrooms}
Método GET. Método empregado para obter infomación sobre os escenarios activos
e o estado dos xogadores. Devolve un obxecto \textit{PendingRooms Response}.

O método \textit{pendingrooms} non ten parámetros.

\subsubsection{roompaths}
Método GET. Ofrece unha lista de nomes de escenarios no mundo actual. Devolve
un obxecto \textit{AllRoomPaths Response}.

O método \textit{roompaths} non ten parámetros.