\chapter{Manual de usuario}
\label{ch:manual}

\section{Introdución}
A linguaxe COE (acrónimo de Código de Obxectos e Escenarios) é a linguaxe
empregada nas partidas no Demiurgo, o software de xestión de
partidas de rol asíncronas. O director de xogo ({\it DX}) pode escribir código
en COE para definir clases no mundo de xogo, e para realizar accións nos escenarios nas que
os distintos obxectos interactúan entre si.
\par
A finalidade deste manual é definir esta linguaxe e ensinar a darlle uso para
poder empregar todas as funcionalidades do Demiurgo sen dificultade, ofrecendo
para isto definicións formais, explicacións e exemplos en cada apartado. Grazas
a este sistema, o DX pode manter un mundo loxicamente coherente que lle faga de
soporte á hora de levar as partidas, evitando así erros e automatizando en gran
medida o apartado técnico das mesmas.




\section{Escenarios, clases e obxectos}
Un mundo de xogo de Demiurgo está composto por obxectos, cada un dos cales
segue a definición dunha clase e atópase dentro dun escenario. A medida que
a partida avanza, os distintos obxectos créanse, modifícanse, móvense dun
escenario a outro e interactúan entre eles.
\subsection{Exemplo dunha situación}
Expresaremos unha situación normal do xogo mediante este exemplo:
\par
{\it A situación transcorre na taberna da capital do reino. Tres ananos están
nun recuncho discutindo entre eles acaloradamente. Detrás da barra pódese ver ao
taberneiro, un home corpulento que se distrae limpando un vaso. Detrás da barra
ten un mosquetón agochado por se as cousas se poñen feas, e por se acaso non
lle quita ollo aos ananos.}
\par
En base a esta descrición da situación, o director de xogo pode definir
internamente o seguinte:
\begin{itemize}
  \item Un escenario que se corresponde coa {\bf taberna}. O escenario debería
  ter un nome recoñecible para o director de xogo, como por exemplo {\it
  /reino/capital/taberna}, máis iso explicarase máis en detalle
  en \ref{subsec:nome-escenarios}.
  \item Como mínimo {\bf 5 obxectos}: \begin{itemize}
    \item Tres obxectos que representen aos {\bf ananos}. Xa que son tres
    obxectos moi semellantes, o normal é que pertenzan á mesma clase; por
    exemplo, unha clase chamada {\it Anano}. Máis isto non é unha norma,
    dependerá de como o director de xogo teña definido o mundo.
    \item Un obxecto que represente ao {\bf taberneiro}. Pode pertencer a unha
    clase diferenciada da dos ananos (como {\it Humano}); non obstante, ben poden
    pertencer tanto o taberneiro como os ananos a unha mesma clase {\it
    Intelixente} ou {\it Criatura}, por exemplo.
    \item Un obxecto que faga referencia á {\bf barra}, xa que o {\bf mosquetón}
    que contén pode resultar relevante no futuro. Por tanto, o normal sería  que
    houbese un obxecto que representase á barra (de clase {\it Mobiliario} por
    exemplo), e no seu interior se atopase un inventario que conteña ao obxecto
    mosquetón (de clase {\it Mosquetón}). Explicarase con detalle o concepto dos
    inventarios en \ref{subsec:inventarios}.
  \end{itemize}
\end{itemize}
\par
Esta configuración é un simple exemplo que non ten por que cumprirse ao
100\% nunha partida real, pero da unha idea do concepto. Poñamos agora outro
exemplo: na mesma situación, comezan a suceder cousas.
\par
{\it Parece que a discusión dos ananos chega a un punto crítico. Dous deles
deciden simultaneamente que non poderán resolver as súas diferencias mediante a
diplomacia, polo que deciden sacar os puños e golpearse mutuamente. O
taberneiro, que non lles quitaba ollo, saca o mosquetón como resposta por se
comeza unha escalada de violencia.}
\par
Isto pode suceder por dous motivos: os dous ananos son personaxes
controlados por xogadores e ambos tomaron a mesma decisión (golpear ao outro) ao
mesmo tempo; ou ben son personaxes non xogadores ({\it NPC}, {\it non player
character}) e foi o propio DX o que tomou a decisión por eles (para que outros
xogadores no escenario presenciaran isto, por exemplo).
\par
En calquera caso, o que lle corresponde ao DX é executar o código necesario para
cambiar o estado do mundo. O código a redactar debe facer tres cousas:
\begin{enumerate}
  \item Un dos ananos debe {\bf golpear} ao outro, cos cambios que iso implique
  (por exemplo, reducir o seu campo {\it vida}).
  \item O segundo anano debe {\bf golpear} ao primeiro.
  \item O mosquetón debe deixar de estar no inventario da barra e debe
  {\bf moverse} ao inventario do taberneiro.
\end{enumerate}
\par
Este sería un exemplo de como podería ser este código:
\begin{lstlisting}
anano1.golpear(anano2)
anano2.golpear(anano1)
barra.contidos[0] >> taberneiro.inventario
\end{lstlisting}
\par
Como entendemos este código? Temos unha variable anano1 que fai referencia ao
obxecto {\it Anano} que da o primeiro golpe, unha variable anano2 que fai
referencia ao segundo, unha variable {\it barra}, unha variable {\it
taberneiro}, e un método {\it golpear()} definido para a clase {\it Anano}. Este
código enténdese así:
\begin{enumerate}
  \item O obxecto correspondente á variable anano1 executa o {\bf método}
  {\it golpear()}, collendo o obxecto da variable anano2 como argumento. Este
  método, definido na clase {\it Anano}, reduce a vida do anano seleccionado
  como argumento. Explicarase máis en detalle isto na sección
  \ref{subsec:metodos}.
  \item O mesmo que no anterior, pero ao revés.
  \item O primeiro obxecto ({\it [0]}) dos contidos da barra (''contidos'' é un
  inventario da clase {\it Mobiliario}), que neste caso entendemos que é o
  mosquetón, {\bf móvese} (>>) ao inventario do taberneiro. 
\end{enumerate}

\subsection{Escenarios}
\subsubsection{Definición de escenario}
O mundo de xogo está dividido en escenarios (tamén chamados {\it habitacións}
polo seu homólogo {\it rooms} en inglés). Cada escenario está composto por:
\begin{itemize}
  \item O seu nome, sobre o cal falaremos na subsección
  \ref{subsec:nome-escenarios}.
  \item O conxunto de obxectos que se atopan ''dentro'' do escenario.
  \item As variables do escenario, as cales poden conter datos tales como
  números enteiros ou texto, pero tamén referencias a obxectos ou outros
  escenarios.
\end{itemize}
\par
O escenario forma unha unidade lóxica fundamental no mundo de xogo, e o habitual
é que as accións en cada escenario afecten só aos obxectos do propio escenario
(aínda que pode haber excepcións).

\subsubsection{Nome do escenario}
\label{subsec:nome-escenarios}
O nome de escenario serve para que o DX o identifique rapidamente. Ademais de
ofrecer un nome descritivo, o nome completo permite organizar os escenarios
nunha estrutura xerárquica, cunha lóxica semellante á das rutas de ficheiros nos
sistemas operativos.
\par
Exemplo:
\begin{lstlisting}
/continente/reinohumano/cidadecapital/armeria
/continente/reinohumano/cidadecapital/taberna
/continente/reinohumano/aldeacosteira
\end{lstlisting}
\par
Estes tres escenarios atópanse dentro de /continente/reinohumano, é dicir,
na estrutura lóxica do mundo de xogo están dentro do mesmo reino, o cal forma
parte do continente. A maiores, os dous primeiros están na mesma cidade. Estes
compoñentes normalmente reciben o nome de {\it rexións}, pero poden ter o
significado semántico que o DX de xogo prefira, por exemplo, nunha partida
futurista poderíamos atopar algo como:
\begin{lstlisting}
/planetamarte/coloniahumana/suburbios/casa3/cocinha
/planetamarte/orbita/sateliteflotante/laboratorio
/espazoexterior/zonadeasteroides
\end{lstlisting}
\par
Por último, o DX pode renunciar completamente a usar a xerarquía de escenarios,
aínda que iso pode dificultar o nomeado de escenarios:
\begin{lstlisting}
/chairas_da_capital_humana
/sala_do_gran_mago
\end{lstlisting}
\par
É importante ter en conta que o nome dos escenarios só pode estar composto por
caracteres alfanuméricos e barras baixas, e non pode conter letras que non
formen parte do alfabeto inglés (tales como {\it ñ} ou {\it ç}). O nome completo
sempre debe comezar por /. Por último, o Demiurgo non distingue entre
minúsculas e maiúsculas.
\par
Nota: {\it /} en si mesmo pode ser un escenario, o coñecido como {\it escenario
raíz}, pero non é habitual o seu uso.

\subsubsection{Contido do escenario}
O escenario pode conter obxectos (e de feito é habitual que os conteña). Estes
obxectos á súa vez poden dispoñer de inventarios (ver
subsección \ref{subsec:inventarios}) que conteñan outros obxectos. Mediante a
interface gráfica, o DX pode ver os distintos obxectos do escenario para operar
con eles no código.

\subsubsection{Variables}
O DX pode definir variables no escenario a través do código. Estas variables
están accesibles dende o propio escenario e poden ter todo tipo de utilidades.
Por exemplo, é habitual referenciar os obxectos do escenario en variables para
poder manexalos co código máis facilidade.
\par
Exemplo: Imaxinemos que no escenario temos un obxecto que representa a un
trasno. O obxecto ten un identificador propio ({\it ID}) que é \#1234 (ver
sección \ref{sec:obxectos}).
No momento de querer actuar, se non empregamos variables precisamos referirnos a
el co seu ID:
\begin{lstlisting}
#1234.facerTrasnadas()
if(#1234.famento == true)
  ! "O trasno ten bastante fame"
\end{lstlisting}
\par
En cambio, se temos unha variable cun nome descritivo (por exemplo, {\it
trasno}) podemos codificalo así:
\begin{lstlisting}
trasno.facerTrasnadas()
if(trasno.famento == true)
  ! "O trasno ten bastante fame"
\end{lstlisting}
\par
Outro uso frecuente das variables é gardar datos. Neste exemplo tírase un dado
de 20 caras e móstrase un texto ou outro segundo o resultado:
\begin{lstlisting}
int resultado = d20
if(resultado > 10) {
  ! "A accion tivo exito"
}
if(resultado < 3) {
  ! "A accion foi un fracaso absoluto"
}
\end{lstlisting}



\subsection{Clases}
\label{sec:clases}
As clases son definicións empregadas a modo de modelo para crear obxectos. Todos
os obxectos que teñan a mesma clase terán unha serie de métodos e campos
comúns definidos na propia clase.
\par
Nesta sección explicarase como funcionan as clases e todo o relacionado con
elas. No capítulo \ref{ch:definicion} explícase polo miúdo a sintaxe formal para
crear clases.
\subsubsection{Campos}
Os campos son variables vinculadas aos obxectos, e a súa finalidade é
almacenar datos relacionados con eles. Por exemplo, nun mundo de rol de fantasía
medieval, é habitual atopar nos personaxes campos como forza, axilidade ou
intelixencia. Nun mundo con máquinas de guerra, normalmente atoparemos campos
como a potencia de disparo ou o blindaxe. E tamén podemos atopar campos en
obxectos inanimados, como o peso, o prezo, a cor, etcéra.
\par
Os campos defínense coa clase, e cando é creado un obxecto desa clase,
automaticamente pasa a posuír todos os campos da mesma.
\par
Exemplo:
\begin{lstlisting}
elfo {
  int forza = 1
  int axilidade = 3
  int constitucion = 1
  int intelixencia = 2
  int vida
}
\end{lstlisting}
\par No exemplo vemos a definición dunha clase chamada {\it Elfo}. Como vemos,
os elfos teñen os campos forza, axilidade, constitución, intelixencia e vida,
os cales son números enteiros {\it int} e teñen asignados uns valores por
defecto (excepto vida).
Cando se crean obxectos da clase {\it Elfo} automaticamente xorden con eses
campos, pero o seu valor pode ser modificado posteriormente.

\subsubsection{Métodos}
\label{subsec:metodos}
Cada clase pode ter definidos unha serie de métodos. Os métodos son fragmentos
de código relacionados coa clase que poden ser invocados en calquera momento do
xogo, útiles para simplificar procesos repetitivos.
\par Por exemplo:
\begin{lstlisting}
elfo {
  bendicir(elfo outro) {
    outro.vida = outro.vida + 5
    ! "O elfo foi bendecido"
  }
}
\end{lstlisting}
\par No exemplo, a clase {\it Elfo} dispón dun método {\it bendicir()} que recibe
como argumento un obxecto {\it Elfo} (outro elfo ou el mesmo!) e aumenta a súa
vida en 5. Deste modo, o DX non precisa escribir ese código cada vez que queira
facer que un elfo bendiga a outro, e pode invocalo do seguinte xeito:
\begin{lstlisting}
elfo1.bendicir(elfo2)
\end{lstlisting}
\paragraph{Argumentos}
Vimos no exemplo do método {\it bendicir()} que o código fai referencia a unha
variable chamada {\it outro}. Isto é porque o ámbito dun método está reducido ao
obxecto que o invoca e aos argumentos que recibe; isto é, non pode acceder ás
variables do escenario.
\par
Cada método pode ter definidos un número indeterminado de argumentos de entrada,
e un argumento de saída (ou ningún). Todos estes argumentos deben especificar o
tipo, sexa un tipo primitivo como número enteiro ou texto, ou a clase se se
trata dun obxecto. Se a chamada do método se realiza con argumentos incorrectos
(por exemplo, enviando un obxecto dunha clase incorrecta, ou enviando un texto
cando require un número) o Demiurgo devolverá un erro.
\par
Dixemos que no ámbito dos métodos tamén está o obxecto que o invoca. Isto
faise mediante a variable {\it this}:
\begin{lstlisting}
elfo {
  ceder_vida(elfo outro) {
    outro.vida = outro.vida + 5
    this.vida = this.vida - 5
  }
}
\end{lstlisting}
\par
Os argumentos sempre teñen prioridade sobre os campos do obxecto en caso de que
os nomes coincidan; debido a isto, o uso de {\it this} é necesario para
referirse aos campos do obxecto cando isto suceda.

\paragraph{Construtor}
\label{subsubsec:construtor}
Por último, hai un método especial moi importante chamado \textbf{construtor},
que se executa automaticamente no momento de crear o obxecto. É útil por tanto
para inicializar clases. O construtor recoñécese polo feito de chamarse igual ca
clase. No caso de ter argumentos, é necesario especificalos na creación do obxecto:
\begin{lstlisting}
elfo {
  int idade
  str nome
  elfo(int novaidade, str novonome) {
    this.idade = novaidade
    this.nome = novonome
  }
}
\end{lstlisting}
\par E no momento de crealo:
\begin{lstlisting}
elfo legolas = new elfo(100, "Legolas o elfo")
\end{lstlisting}

\subsubsection{Herdanza de clases}
Un aspecto fundamental das clases é a herdanza das mesmas. Unha clase pode
especificarse como herdeira ou filla doutra clase, e deste modo herdará todos os
campos e métodos da clase pai. Esta característica, propia das linguaxes
orientadas a obxectos, permite aforrar moito traballo:
\begin{lstlisting}
animal {
  int altura
  str cor
  int peso
  str ruido
  
  facerRuido() {
  ! "este animal fai " + ruido
  }
}

can : animal {
  str tipoPelo
  str raza
  
  can() {
    ruido = "guau"
  }
}
\end{lstlisting}
\par
Outra utilidade da herdanza é que podemos gardar un obxecto da clase filla nunha
variable da clase pai. Por exemplo:
\begin{lstlisting}
personaxe {
  int vida = 10
}

guerreiro : personaxe {
  int ataque = 5
  int vida = 20
  atacar(personaxe outro) {
    outro.vida = outro.vida - ataque 
  }
}
\end{lstlisting}
\par Neste exemplo, o método {\it atacar()} pode recibir un personaxe como
argumento, ou calquera obxecto dunha clase filla (como outro guerreiro).

\paragraph{Clase Object}
Realmente todas as clases herdan dalgunha outra clase. Se a clase pai non é
definida, esta será automaticamente a clase {\it Object}, unha clase auxiliar
que non conta con campos. É posible crear obxectos da clase Object
directamente, pero realmente non aportan demasiado ao xogo xa que non conteñen
valor semántico ningún.

\subsection{Obxectos}
\label{sec:obxectos}
Os obxectos son o elemento central do xogo. Cada obxecto representa a unha
entidade no xogo, que pode ser física (como un cervo ou unha mesa) ou un
concepto (como un estado alterado ou unha idea). Todos os obxectos teñen as
seguintes características:
\begin{itemize}
  \item Identifícanse individualmente por un número enteiro coñecido como ID.
  \item Correspóndense cunha clase determinada.
  \item Atópanse nunha localización determinada, que pode ser un escenario, ou o
  inventario de outro obxecto.
\end{itemize}
\par
Os obxectos dispoñen dos campos definidos pola súa clase, e poden chamar
métodos da mesma. Estes aspectos explicáronse na sección ''Clases'' na sección
\ref{sec:clases}.

\subsubsection{Inventarios}
\label{subsec:inventarios}
Os obxectos poden dispoñer duns campos especiais que non existen no caso das
variables de escenario: os inventarios. Un inventario é unha localización
vinculada a un determinado obxecto. Os inventarios, como o resto de campos,
defínense na definición da clase, pero a diferencia do resto de campos non
poden ser reasignados: o obxecto nace cos seus
inventarios e só desaparecerán cando desapareza o obxecto.
\par Exemplo:
\begin{lstlisting}
heroe {
  % inventario
  % estados
  
  personaxe() {
    new espada() >> inventario
  }
}
\end{lstlisting}
\par
No exemplo podemos ver como a clase {\it Heroe} conta con dous
inventarios: un deles chámase ''inventario'' directamente, e o outro
''estados''. No construtor da clase (ver \ref{subsubsec:construtor})
especifícase que no momento de crearse o obxecto, crearase tamén un obxecto de
clase {\it Espada} e gardarase no inventario.
\par
Os inventarios, a pesar de ter este nome, poden ter o valor semántico que desexe
o DX: poden conter os tesouros que atopara o personaxe, conter unha lista de
estados alterados, conter as ideas dun NPC, etcétera. 





\section{Accións e usuarios}
A acción do xogo desenvólvese mediante accións, que é o que comunica aos
usuarios co mundo. Os usuarios escriben o que queren facer cos seus personaxes,
e é labor do DX executar o código pertinente e mostrarlles unha narración
por escrito que describa o sucedido.
\subsection{Decisións}
Un xogador nun momento dado pode escribir o que desexa que o seu personaxe faga.
Isto no Demiurgo coñécese como {\it decisión}. Este texto non ten ningún tipo de
efecto no sistema, simplemente é mostrado ao DX cando abre o escenario no que se
atopa o obxecto do xogador.

\subsection{Accións}
Por acción enténdese todo o sucedido entre dúas narracións, é dicir, todo o
código executado e as decisións correspondentes que o provocaron. O DX pode
executar todo o código que precise, e facer varias execucións diferenciadas,
antes de engadir unha narración e concluír por tanto a acción.
\subsubsection{Testemuñas}
As testemuñas son os xogadores que se atopan no escenario no momento de rematar
a acción. Estas testemuñas recóllense xusto antes de executar o último código (o
código que iniciou o proceso de narración), polo que se incluirán entre elas os
personaxes que cambiasen de escenario no último momento.

\subsection{Narracións}
A narración é o resultado textual da acción. O DX redáctao baseándose no estado
do escenario e nos {\it echos} (ver sección \ref{sec:echo}) producidos. Os
xogadores que sexan testemuñas poderán ver este resultado textual.
\subsubsection{Fragmentos ocultos}
O DX poderá marcar fragmentos da narración como ocultos empregando as etiquetas
[o][/o]. Por exemplo:
\begin{lstlisting}
Este texto e visible para todos.
[o=gamer]
Este texto so e visible para o xogador 'gamer'
[/o]
[o=gamer user]
Este texto e visible para 'gamer' e 'user'.
[/o]
\end{lstlisting}





\section{Definición da linguaxe COE}
\label{ch:definicion}
No capítulo anterior abordouse o funcionamento dos escenarios, as clases e os
obxectos. Neste capítulo mostrarase unha definición formal da linguaxe, tanto na
definición de clases como no código das accións.
\par
Nas descricións sintácticas é preciso ter en conta as seguintes marcas:
\begin{itemize}
  \item {\bf \textless elemento \textgreater}: Os elementos marcados deste
  xeito son elementos obrigatorios, pero non deben ser entendidos como código
  literal, xa que o contido é unha descrición do elemento. Exemplo: \textless
  argumento \textgreater.
  \item {\bf [ elemento ]}: Os elementos marcados deste xeito son opcionais.
  \item {\bf elemento \ldots}: Cando aparecen os puntos suspensivos, o último
  elemento antes deles pódese repetir varias veces.
  Exemplo:
  \textless arg\textgreater {}[\textless arg\textgreater] \ldots.
\end{itemize}

\subsection{Tipos de datos}
No Demiurgo e na linguaxe COE podemos atopar os seguintes tipos de datos:
\begin{itemize}
  \item {\bf Número enteiro:} Representa un número sen decimais. Útil para facer
  cálculos, e para realizar tiradas de dados. Exemplo: {\it 27}. Tamén
  representa valores lóxicos, sendo 0 o equivalente a dicir ``falso'' e calquera
  outro valor (normalmente 1) o equivalente a dicir ``verdadeiro''. O seu tipo
  denótase coa palabra reservada {\it int}.
  \item {\bf Número con decimais:} Tamén coñecido como número con punto
  flotante ou simplemente flotante, representa un número non enteiro. Pode
  empregarse cando se empreguen campos que precisen decimais. Exemplo: {\it
  3.14}. Debe terse en conta que ao converter números con decimais a números
  enteiros, perderanse os decimais (non se redondea). O seu tipo
  denótase coa palabra reservada {\it float}.
  \item {\bf Cadea de texto:} Calquera fragmento de texto. Útil para mostrar
  mensaxes aos xogadores, ou para gardar anotacións do DX. Exemplo: {\it ''este
  é un texto de exemplo''}. O seu tipo
  denótase coa palabra reservada {\it str}.
  \item {\bf Escenario:} As variables e campos de escenario non conteñen o
  escenario como tal, senón unha referencia ao mesmo. Pode ser útil almacenar
  referencias a escenarios concretos para automatizar o movemento de obxectos
  entre eles. Aparte diso, non serven para facer operacións. Para obter
  unha referencia a un escenario concreto, emprégase o carácter {\it @}.
  Exemplo: {\it @/rexion1/cidade1/taberna}. No caso de non comezar por /,
  enténdese que é un nome parcial; por exemplo se o código se executa en
  /rexion1/cidade1/taberna, e se chama a @casa3, o resultado é o mesmo que
  chamando a @/rexion1/cidade1/casa3. Tamén é posible chamar ao propio
  escenario co punto (@.), a un nivel superior cos dous puntos (@..) ou calquera
  combinación.
  O seu tipo denótase co carácter {\it @}.
  \item {\bf Inventario:} Exclusivo dos campos de obxecto, fan referencia ao
  inventario pertinente. Funcionan de maneira moi similar aos escenarios, salvo
  polo feito de que non se poden enviar como argumento nin sobreescribir o seu
  valor con outra referencia. O seu tipo denótase co carácter \%.
  \item {\bf Obxecto:} Referencia a un obxecto concreto. Permite interactuar co
  obxecto dende outro escenario, ou gardalo nunha variable cun nome descritivo.
  É posible obter unha referencia a un obxecto coñecendo a súa ID. Exemplo:
  \#{\it 1024}. O seu tipo
  denótase co nome da clase correspondente.
  \item {\bf Lista:} Lista de datos de calquera dos outros tipos. Exemplo:
  \{{\it 1, 2, 3, 4}\}. Tamén se poden crear listas baleiras: \{\}. O seu tipo
  denótase co tipo de elementos que contén e uns corchetes, por exemplo int[].
  Tamén se poden especificar listas de listas: int[][].
\end{itemize}


\subsection{Echo}
\label{sec:echo}
{\it Echo} é o termo que recibe o feito de emitir un resultado textual. No
Demiurgo, pódese facer echo de calquera valor que poida ser lido como unha cadea
de texto (tamén números enteiros e flotantes, e listas). Facer echo no código
ten unha utilidade: cando o DX estea redactando a narración da acción, poderá
ver todos os echos, polo que é útil para ver información antes de facer a
narración completa.
\par O echo faise co carácter reservado {\bf !}.
\begin{lstlisting}
! <valor_a_mostrar>
\end{lstlisting}
\par Exemplo:
\begin{lstlisting}
! "este e un texto de proba"
\end{lstlisting}

\subsection{Operacións}
No Demiurgo atopamos as seguintes operacións:
\begin{itemize}
  \item + {\bf Sumar:} Suma dous valores. Se os dous elementos son enteiros, o
  resultado é un enteiro; se un é flotante e o outro enteiro ou flotante, o
  resultado é un flotante. Se un dos elementos é unha cadea de texto, a
  operación no canto de ser unha suma de números convírtese nunha concatenación
  de texto: ``proba de '' + ``texto'' devolvería ``proba de texto'', e ``valor
  de '' + 4 devolvería ``valor de 4''. En calquera outro caso, emite un erro.
  \item - {\bf Restar:} Resta dous valores. Semellante a suma, salvo polo feito
  de que non serve para concatenar texto. Pódese usar cun só operando para obter
  o seu negativo.
  \item ! {\bf Operador lóxico ``non'':} Operador unario, é dicir, que acepta un
  operando. Devolve 1 se o dato é falso, ou 0 se o dato é verdadeiro. Un número
  enteiro ou flotante é verdadeiro se é distinto de 0. Unha lista devolverá unha
  lista de 1s  0s; o resto de tipos devolverá sempre un valor falso.
  \item == {\bf Igual:} Compara dous datos. Se os datos son loxicamente iguais,
  devolve un 1. En caso contrario, devolve un 0. Dúas referencias son iguais se
  referencian ao mesmo obxecto ou escenario.
  \item != {\bf Non igual:} Devolve 0 se os datos son loxicamente iguais, 1 en
  caso contrario.
  \item \textgreater {\bf Maior que:} Devolve 1 se o primeiro valor é maior co
  segundo, 0 en caso contrario. Funciona con enteiros e flotantes.
  \item \textless {\bf Menor que:} Devolve 1 se o primeiro valor é menor co
  segundo, 0 en caso contrario.
  \item \textgreater= {\bf Maior ou igual que:} Devolve 1 se o primeiro valor é maior ou
  igual co segundo, 0 en caso contrario.
  \item \textless= {\bf Menor ou igual que:} Devolve 1 se o primeiro valor é
  menor ou igual co segundo, 0 en caso contrario.
  \item \& {\bf Operador lóxico ``e'':} Compara dous elementos lóxicos. Se os
  dous son verdadeiros, devolve 1, en caso contrario, devolve 0.
  \item \textbar {\bf Operador lóxico ``ou'':} Compara dous elementos lóxicos.
  Se polo menos un dos dous é verdadeiro, devolve 1, en caso contrario, devolve
  0.
  \item = {\bf Asignar:} Asigna o valor da dereita á variable ou campo da
  esquerda. Ademais, se se usa en conxunto con outras operacións, devolve o
  valor asignado. En caso de non ser compatibles o valor e a variable, emite un
  erro.
  \item \textgreater\textgreater {\bf Mover:} Move o obxecto da esquerda á
  localización da dereita. Tamén se poden mover listas de obxectos: en tal caso,
  móvense todos os obxectos da lista.
  \item ++ {\bf Concatenar:} Concatena dúas listas. Deben ser do mesmo tipo. Se
  un dos dous operandos non é unha lista pero ten o mesmo tipo que o tipo
  interno da lista, engádese ao comezo ou ao final (dependendo da orde dos
  operandos).
  \item D {\bf Tirada de dados:} Pódese usar como operador unario ou binario. Se
  se usa cun só operando, fai unha tirada dun dado onde o número de caras é o
  número especificado polo operando (que debe ser un enteiro) e devolve o
  resultado. Se se usa con dous operandos, fai X tiradas de Y caras, onde X é o
  valor do primeiro operando e Y o do segundo, e devolve unha lista de
  resultados.
\end{itemize}
\par A maiores, é posible recoller operacións entre parénteses para darlles
prioridade.
\par A prioridade de operadores é a seguinte:
\par
Tiradas de dados \textgreater{} parénteses \textgreater{} multiplicar ou dividir
\textgreater{} sumar ou restar \textgreater{} comparacións \textgreater{} operacións
lóxicas \textgreater{} asignacións \textgreater{} movementos
\par
Débese ter en conta que as asignacións sempre collerán unha variable do
lado esquerdo, polo que teñen a máxima prioridade por este lado.
\par Exemplos de operacións:
\begin{lstlisting}
(1+2*(3-4))                // Devolve -1
3d20 > 10                  // Devolve unha lista de 
                           // 3 valores 0 ou 1
#42 >> @/zona1/estancia3   // Move o obxecto de
                           // ID 42 ao escenario
\end{lstlisting}
\subsubsection{Operacións con listas}
As listas permiten realizar practicamente todas as operacións dos elementos que
conteñen, pero teñen unha característica: o resultado cambia segundo a operación
sexa con outra lista ou cun elemento dun tipo distinto.
\paragraph{Operacións entre elementos e listas}
Se se realiza unha operación entre un dato que non sexa lista e unha lista, o
Demiurgo o que devolverá será unha lista de resultados entre o dato e cada
elemento da lista.
\par Exemplo:
\begin{lstlisting}
3*{1,2,3}
\end{lstlisting}
\par Este exemplo devolvería a lista \{3,6,9\}.
\par Isto funciona tamén con listas de listas:
\begin{lstlisting}
5+{{5,5},{10,10}}
\end{lstlisting}
\par Este exemplo devolve a lista \{\{10,10\},\{15,15\}\}.
\paragraph{Operacións entre listas}
As operacións entre listas simplemente devolven unha lista cos resultados
elemento a elemento. Se o tamaño das listas non coincide, devolverá un erro.
\par Exemplo:
\begin{lstlisting}
{2,3,6}+{1,2,5}
\end{lstlisting}
\par Este exemplo devolve a lista \{3,5,11\}.
\par Tamén valen combinacións entre os dous tipos de operación:
\begin{lstlisting}
{2,4}*{{10,10},{100,100}}
\end{lstlisting}
\par Devolvería a lista \{\{20,20\},\{400,400\}\}.

\subsection{Variables}
As variables de escenario créanse especificando o seu tipo e o nome da variable.
No momento de definir unha variable nova, é posible asignarlle un valor na mesma
liña:
\begin{lstlisting}
<tipo> <nome_variable> [ = <valor> ]
\end{lstlisting}
\par Exemplo de variables:
\begin{lstlisting}
int a               // define o enteiro 'a'
str b = "proba"     // define a cadea de texto 'b' co
                    // valor "proba"
elfo c = new elfo() // define a referencia 'c' e garda
                    // un novo Elfo 
\end{lstlisting}

\subsection{Condicionais}
É posible marcar fragmentos do código para que sexan executados ou non en base a
unha condición. Se a condición é verdadeira, o código execútase; en caso
contrario, non se executa nada. Tamén é posible especificar un código
alternativo para este último caso.
\par Definición formal:
\begin{lstlisting}
if (<condicion>) {
  <codigo_condicion_verdadeira>
}
[else {
  <codigo_condicion_falsa>
}]
\end{lstlisting}
\par Se o código só é unha liña, pódese prescindir das chaves.
\par Exemplo de condicional:
\begin{lstlisting}
if(d6 > 3) {  // tira un dado de 6 e o compara co
              // enteiro 3
  ! "exito"
}
else {
  "fracaso"
}
\end{lstlisting}

\subsection{Bucles}
É posible especificar bucles no código, é dicir, fragmentos de código que se
executarán varias veces. O Demiurgo conta co bucle {\bf for}, que permite
executar un fragmento de código unha vez por cada elemento dentro dunha lista.
\par Definición formal:
\begin{lstlisting}
for (<variable> : <lista>) {
  <codigo_for>
}
\end{lstlisting}
\par Por cada elemento da lista, farase unha execución do código asignando tal
elemento á variable especificada. Se no canto dunha lista se lle pasa unha cadea
de texto, contará como unha lista de caracteres.
\par Se o código só é unha liña, pódese prescindir das chaves.
\par Exemplo de bucle:
\begin{lstlisting}
for(i : 5d10) {
  ! "resultado: " + i
}
\end{lstlisting}

\subsection{Funcións integradas}
É posible empregar funcións predefinidas no propio Demiurgo para realizar
determinadas operacións complexas. Estas funcións chámanse do seguinte xeito:
\begin{lstlisting}
[<saida>] = <nome_funcion> ( <arg> [, <arg> ] ... )
\end{lstlisting}
\par As funcións dispoñibles son as seguintes:
\begin{itemize}
  \item count({\it lista}): Recibe unha lista e devolve o número de elementos.
  Se o argumento é unha cadea de texto, devolve o número de caracteres.
  \item seq({\it inicio, fin}): Recibe dous números enteiros e devolve unha
  lista formada pola secuencia de enteiros entre ambos números, cun paso de 1. O
  número de inicio debe ser menor ou igual co de fin.
  \item reverse({\it lista}): Recibe unha lista e devolve a lista do revés, é
  dicir, co primeiro elemento en última posición e así sucesivamente. Se o
  argumento é unha lista de listas, as listas do interior non son modificadas,
  só a súa orde.
  \item sub({\it lista, inicio, fin}): Recibe unha lista e dous números
  enteiros e devolve unha sublista composta polos elementos entre a posición de inicio
  (inclusive) e a posición de fin (exclusive).
  \item sum({\it lista}): Recibe unha lista de enteiros e devolve a suma dos
  seus elementos.
  \item zeros({\it total}): Recibe un número enteiro e devolve unha lista
  composta por tantos ceros como especifique o argumento.
  \item destroy({\it obxecto}): Recibe unha referencia de obxecto, e destrúe ese
  obxecto. Desaparecerá do mundo de xogo, e con el os seus inventarios e os
  obxectos que estes conteñan. Esta función non devolve ningún valor.
\end{itemize}
\par Exemplo:
\begin{lstlisting}
sum( seq(1,10) )          // suma os enteiros do 1 ao 10
\end{lstlisting}

\subsection{Obxectos e métodos}
\subsubsection{Creación de obxectos}
Os obxectos créanse coa palabra reservada ``new'' o nome da clase, e se é
preciso os argumentos do método construtor. A definición é esta:
\begin{lstlisting}
new <nome_clase> ( [ <arg> [, <arg> ] ... ] )
\end{lstlisting} 
\par Ademais de crear o obxecto, este método devolve unha referencia ao mesmo,
que pode ser gardada nunha variable mediante unha asignación.
\par
Os obxectos no momento de ser creados aparecen no escenario no que se
estea escribindo o código.
\par Exemplo de creación de obxecto:
\begin{lstlisting}
elfo legolas = new elfo(100, "Legolas o elfo")
\end{lstlisting}

\subsubsection{Invocación de métodos}
Para invocar un método cun obxecto é preciso ter a referencia dese obxecto,
extraer o seu método co punto ({\bf .}) e engadir os argumentos en caso de ser
necesarios. Se o método ten argumento de saída, pódese gardar o seu valor nunha
variable, ou empregarse directamente para facer operacións.
\begin{lstlisting}
<referencia_obxecto>.<nome_metodo> ([<arg> [, <arg>] ...])
\end{lstlisting} 
\par Exemplo:
\begin{lstlisting}
legolas.bendicir(outroelfo)
\end{lstlisting}

\subsection{Definición de clases}
No momento de crear a clase, especifícanse os atributos e métodos que terán os
obxectos creados a partir dela. A definición dunha clase é a seguinte:
\begin{lstlisting}
<nome_clase> {
  <atributos>
  <metodos>
}
\end{lstlisting}
\par Os atributos defínense como as variables de escenario. Os métodos defínense
da seguinte forma:
\begin{lstlisting}
[<tipo> <arg_saida> =]
  <nome_metodo>([ <tipo> <arg>[, <tipo> <arg>]...])
{
  <codigo_metodo>
}
\end{lstlisting}
\par Se no código do método se asigna un valor ao argumento de saída, este
devolverase cando o método sexa chamado.
\par Exemplo de clase:
\begin{lstlisting}
anano : criatura {
  int forza = 3
  int axilidade = 1
  int intelixencia = 1
  int vida = 30
  str nome
  
  anano(str nome) {
    this.nome = nome
  }
  
  int dano = golpear(criatura outro) {
    dano = sum((forza)d6)
    criatura.vida = criatura.vida - dano
  }
}
\end{lstlisting}

\subsection{Contidos das localizacións}
Pódese obter unha lista dos obxectos que se atopan nun escenario ou nun
inventario empregando o carácter reservado \% do seguinte xeito:
\begin{lstlisting}
<localizacion>.%
\end{lstlisting}
\par Por exemplo:
\begin{lstlisting}
personaxe.inventario.% >> @.
\end{lstlisting}
\par Este exemplo colle unha lista de obxectos no inventario do personaxe e
móveos todos ao escenario actual.

\subsection{Usuarios}
Os usuarios non poden ser manexados como valores por ningún tipo de dato; pero
hai unha operación especial ({\bf -\textgreater}) que permite asignarlle un
obxecto a un usuario coñecendo o seu nome.
\begin{lstlisting}
$<nome_usuario> -> <referencia_obxecto>
\end{lstlisting}
\par Exemplo:
\begin{lstlisting}
$gamer -> #30
\end{lstlisting}