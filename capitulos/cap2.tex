\chapter{Xestión do proxecto}

% Planificación e presupostos: debe incluír a estimación do costo (presuposto) e
% dos recursos necesarios para efectuar a implantación do Traballo, xunto coa
% planificación temporal do mesmo e a división en fases e tarefas. Recoméndase
% diferenciar os costos relativos a persoal dos relativos a outros gastos como
% instalacións e equipos.

A xestión de proxectos conleva unha serie de metodoloxías e técnicas encamiñadas
a planificar temporalmente o proxecto para acadar o seu alcance no prazo
previsto, así como garantir a calidade do produto final. Neste capítulo
definirase a metodoloxía de desenvolvemento empregada no proxecto, e farase unha
estimación de prazos e custos.

\section{Metodoloxía de desenvolvemento}
Unha das decisións máis importantes que se deben tomar ao comezo do proxecto é a
metodoloxía de desenvolvemento a empregar. Esta metodoloxía debe permitirnos
estruturar o traballo para garantir a súa finalización cuns mínimos
de calidade aceptables.
\par
A escolla da metodoloxía fundaméntase nas características do proxecto. neste
caso concreto, o noso programa ten unha temática pouco explorada no
mercado, e empregará diversas tecnoloxías e recursos relativamente novos para o
desenvolvedor. A maiores, este ten unha experiencia reducida na xestión de
proxectos grandes. Tendo todas estas cuestións en conta, parece apropiado optar
por unha metodoloxía de desenvolvemento áxil e descartar o uso de metodoloxías
menos tolerantes ao cambio.

\subsection{Scrum}
Optouse polo emprego de \textit{Scrum}\cite{scrum} para o desenvolvemento do
software.
En Scrum, o desenvolvemento realízase de forma incremental mediante iteracións:
as tarefas distribúense temporalmente en sprints de unha ou dúas semanas por
norma xeral. Poténciase especialmente a interacción co cliente e márcanse uns
prazos fixos para cumprir obxectivos, para o cal é preciso priorizar os
requisitos regularmente.

\subsubsection{Actividades}
No modelo de Scrum as reunións xiran en torno aos sprints, e podemos atopar as
seguintes actividades:
\begin{itemize}
  \item Daily Scrum ou Stand-up Meeting: Reunións diarias nas que se revisa o
  o estado do proxecto por parte do equipo. Neste caso concreto non son
  relevantes por ser un proxecto dunha soa persoa.
  \item Sprint Planning Meeting: Reunión de planificación do sprint. Realízase
  ao inicio do sprint, e nela estipúlase o traballo que se realizará no mesmo.
  \item Sprint Review Meeting: Reunión de revisión do sprint. Faise ao final do
  sprint, e revísase o traballo completado e non completado.
  \item Sprint Retrospective: Na retrospectiva do sprint, os membros do equipo
  dan as súas impresións sobre o mesmo. Isto ten a finalidade de mellorar o
  proceso de forma continuada.
\end{itemize}

\subsection{Ferramentas de xestión empregadas}
Hai numerosas ferramentas dispoñibles para xestionar proxectos de tipo Scrum.
Neste proxecto empregarase Acunote, que permite levar o control da lista de
requisitos e dos sprints coas súas tarefas.

\subsubsection{Acunote}
Trátase dunha ferramenta web que ofrece distintas funcionalidades enfocadas na
xestión de proxectos con metodoloxías Scrum, tales como a creación e mantemento
dunha lista de requisitos priorizada ou \textit{product backlog} para o
proxecto, xestión de sprints (con nome, data de inicio e fin, e tarefas) e, xa
dentro de cada sprint, o control do estado de cada tarefa, incluído o tempo
investido nela e o tempo total requerido. A maiores, Acunote ofrece gráficas e
estatísticas do proceso xeradas automaticamente para levar o control do mesmo.

\section{Planificación temporal}
A planificación temporal desenvolveuse inicialmente en base ao anteproxecto
existente. A metodoloxía Scrum non ofrece unha estruturación concreta do
proxecto en fases de vida debido ao seu carácter iterativo, mais pódese facer
unha aproximación dunha división de etapas do mesmo segundo o tipo de traballo
desenvolvido en cada unha delas.

\subsection{Fase inicial}
Nesta primeira fase desenvolveuse unha descrición extensa do proxecto e das
funcionalidades que o produto resultante debe ofrecer. Dividiuse o programa en
compoñentes e determinouse a forma de interactuar destes entre si. Ademais
disto, tomáronse as decisións necesarias sobre a organización do traballo e a
xestión da documentación e de versións.

\subsection{Fase de análise}
A principal tarefa desta fase foi a de realizar unha análise de requisitos a
partir de distintas historias de usuario. Definíronse os conceptos chave do
sistema de forma máis precisa, e marcáronse formalmente os requisitos e as
limitacións do software. Tamén se decidiron as principais tecnoloxías a
empregar no software.

\subsection{Fase de definición da linguaxe}
Esta fase estivo centrada no desenvolvemento dunha definición formal da linguaxe
do xogo, a linguaxe {\it COE}, e a súa implementación en Java. Non se inclúe o
desenvolvemento do analizador semántico, tarefa que forma parte da seguinte
fase.

\subsection{Fase de desenvolvemento}
Na fase de desenvolvemento implementáronse os distintos compoñentes software do
proxecto coas súas distintas funcionalidades. Trátase da fase máis longa, e a
súa realización documéntase en maior detalle no capítulo \ref{ch:implementacion}
pertinente deste documento.

\subsection{Fase de probas}
Nesta última fase fanse probas unitarias do sistema. A maiores, o sistema
instálase nun servidor web para realizar probas con usuarios que poden ser
útiles para extraer posibles melloras en usabilidade ou detectar erros. 

\section{Xestión da configuración}
Os elementos de configuración son os distintos elementos cos que traballamos no
proxecto, dos cales precisamos manter un control
sobre os cambios que se poidan realizar neles, xa que poden afectar
sensiblemente ao desenvolvemento do propio proxecto.
\par
Xa que estamos a falar dun proxecto exclusivamente de desenvolvemento de
software, será o propio código fonte o que consideraremos elemento de
configuración. Para manter a súa integridade e realizar o control de cambios
empregaremos a ferramenta Git.

\subsection{Git}
Git \cite{git} é un software de control de versións que nos permite estabelecer
repositorios de software e controlar os cambios que se realizan no código. Con
Git podemos crear un repositorio para cada elemento independente do proxecto e
manter un control dende calquera ordenador no que traballemos. Permite tamén
visualizar os cambios realizados en cada {\it submit} e desfacelos en caso de
ser preciso.
\subsubsection{GitHub}
GitHub é unha ferramenta web para desenvolvedores que ofrece aloxamento gratuíto
para repositorios Git. Deste modo, é posible manter o repositorio en liña e
traballar co mesmo dende distintos equipos sen necesidade de configurar un
servidor web.

\section{Análise de custos}
Os custos deste proxecto recaen principalmente na man de obra, os equipos
informáticos e os servidores. Non hai custos adicionais en adquirir paquetes de
software, xa que un dos requisitos especificados dende o principio é o uso
exclusivo de ferramentas libres.
\par
O servidor empregado é un VPS \footnote{VPS: Virtual Private Server. É un
servidor virtual dentro dunha máquina real. Habitualmente é ofrecido por
empresas de hosting como unha alternativa barata aos servidores dedicados.}
contratado cunha empresa, que empregaremos para instalar o noso software e facer
probas de usuario.
Calcúlase que transcorrerán 2 meses dende o momento no que é contratado ata a
exposición do proxecto.
\par
Adicionalmente, consideramos un 20\% de custos indirectos.

\begin{tabular} { | c c r | r | }
\hline
Item & Cantidade & Custo unitario & Custo total \\
\hline
PC & 1 & 1000,00 \euro{} & 1000,00 \euro{} \\
\hline
Persoal & 412,5 horas & 15,00 \euro{} & 6187,50 \euro{} \\
\hline
VPS & 2 meses & 20 \euro{} & 40 \euro{} \\
\hline
& & Subtotal: & 7227,50 \euro{} \\
\hline
&& Custos indirectos: & +20\% \\
\hline
&& Total: & 8673 \euro{} \\
\hline
\end{tabular}