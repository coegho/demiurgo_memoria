\chapter{Análise}

Antes de comezar a desenvolver o software deberase facer unha especificación dos
requisitos que este debe cumprir. Neste caso o primeiro que faremos será definir
casos de uso do sistema, para posteriormente extraer a partir deles unha serie
de requisitos funcionais e non funcionais.

\section{Definicións}
Nos casos de uso e requisitos faremos referencia a estas definicións, polo que é
necesario explicalas aquí con precisión.

\subsubsection{Partida}
Instancia concreta deste software xestionada por un administrador. Cada partida
debe ter unha base de datos propia e un directorio cos ficheiros que compoñen o
mundo. Non se contempla a comunicación entre partidas distintas.

\subsubsection{Mundo}
O mundo de xogo é a abstracción de todos os obxectos e clases correspondentes a
unha partida. Pódese distinguir entre a definición do mundo e o seu estado:
\begin{itemize}
\item Definición: O conxunto das clases descritas polos directores de xogo.
\item Estado: O conxunto dos obxectos instanciados a partir das clases.
\end{itemize}

\subsubsection{Xogador}
Usuario que ten control dun personaxe e toma decisións en base a este. Os
xogadores non poden modificar o mundo directamente, senón que as súas
interaccións co mundo realízanse a través dos personaxes.

\subsubsection{Director de xogo (DX)}
Usuario con privilexios que pode alterar o mundo de forma directa, e resolver as
decisións dos xogadores. Poden tanto alterar a definición do mundo (creando e
modificando clases) como cambiar o estado do mesmo (instanciando obxectos e
modificándoos). Os directores de xogo non teñen personaxe propio.

\subsubsection{Clase}
No ámbito do mundo de xogo, abstracción dun conxunto de obxectos con atributos e
métodos comúns, seguindo o paradigma da programación orientada a obxectos. Os
directores de xogo poden crear e modificar estas clases.
\par
Un exemplo de clase sería {\it espada}, {\it porta}, {\it cabalo} ou {\it
guerreiro}.

\subsubsection{Obxecto}
No ámbito do mundo de xogo, cada unha das instancias dunha clase concreta. Os
obxectos {\it existen} no mundo, e sitúanse nun determinad escenario. É labor
dos directores de xogo crealos, modificalos e destruílos. Todos os obxectos
posúen un identificador global dentro do mundo.
\par
Un exemplo de obxecto sería {\it a segunda mesa da posada}, {\it o personaxe dun
dos xogadores} ou {\it o can que acompaña ao grupo de personaxes}.

\subsubsection{Personaxe}
Obxecto especial que representa a un xogador no mundo. Toda a información
percibida polo personaxe será posta en coñecemento do xogador correspondente, e
toda decisión tomada polo xogador determinará as accións do personaxe.

\subsubsection{Decisión}
Descrición verbal dun xogador explicando a acción que pretende que realice o seu
personaxe. As decisións están redactadas en linguaxe informal, pero con
suficiente información como para que o director de xogo poda entender o que o
xogador pretende. É labor do director de xogo xerar as accións a partir das
decisións dos xogadores.
\par
Un exemplo de decisión sería: {\it ''O meu personaxe achégase á espada
incrustada na pedra, di en voz alta 'alá vou' e trata de arrincala.''}

\subsubsection{Acción}
Conxunto de liñas de código que alterarán o mundo de xogo en función dunha
decisión dun xogador. A acción é redactada por un director de xogo segundo o que
interprete lendo a decisión correspondente.

\subsubsection{Suceso}
Conxunto de liñas de código que alterarán o mundo de xogo pero que non depende
da decisión de ningún xogador. Os sucesos son provocados polos directores de
xogo cando precisen alterar algo do mundo sen ser consecuencia directa das
accións dos xogadores.

\subsubsection{Rolda}
No mundo de xogo, en cada escenario as accións se resolven divididas por roldas.
Por cada rolda, cada personaxe pode facer como máximo unha acción. Estas roldas
representan simbolicamente o paso do tempo dentro do mundo, aínda que o avance
do tempo pode non coincidir exactamente en todos as escenarios por igual.

\subsubsection{Rexistro de personaxe}
Todo o que perciba un personaxe determinado será almacenado nun rexistro. O
xogador ten acceso a este rexistro, polo que poderá consultar en todo momento
todo o que o seu personaxe puido oír, ver ou percibir de calquera xeito.

\subsubsection{Escenario}
Cada un dos recintos pechados nos que se divide o mundo de xogo. Tamén coñecidos
como {\it habitacións} ou {\it estancias} noutros sistemas semellantes. Todos os
obxectos están situados nun escenario, incluídos os personaxes, e en calquera
momento poden cambiar de escenario. Os escenarios pódense agrupar en {\it
rexións} mediante nomes compostos.
\par
Exemplos de escenarios poderían ser {\it Capital/mercado}, {\it montañas/mina
abandonada/terceira sección} ou {\it castelo antigo/segunda planta/habitación
1}.

\subsubsection{Tirada}
Resultado aleatorio que representa unha tirada de dados. A tirada especifica o
número de dados tirados e o número de caras de cada dado. Este resultado
emprégase xeralmente para determinar o éxito ou o fracaso das accións dos
personaxes ou doutros obxectos.

\subsubsection{Recipiente}
Calquera obxecto coa capacidade de almacenar outros obxectos. Estes últimos
estarán vinculados ao obxecto recipiente, por tanto atoparanse sempre no mesmo
escenario. Un obxecto pode ser recipiente de varias listas distintas.
Un exemplo de recipiente é {\it un cofre do tesouro}, {\it un barril} ou un {\it
xogador co seu inventario}. Un exemplo de recipiente con varias listas pode ser
{\it un xogador no que se diferencien os obxectos que leva nas mans dos que leva
nas costas}.


\section{Casos de uso}
Mediante os casos de uso podemos representar situacións típicas do noso
software, as cales suporán un bo xeito de extraer requisitos funcionais.

\subsection{Descrición de actores}
Inicialmente teremos que definir os actores que se relacionan co sistema nos
nosos casos de uso. Identificamos tres actores diferenciados: director de xogo,
xogador e administrador.
\begin{itemize}
\item {\bf Director de xogo:} Este actor correspóndese coa definición descrita previamente; o seu cometido por tanto será dirixir a partida, recibindo decisións dos xogadores e transcribíndoas como accións dos seus personaxes.
\item {\bf Xogador:} Calquera usuario que participe na partida cun personaxe. Redacta decisións e envíaas ao DX.
\item {\bf Administrador:} Usuario especial que crea a partida e se encarga de administrala. As súas principais funcións será a asignación do rango de DX a outros usuarios, e a configuración básica da partida.
\end{itemize}

\subsection{Descrición de casos de uso}
A continuación atopamos os seguintes casos de uso, elaborados a partir de
entrevistas con usuarios potenciais. Son posibles situacións que se poden dar
durante a execución do software.
Por simplificar daremos por feito que todos os casos de uso requerirán ao
usuario en cuestión estar autenticado (salvo o caso de uso CU-01).

\subsubsection{CU-01: Alta de xogadores}
\paragraph{Actores}
Xogador
\paragraph{Escenario principal}
O caso de uso comeza cando un usuario anónimo se rexistra no sistema. Para isto
emprega a opción {\it crear novo usuario}, onde introducirá os datos básicos da
súa conta. Por defecto, o seu rol no sistema será o de xogador.

\subsubsection{CU-02: Xestión de usuarios}
\paragraph{Actores}
Administrador
\paragraph{Escenario principal}
O caso de uso comeza cando o administrador desexa asignar un usuario como DX.
Neste caso, abre a sección de {\it xestión de usuarios} e escribe o nome do
usuario cuxos permisos desexa cambiar. O administrador cambia entón o seu tipo
de usuario de {\it xogador} a {\it director de xogo}. De ter xa un personaxe
asignado, deixará de estalo neste mesmo momento.


\subsubsection{CU-03: Definición dunha nova clase}
\paragraph{Actores}
Director de xogo
\paragraph{Escenario principal}
O caso de uso comeza cando un director de xogo pretende crear unha nova clase na
partida. Na interface gráfica deberá acoder á sección de {\it definición do
mundo}. Unha vez aí, escollerá a opción de crear unha clase nova, e implementará
a clase como desexe. Unha vez rematada, o sistema analizará a clase para
comprobar se ten algún erro (léxico ou sintáctico), e en caso de non atopar
ningún, a clase será almacenada e estará lista para o seu uso posterior.

\paragraph{Escenario alternativo}
A clase ten algún erro, por tanto o sistema non a garda nun ficheiro e, no seu
lugar, mostra unha mensaxe de erro instando ao DX a reescribila.

\subsubsection{CU-04: Creación de personaxe}
\paragraph{Actores}
Director de xogo, xogador
\paragraph{Escenario principal}
O caso de uso comeza cando un xogador desexa crear o seu personaxe. O xogador
accede á sección de {\it creación de personaxe}, onde atopará un cadro de texto
que lle solicitará unha breve descrición do que desexa. O xogador redacta o
texto e envíao ao sistema.
\par
O director de xogo trata este texto como unha decisión máis, e a acción
consecuente é crear un obxecto personaxe ligado ao xogador que o solicitou,
situado nun escenario concreta.

\paragraph{Escenario alternativo}
O DX non ten información suficiente na descrición como para facer un personaxe.
O DX elimina esta descrición e escribe unha resposta ao xogador, que se verá
obrigado a repetir o proceso.

\subsubsection{CU-05: Resolución de decisións en accións}
\paragraph{Actores}
Director de xogo, xogador
\paragraph{Escenario principal}
O caso de uso comeza cun xogador desexando facer algo na partida. Accede á
sección de {\it actuar}, onde poderá ler as últimas accións que sucederon no
escenario no que se atopa o seu personaxe. O xogador redacta a súa decisión e
envíaa ao DX. Unha vez feito isto, non poderá tomar máis decisións ata que a
primeira se resolva.
\par
O director de xogo recibe unha notificación de que hai escenarios con decisións
non resoltas. Accese á sección de {\it resolver accións}, onde ve unha lista de
escenarios con decisións sen resolver. Abre un dos escenarios e le a lista de
decisións e os xogadores que as enviaron, ademais do estado actual do escenario.
En consecuencia, escribirá as accións que mellor representen as decisións
tomadas polos xogadores, que pasarán a ser executada polo sistema. Finaliza o
proceso, e o sistema executa o código das accións introducidas polo DX. O
xogador observa entón un breve texto describindo o sucedido.

\paragraph{Escenario alternativo}
O DX comete algún erro escribindo o código das accións, polo que non se poderán
executar. O sistema devolve unha mensaxe de erro.

\subsubsection{CU-06: Creación dun escenario}
\paragraph{Actores}
Director de xogo
\paragraph{Escenario principal}
O caso de uso comeza cando o director de xogo pretende engadir un novo
escenario ao mundo. Accede á sección de {\it lista de escenarios}, onde
aparecerán os escenarios que xa existen debidamente ordenados. Mediante a opción
{\it crear novo escenario} poderá crear un escenario novo de cero (sen
obxectos).
Finaliza o proceso e o escenario créase.

\paragraph{Escenario alternativo}
O DX pode optar por crear o escenario a partir dun modelo e non dende cero.
Neste caso, executarase un código determinado no momento de crearse o escenario,
que a encherá de obxectos predefinidos.

\subsubsection{CU-07: Execución dun suceso}
\paragraph{Actores}
Director de xogo
\paragraph{Escenario principal}
O caso de uso comeza cando o director de xogo pretende causar un suceso nun
escenario. O DX accede á sección de {\it lista de escenarios}, e selecciona o
escenario na que desexe causar o suceso. Unha vez no panel do escenario,
selecciona a opción {\it causar suceso}, e escribe o código do suceso. Cando
finaliza a opción, o sistema executa o código.

\paragraph{Escenario alternativo}
O DX comete algún erro escribindo o código do suceso, polo que non se poderán
executar. O sistema devolve unha mensaxe de erro.

\subsubsection{CU-08: Cambio de escenario dun obxecto}
\paragraph{Actores}
Director de xogo
\paragraph{Escenario principal}
O caso de uso comeza cando o director de xogo pretende mover un obxecto dun
escenario a outro, sexa como consecuencia dunha decisión dun xogador ou non. O
DX escribe o código do cambio de escenario, especificando o obxecto que se
desprazará e o escenario de destino. Cando se executa a acción, o obxecto
abandona o escenario de orixe e pasa a estar no escenario de destino. No momento
de resolver as accións do escenario de destino, o sistema avisará da entrada de
novos obxectos no mesmo para que o DX o teña en conta.


\section{Requisitos funcionais}
\subsubsection{Requisito FN.01}~\\
{\bf Título:} Alta de usuarios\\
{\bf Descrición:} A aplicación debe permitir que se creen novas contas de usuario.\\
{\bf Casos de uso relacionados:} CU-01\\
{\bf Importancia:} Esencial

\subsubsection{Requisito FN.02}~\\
{\bf Título:} Modificación de usuarios\\
{\bf Descrición:} A aplicación debe permitir que se modifiquen contas de usuario
existentes.\\
{\bf Casos de uso relacionados:} CU-02\\
{\bf Importancia:} Esencial

\subsubsection{Requisito FN.03}~\\
{\bf Título:} Identificación de usuarios\\
{\bf Descrición:} A aplicación debe permitir que os usuarios se identifiquen
correctamente.\\
{\bf Casos de uso relacionados:} CU-02, CU-03, CU-04, CU-05, CU-06, CU-07, CU-08\\
{\bf Importancia:} Esencial

\subsubsection{Requisito FN.04}~\\
{\bf Título:} Creación de clases\\
{\bf Descrición:} A aplicación debe permitir ao director de xogo crear clases
novas no mundo.\\
{\bf Casos de uso relacionados:} CU-03\\
{\bf Importancia:} Esencial

\subsubsection{Requisito FN.05}~\\
{\bf Título:} Modificación de clases
{\bf Descrición:} A aplicación debe permitir ao director de xogo modificar
clases existentes.\\
{\bf Casos de uso relacionados:} CU-03\\
{\bf Importancia:} Esencial

\subsubsection{Requisito FN.06}~\\
{\bf Título:} Eliminación de clases\\
{\bf Descrición:} A aplicación debe permitir ao director de xogo eliminar clases
existentes. A aplicación debe garantir que non se borra unha clase da que hai
obxectos instanciados.\\
{\bf Casos de uso relacionados:} CU-03\\
{\bf Importancia:} Esencial

\subsubsection{Requisito FN.04}~\\
{\bf Título:} Herdanza de clases\\
{\bf Descrición:} A aplicación debe permitir facer que unhas clases sexan
fillas de outras, herdando deste xeito os seus métodos.\\
{\bf Casos de uso relacionados:} CU-03\\
{\bf Importancia:} Esencial

\subsubsection{Requisito FN.07}~\\
{\bf Título:} Creación de obxecto\\
{\bf Descrición:} A aplicación debe permitir crear novos obxectos dende as
clases xa existentes. Estes obxectos créanse no contexto dun escenario
determinado.\\
{\bf Casos de uso relacionados:} CU-04, CU-05, CU-06, CU-07
{\bf Importancia:} Esencial

\subsubsection{Requisito FN.08}~\\
{\bf Título:} Modificación de obxecto\\
{\bf Descrición:} A aplicación debe permitir modificar os atributos de obxectos existentes.\\
{\bf Casos de uso relacionados:} CU-05, CU-07
{\bf Importancia:} Esencial

\subsubsection{Requisito FN.09}~\\
{\bf Título:} Eliminación de obxecto\\
{\bf Descrición:} A aplicación debe permitir eliminar obxectos nos escenarios.\\
{\bf Casos de uso relacionados:} CU-05, CU-07
{\bf Importancia:} Esencial

\subsubsection{Requisito FN.10}~\\
{\bf Título:} Ligazón de obxecto personaxe a xogador\\
{\bf Descrición:} A aplicación debe permitir que un determinado obxecto
personaxe se asocie a un xogador, para deste xeito permitir ao xogador ver os
resultados das súas accións e enviar as súas decisións á estancia axeitada.\\
{\bf Casos de uso relacionados:} CU-04\\
{\bf Importancia:} Esencial

\subsubsection{Requisito FN.11}~\\
{\bf Título:} Creación dun escenario baleiro\\
{\bf Descrición:} A aplicación debe permitir ao director de xogo crear un novo
escenario de cero que non conteña obxectos.\\
{\bf Casos de uso relacionados:} CU-06\\
{\bf Importancia:} Esencial

\subsubsection{Requisito FN.12}~\\
{\bf Título:} Eliminación dun escenario\\
{\bf Descrición:} A aplicación debe permitir ao director de xogo eliminar un
escenario do mundo. Todos os obxectos que conteña o escenario serán eliminados.
O escenario non pode conter personaxes.\\
{\bf Casos de uso relacionados:} CU-06\\
{\bf Importancia:} Opcional

\subsubsection{Requisito FN.13}~\\
{\bf Título:} Creación dun modelo de escenario\\
{\bf Descrición:} A aplicación debe permitir ao director de xogo deseñar un
modelo de escenario. Este modelo conterá código que se executará no momento de
crear a estancia.\\
{\bf Casos de uso relacionados:} CU-06\\
{\bf Importancia:} Opcional

\subsubsection{Requisito FN.14}~\\
{\bf Título:} Creación dun escenario dende modelo
{\bf Descrición:} A aplicación debe permitir ao director de xogo crear un novo
escenario empregando un modelo de escenario, executando o código que este modelo contén.\\
{\bf Casos de uso relacionados:} CU-06\\
{\bf Importancia:} Opcional

\subsubsection{Requisito FN.15}~\\
{\bf Título:} Cambio de escenario dun obxecto\\
{\bf Descrición:} A aplicación debe permitir ao director de xogo mover un
obxecto dun escenario a outro. O identificador global do obxecto permanece
igual, pero no novo escenario pode ter un identificador local diferente.\\
{\bf Casos de uso relacionados:} CU-05, CU-07\\
{\bf Importancia:} Esencial

\subsubsection{Requisito FN.16}~\\
{\bf Título:} Envío de decisións\\
{\bf Descrición:} A aplicación debe permitir aos xogadores enviar as súas
decisións para que o DX poda tratalas.\\
{\bf Casos de uso relacionados:} CU-05\\
{\bf Importancia:} Esencial

\subsubsection{Requisito FN.17}~\\
{\bf Título:} Visualización de resultados\\
{\bf Descrición:} A aplicación debe permitir que os xogadores podan ver os
resultados das accións que involucren aos seus personaxes, sexan accións
causadas polas súas decisións, ou accións e sucesos no escenario na que se atopa
o seu personaxe.
{\bf Casos de uso relacionados:} CU-05, CU-07
{\bf Importancia:} Esencial



\section{Restricións de deseño}

\subsubsection{Requisito RD.01}~\\
{\bf Título:} Uso de ferramentas libres\\
{\bf Descrición:} A aplicación debe realizarse empregando unicamente
ferramentas libres, como unha forma non só de reducir custos, senón tamén de
permitir unha licencia libre do propio software resultante.\\
{\bf Importancia:} Esencial


\section{Requisitos non funcionais}

\subsubsection{Requisito NF.01}~\\
{\bf Título:} Linguaxe: Lóxica de escenario\\
{\bf Descrición:} A linguaxe debe proporcionar ao DX a capacidade de crear e
eliminar escenarios, mover obxectos dun escenario a outro, obter a lista de
personaxes nun escenario, e asignar variables locais para nomear aos obxectos ou
almacenar datos de tipos primitivos.O código executado nun escenario non debe
afectar nun principio a outros escenarios diferentes, salvo no movemento de
obxectos entre eles.\\
{\bf Importancia:} Esencial

\subsubsection{Requisito NF.02}~\\
{\bf Título:} Linguaxe: Lóxica de visibilidade e ocultación\\
{\bf Descrición:} A linguaxe debe proporcionar a capacidade de determinar que
obxectos son visibles por un personaxe e cales non mediante o uso de
condicións. Isto non ten un efecto real na xogabilidade (un personaxe pode
interactuar cun obxecto cuxa existencia descoñece se así o determina o DX no
seu código), mais é relevante á hora de ofrecer información aos xogadores. \\
{\bf Importancia:} Opcional

\subsubsection{Requisito NF.03}~\\
{\bf Título:} Linguaxe: Lóxica de inventario\\
{\bf Descrición:} A linguaxe debe permitir que uns obxectos se conteñan aos
outros. Na práctica isto ten varias consecuencias, sendo a máis importante o
feito de que o obxecto dependente deba estar forzosamente no mesmo escenario co
obxecto recipiente. Tamén ten o seu efecto na visibilidade, xa que se un
personaxe non pode ver ao recipiente nunca poderá ver os obxectos que contén. A
linguaxe debe ofrecer a capacidade de especificar as clases que poden
converterse en recipientes, métodos para introducir uns obxectos noutros, e
tamén proporcionará a capacidade de recibir a lista de obxectos dun recipiente.\\
{\bf Importancia:} Esencial

\subsubsection{Requisito NF.04}~\\
{\bf Título:} Linguaxe: Tipos de datos básicos\\
{\bf Descrición:} A linguaxe debe implementar como tipos primitivos os números
(enteiros ou en punto flotante) e as cadeas de texto. Tamén deben implementarse
os arrays, sexan de números ou de obxectos.\\
{\bf Importancia:} Esencial

\subsubsection{Requisito NF.05}~\\
{\bf Título:} Linguaxe: Lóxica de eventos\\
{\bf Descrición:} A linguaxe debe ofrecer a opción de definir código de clase
que se executa automaticamente ante determinados eventos, tales como o cambio
de rolda, a chegada de novos obxectos ao escenario ou o cambio de escenario do
propio obxecto.\\
{\bf Importancia:} Opcional

\subsubsection{Requisito NF.06}~\\
{\bf Título:} Linguaxe: Condicionais e tiradas de dados\\
{\bf Descrición:} A linguaxe debe ofrecer a capacidade de empregar condicións
lóxicas para alterar o fluxo do código. Polo outro lado, a linguaxe debe prover
dun método de obter valores aleatorios para simular as tiradas de dados, nas
que o DX poda especificar o número de dados lanzados e o número de caras de
cada dado.\\
{\bf Importancia:} Esencial

\subsubsection{Requisito NF.07}~\\
{\bf Título:} Linguaxe: Iteracións\\
{\bf Descrición:} A linguaxe debe prover de funcións que permitan traballar en
listas de elementos de forma iterativa, tratando cada elemento por separado.\\
{\bf Importancia:} Esencial

\subsubsection{Requisito NF.08}~\\
{\bf Título:} Linguaxe: Operacións lóxicas e matemáticas\\
{\bf Descrición:} A linguaxe debe ofrecer as operacións lóxicas e matemáticas
básicas, tales como comparacións, sumas e multiplicacións. Tamén debe permitir
a operación en arrays, sexa entre dous arrays ou entre un array e un número
(elemento a elemento).\\
{\bf Importancia:} Esencial

\subsubsection{Requisito NF.09}~\\
{\bf Título:} Linguaxe: Prioridade de decisións\\
{\bf Descrición:} A linguaxe debe permitir definir a orde na que se resolverán
as decisións dos xogadores mediante funcións de comparación. No momento de
resolver as decisións, estas aparecerán ante o DX ordenadas como corresponda.\\
{\bf Importancia:} Opcional

\subsubsection{Requisito NF.10}~\\
{\bf Título:} Resultado textual: Seccións comúns e particulares\\
{\bf Descrición:} No momento de mostrar o texto resultante dunha acción, o DX
debe ter a opción de escribir unha narración común en 3ª persoa para todos os
personaxes, e narracións personalizadas en 2ª persoa para cada xogador por separado.\\
{\bf Importancia:} Esencial

\subsubsection{Requisito NF.11}~\\
{\bf Título:} Resultado textual: Condicións\\
{\bf Descrición:} No momento de mostrar o texto resultante dunha acción, o DX
deberá ter a opción de definir condicións en determinados bloques de texto. Os
xogadores que non cumplan tales condicións non poderán ler eses bloques de
texto. As condicións poderán estar determinadas por variables locais do
escenario e por variables de obxectos concretos.\\
{\bf Importancia:} Esencial

\subsubsection{Requisito NF.12}~\\
{\bf Título:} Resultado textual: Formato especial\\
{\bf Descrición:} No momento de mostrar o texto resultante dunha acción, o DX
poderá empregar formato especial, tal como letra en negriña ou cursiva, ou o
uso de imaxes e enlaces web.\\
{\bf Importancia:} Opcional

\subsubsection{Requisito NF.13}~\\
{\bf Título:} Resultado textual: Anonimato\\
{\bf Descrición:} No momento de mostrar o texto resultante dunha acción, os
xogadores non lerán directamente o nome dos personaxes mencionados, senón que
verán o nome co que eles coñezan a tales personaxes.\\
{\bf Importancia:} Opcional



\section{Requisitos de proxecto}
\subsubsection{Requisito PR.01}~\\
{\bf Título:} Data límite do proxecto\\
{\bf Descrición:} O proxecto deberá rematarse antes do 8 de febreiro do 2017,
data establecida no regulamento de traballos fin de grao para GREI da USC.\\
{\bf Importancia:} Esencial

\section{Requisitos de calidade}
\subsubsection{Requisito CA.01}~\\
{\bf Título:} Internacionalización\\
{\bf Descrición:} O software deberá estar deseñado de tal xeito que se poda
traducir facilmente a distintos idiomas sen realizar cambios no código.\\
{\bf Importancia:} Opcional

\subsubsection{Requisito CA.02}~\\
{\bf Título:} Manual de usuario\\
{\bf Descrición:} O software resultante deberá ir acompañado dun manual de
usuario que explique o funcionamento do mesmo, incidindo especialmente no uso
da linguaxe propia.\\
{\bf Importancia:} Esencial

\subsubsection{Requisito CA.03}~\\
{\bf Título:} Protección contra ataques\\
{\bf Descrición:} O software resultante deberá estar protexido contra os
principais ataques, como por exemplo a inxección de código SQL e a introdución
de código HTML e Javascript non intencionado.\\
{\bf Importancia:} Esencial

\section{Requisitos de almacenamento}
\subsubsection{Requisito AL.01}~\\
{\bf Título:} Persistencia da información\\
{\bf Descrición:} A aplicación debe garantir que a información do mundo
(clases, obxectos, escenarios) se conservará cando a aplicación remate e volva a
poñerse en funcionamento.\\
{\bf Importancia:} Esencial
