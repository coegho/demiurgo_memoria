\chapter{Análise}

Antes de comezar a desenvolver o software deberase facer unha especificación dos
requisitos que este debe cumprir. Habitualmente, nas metodoloxías tradicionais
de desenvolvemento de software, faise unha busca exhaustiva de requisitos no
comezo do proxecto, coa intención de realizar o menor número de cambios posible
aos mesmos. En Scrum, por ser unha metodoloxía áxil esta formulación cambia
radicalmente: os requisitos introdúcense nun {\it Backlog}, a lista priorizada
de requisitos propia desta metodoloxía, e vanse modificando segundo sexa
pertinente por cada iteración.
\par
Polo xeral, os requisitos en Scrum só se analizan en profundidade antes de
realizar o sprint no que serán cumpridos. Debido a isto, a interacción co
cliente é importante e é habitual que os requisitos muden varias veces ao longo
do proxecto, sen por iso ter que realizar custosos procesos de reaxuste.

\section{Definicións}
Antes de comezar cos requisitos será necesario definir correctamente todos os
termos que se empregarán neste capítulo e tamén en capítulos posteriores. Xa que
os termos fan referencia a conceptos moi concretos dos xogos de rol, unha boa
definición é indispensable para poder entender correctamente o documento. 

\subsubsection{Partida}
Cada unha das instancias independentes do software na que poden participar os
usuarios. A partida é por tanto a suma dun mundo de xogo máis os usuarios que
participan nel. Cada partida ten inicialmente un director de xogo e varios
xogadores. Non se contempla a comunicación entre distintas partidas, cada unha
funciona de xeito independente. Unha mesma execución do motor de xogo pode
manter varias partidas funcionando simultaneamente.

\subsubsection{Mundo de xogo}
O mundo de xogo (ou simplemente mundo) é a abstracción de todos os obxectos e
clases correspondentes a unha partida. Pódese distinguir entre a definición do
mundo e o seu estado:
\begin{itemize}
\item Definición: O conxunto das clases descritas polos directores de xogo.
\item Estado: O conxunto dos obxectos instanciados a partir das clases.
\end{itemize}

\subsubsection{Xogador}
Usuario que está asociado a un determinado  personaxe e toma decisións en base a
este. Os xogadores non poden modificar o mundo directamente, senón que as súas
interaccións co mundo realízanse a través dos personaxes.

\subsubsection{Director de xogo (DX)}
Usuario con privilexios que pode alterar o mundo de forma directa, e resolver as
decisións dos xogadores. Poden tanto alterar a definición do mundo (creando e
modificando clases) como cambiar o estado do mesmo (instanciando obxectos e
modificándoos). Os directores de xogo non precisan ter un personaxe propio.

\subsubsection{Clase}
No ámbito do mundo de xogo, abstracción dun conxunto de obxectos con atributos e
métodos comúns, seguindo o paradigma da programación orientada a obxectos. Os
directores de xogo poden crear e modificar estas clases.
\par
Un exemplo de clase sería {\it espada}, {\it porta}, {\it cabalo} ou {\it
guerreiro}.

\subsubsection{Obxecto}
No ámbito do mundo de xogo, cada unha das instancias dunha clase concreta. Os
obxectos {\it existen} no mundo, e sitúanse nun determinado escenario. É labor
dos directores de xogo crealos, modificalos e destruílos. Todos os obxectos
posúen un identificador global numérico dentro do mundo.
\par
Un exemplo de obxecto sería {\it a segunda mesa da posada}, {\it o personaxe dun
dos xogadores} ou {\it o can que acompaña ao grupo de personaxes}.

\subsubsection{Personaxe}
Obxecto asociado a un determinado xogador. Toda a información percibida polo
personaxe será posta en coñecemento do xogador correspondente, e toda decisión
tomada polo xogador determinará as accións do personaxe (isto último baixo a
supervisión do DX).

\subsubsection{Decisión}
Descrición verbal dun xogador explicando a acción que pretende que realice o seu
personaxe. As decisións están redactadas en linguaxe informal, pero con
suficiente información como para que o director de xogo poda entender o que o
xogador pretende. É labor do director de xogo xerar as accións a partir das
decisións dos xogadores.
\par
Un exemplo de decisión sería: {\it ''O meu personaxe achégase á espada
incrustada na pedra, di en voz alta 'alá vou' e trata de arrincala.''}

\subsubsection{Acción}
Conxunto de liñas de código que alterarán o mundo de xogo nun determinado
escenario. O código da acción é escrito por un director de xogo segundo o que
interprete lendo as decisións dos xogadores que se atopen no escenario
correspondente (ou pola súa propia iniciativa se non hai xogadores).

\subsubsection{Narración}
Texto redactado polo DX asociado a unha determinada acción. Os xogadores cuxos
personaxes se atopen no escenario no momento da narración terán acceso a esta.
Esta é a principal vía que teñen os xogadores de recibir información do mundo.


\subsubsection{Rexistro de personaxe}
Todo o que perciba un personaxe determinado (as narracións ás que teña acceso)
formará un rexistro. O xogador ten acceso a este rexistro, polo que poderá
consultar en todo momento todo o que o seu personaxe puido oír, ver ou percibir
de calquera outro xeito.

\subsubsection{Escenario}
Cada un dos recintos pechados nos que se divide o mundo de xogo. Tamén coñecidos
como {\it habitacións} ou {\it estancias} noutros sistemas semellantes. Todos os
obxectos están situados nun escenario, incluídos os personaxes, e en calquera
momento o DX pode cambialos de escenario mediante o código. Os escenarios
pódense agrupar en {\it rexións} mediante nomes compostos.
\par
Exemplos de escenarios poderían ser {\it /Capital/mercado}, {\it
/montañas/minaabandonada/terceiraseccion} ou {\it
/casteloantigo/segundaplanta/habitacion1}.

\subsubsection{Tirada}
Resultado aleatorio que representa unha tirada de dados. A tirada especifica o
número de dados tirados e o número de caras de cada dado. Este resultado
emprégase xeralmente para determinar o éxito ou o fracaso das accións dos
personaxes ou doutros obxectos.

\subsubsection{Inventario}
Os obxectos teñen a capacidade de almacenar outros obxectos. Para iso existe o
concepto do inventario: un inventario é en esencia unha localización especial
de obxectos (semellante aos escenarios) vinculada a un obxecto específico.
Deste modo, un obxecto A pode dispoñer dun inventario que contén aos obxectos B
e C. Á hora da xogabilidade, os personaxes contidos nun inventario
considéranse situados no {\it escenario real} do obxecto que os contén. Un
exemplo de uso de inventarios é {\it un cofre do tesouro}, {\it un barril} ou
un {\it xogador co seu inventario}. Un exemplo de inventarios diferenciados para
o mesmo obxecto pode ser {\it un xogador no que se diferencien os obxectos que
leva nas mans dos que leva nas costas}.


\section{Historias de usuario}
As historias de usuario son características das metodoloxías áxiles. Trátase de
breves descricións do que os distintos usuarios queren que o sistema lles
permita facer, e axudan á hora de extraer requisitos para os sprints.

\subsection{Descrición de roles}
Inicialmente teremos que definir os roles cos que interactuar co sistema.
Identificamos dous roles fundamentais: o director de xogo e o xogador.
\begin{itemize}
\item {\bf Director de xogo:} Este actor correspóndese coa definición descrita
previamente; o seu cometido por tanto será dirixir a partida, recibindo
decisións dos xogadores e transcribíndoas como accións dos seus personaxes.
\item {\bf Xogador:} Calquera usuario que participe na partida cun personaxe.
Redacta decisións e envíaas ao DX.
\end{itemize}

\subsection{Lista de historias de usuario}
As seguintes historias de usuario elaboráronse a partir de entrevistas con
usuarios potenciais. Son situacións típicas que se darán durante a
execución do software.

\subsubsection{HU-01: Alta de xogadores}
Como xogador, quero poder rexistrarme no sistema coa finalidade de poder acceder
a unha partida.

\subsubsection{HU-02: Definición dunha nova clase}
Como director de xogo, quero poder definir unha clase nova no mundo de xogo,
para poder crear obxectos da mesma no futuro.

\subsubsection{HU-03: Creación de personaxe}
Como director de xogo, quero poder asociar un obxecto a un xogador concreto,
para que poda dispoñer dun personaxe propio.

\subsubsection{HU-04: Envío de decisións}
Como xogador, quero poder enviarlle as miñas decisións ao DX, para que este as
teña en conta á hora de resolver accións.
\subsubsection{HU-05: Resolución de accións}
Como director de xogo, quero poder ler as decisións dos xogadores, para deste
modo saber como resolver as accións.

\subsubsection{HU-06: Creación dun escenario}
Como director de xogo, quero crear escenarios para poder introducir obxectos
neles no futuro.

\subsubsection{HU-07: Cambio de escenario dun obxecto}
Como director de xogo, quero mover obxectos dun escenario a outro, para poder
desprazar aos xogadores polo mundo de xogo.


\section{Requisitos funcionais}
\subsubsection{Requisito FN.01}~\\
{\bf Título:} Alta de usuarios\\
{\bf Descrición:} A aplicación debe permitir que se creen novas contas de usuario.\\
{\bf Importancia:} Esencial

\subsubsection{Requisito FN.02}~\\
{\bf Título:} Modificación de usuarios\\
{\bf Descrición:} A aplicación debe permitir que se modifiquen os datos
(enderezo electrónico, contrasinal) de contas de usuario existentes.\\
{\bf Importancia:} Esencial

\subsubsection{Requisito FN.03}~\\
{\bf Título:} Identificación de usuarios\\
{\bf Descrición:} A aplicación debe permitir que os usuarios se identifiquen
correctamente, e se recoñeza o seu rol dentro do sistema.\\
{\bf Importancia:} Esencial

\subsubsection{Requisito FN.04}~\\
{\bf Título:} Creación de clases\\
{\bf Descrición:} A aplicación debe permitir ao director de xogo crear clases
novas no mundo.\\
{\bf Importancia:} Esencial

\subsubsection{Requisito FN.05}~\\
{\bf Título:} Modificación de clases
{\bf Descrición:} A aplicación debe permitir ao director de xogo modificar
clases existentes.\\
{\bf Importancia:} Esencial

\subsubsection{Requisito FN.06}~\\
{\bf Título:} Eliminación de clases\\
{\bf Descrición:} A aplicación debe permitir ao director de xogo eliminar clases
existentes. A aplicación debe garantir que non existan obxectos non asociados
a ningunha clase.\\
{\bf Importancia:} Esencial

\subsubsection{Requisito FN.04}~\\
{\bf Título:} Herdanza de clases\\
{\bf Descrición:} A aplicación debe permitir que unhas clases sexan
fillas de outras, herdando deste xeito os seus métodos e campos.\\
{\bf Importancia:} Esencial

\subsubsection{Requisito FN.07}~\\
{\bf Título:} Creación de obxecto\\
{\bf Descrición:} A aplicación debe permitir crear novos obxectos dende as
clases xa existentes. Estes obxectos créanse no contexto dun escenario
determinado.\\
{\bf Importancia:} Esencial

\subsubsection{Requisito FN.08}~\\
{\bf Título:} Modificación de obxecto\\
{\bf Descrición:} A aplicación debe permitir modificar o valor dos campos de
obxectos existentes.\\
{\bf Importancia:} Esencial

\subsubsection{Requisito FN.09}~\\
{\bf Título:} Eliminación de obxecto\\
{\bf Descrición:} A aplicación debe permitir eliminar obxectos nos escenarios.\\
{\bf Importancia:} Esencial

\subsubsection{Requisito FN.10}~\\
{\bf Título:} Ligazón de obxecto personaxe a xogador\\
{\bf Descrición:} A aplicación debe permitir que un determinado obxecto se
asocie a un xogador, para deste xeito permitir ao xogador ver as narracións das
súas accións e enviar as súas decisións á estancia axeitada.\\
{\bf Importancia:} Esencial

\subsubsection{Requisito FN.11}~\\
{\bf Título:} Creación dun escenario baleiro\\
{\bf Descrición:} A aplicación debe permitir ao director de xogo crear un novo
escenario de cero que non conteña obxectos.\\
{\bf Importancia:} Esencial

\subsubsection{Requisito FN.12}~\\
{\bf Título:} Eliminación dun escenario\\
{\bf Descrición:} A aplicación debe permitir ao director de xogo eliminar un
escenario do mundo. Todos os obxectos que conteña o escenario serán eliminados.\\
{\bf Importancia:} Opcional

\subsubsection{Requisito FN.13}~\\
{\bf Título:} Creación dun modelo de escenario\\
{\bf Descrición:} A aplicación debe permitir ao director de xogo deseñar un
modelo de escenario. Este modelo conterá código que se executará no momento de
crear a estancia.\\
{\bf Importancia:} Opcional

\subsubsection{Requisito FN.14}~\\
{\bf Título:} Creación dun escenario dende modelo
{\bf Descrición:} A aplicación debe permitir ao director de xogo crear un novo
escenario empregando un modelo de escenario, executando o código que este modelo contén.\\
{\bf Importancia:} Opcional

\subsubsection{Requisito FN.15}~\\
{\bf Título:} Cambio de escenario dun obxecto\\
{\bf Descrición:} A aplicación debe permitir ao director de xogo mover un
obxecto dun escenario a outro. O identificador global do obxecto permanece
igual.\\
{\bf Importancia:} Esencial

\subsubsection{Requisito FN.16}~\\
{\bf Título:} Envío de decisións\\
{\bf Descrición:} A aplicación debe permitir aos xogadores enviar as súas
decisións para que o DX poda tratalas.\\
{\bf Importancia:} Esencial

\subsubsection{Requisito FN.17}~\\
{\bf Título:} Visualización de narracións\\
{\bf Descrición:} A aplicación debe permitir que os xogadores podan ver os
resultados das accións que involucren aos seus personaxes, sexan accións
causadas polas súas decisións ou non.\\
{\bf Importancia:} Esencial


\section{Restricións de deseño}

\subsubsection{Requisito RD.01}~\\
{\bf Título:} Uso de ferramentas libres\\
{\bf Descrición:} A aplicación debe realizarse empregando unicamente
ferramentas libres, como unha forma non só de reducir custos, senón tamén de
permitir unha licencia libre do propio software resultante.\\
{\bf Importancia:} Esencial


\section{Requisitos non funcionais}

\subsubsection{Requisito NF.01}~\\
{\bf Título:} Linguaxe: Lóxica de escenario\\
{\bf Descrición:} A linguaxe debe proporcionar ao DX a capacidade de crear e
eliminar escenarios, mover obxectos dun escenario a outro, obter a lista de
personaxes nun escenario, e asignar variables locais para nomear aos obxectos ou
almacenar datos de tipos primitivos.O código executado nun escenario non debe
afectar nun principio a outros escenarios diferentes, salvo no movemento de
obxectos entre eles.\\
{\bf Importancia:} Esencial

\subsubsection{Requisito NF.02}~\\
{\bf Título:} Linguaxe: Lóxica de visibilidade e ocultación\\
{\bf Descrición:} A linguaxe debe proporcionar a capacidade de determinar que
obxectos son visibles por un personaxe e cales non mediante o uso de
condicións. Isto non ten un efecto real na xogabilidade (un personaxe pode
interactuar cun obxecto cuxa existencia descoñece se así o determina o DX no
seu código), mais é relevante á hora de ofrecer información aos xogadores. \\
{\bf Importancia:} Opcional

\subsubsection{Requisito NF.03}~\\
{\bf Título:} Linguaxe: Lóxica de inventario\\
{\bf Descrición:} A linguaxe debe permitir que uns obxectos se conteñan aos
outros. Na práctica isto ten varias consecuencias, sendo a máis importante o
feito de que o obxecto dependente deba estar forzosamente no mesmo escenario co
obxecto recipiente. Tamén ten o seu efecto na visibilidade, xa que se un
personaxe non pode ver ao recipiente nunca poderá ver os obxectos que contén. A
linguaxe debe ofrecer a capacidade de especificar as clases que conteñen
inventarios, métodos para introducir uns obxectos noutros, e tamén proporcionará
a capacidade de recibir a lista de obxectos dun inventario.\\
{\bf Importancia:} Esencial

\subsubsection{Requisito NF.04}~\\
{\bf Título:} Linguaxe: Tipos de datos básicos\\
{\bf Descrición:} A linguaxe debe implementar como tipos primitivos os números
(enteiros ou en punto flotante) e as cadeas de texto. Tamén deben implementarse
os arrays, sexan de números, cadeas de texto ou obxectos.\\
{\bf Importancia:} Esencial

\subsubsection{Requisito NF.05}~\\
{\bf Título:} Linguaxe: Lóxica de eventos\\
{\bf Descrición:} A linguaxe debe ofrecer a opción de definir código de clase
que se executa automaticamente ante determinados eventos, tales como o cambio
de rolda, a chegada de novos obxectos ao escenario ou o cambio de escenario do
propio obxecto.\\
{\bf Importancia:} Opcional

\subsubsection{Requisito NF.06}~\\
{\bf Título:} Linguaxe: Condicionais e tiradas de dados\\
{\bf Descrición:} A linguaxe debe ofrecer a capacidade de empregar condicións
lóxicas para alterar o fluxo do código. Polo outro lado, a linguaxe debe prover
dun método de obter valores aleatorios para simular as tiradas de dados, nas
que o DX poda especificar o número de dados lanzados e o número de caras de
cada dado.\\
{\bf Importancia:} Esencial

\subsubsection{Requisito NF.07}~\\
{\bf Título:} Linguaxe: Iteracións\\
{\bf Descrición:} A linguaxe debe prover de funcións que permitan traballar en
listas de elementos de forma iterativa, tratando cada elemento por separado.\\
{\bf Importancia:} Esencial

\subsubsection{Requisito NF.08}~\\
{\bf Título:} Linguaxe: Operacións lóxicas e matemáticas\\
{\bf Descrición:} A linguaxe debe ofrecer as operacións lóxicas e matemáticas
básicas, tales como comparacións, sumas e multiplicacións. Tamén debe permitir
a operación en arrays, sexa entre dous arrays ou entre un array e un número
(elemento a elemento).\\
{\bf Importancia:} Esencial

\subsubsection{Requisito NF.09}~\\
{\bf Título:} Linguaxe: Prioridade de decisións\\
{\bf Descrición:} A linguaxe debe permitir definir a orde na que se resolverán
as decisións dos xogadores mediante funcións de comparación. No momento de
resolver as decisións, estas aparecerán ante o DX ordenadas como corresponda.\\
{\bf Importancia:} Opcional

\subsubsection{Requisito NF.10}~\\
{\bf Título:} Resultado textual: Seccións comúns e particulares\\
{\bf Descrición:} No momento de mostrar o texto resultante dunha acción, o DX
debe ter a opción de escribir unha narración común en 3ª persoa para todos os
personaxes, e narracións personalizadas en 2ª persoa para cada xogador por separado.\\
{\bf Importancia:} Esencial

\subsubsection{Requisito NF.11}~\\
{\bf Título:} Resultado textual: Condicións\\
{\bf Descrición:} No momento de mostrar a narración, o DX deberá ter a opción de
definir condicións en determinados bloques de texto. Os xogadores que non
cumplan tales condicións non poderán ler eses bloques de texto.\\
{\bf Importancia:} Esencial

\subsubsection{Requisito NF.12}~\\
{\bf Título:} Resultado textual: Formato especial\\
{\bf Descrición:} No momento de mostrar o texto resultante dunha acción, o DX
poderá empregar formato especial, tal como letra en negriña ou cursiva, ou o
uso de imaxes e enlaces web.\\
{\bf Importancia:} Opcional

\subsubsection{Requisito NF.13}~\\
{\bf Título:} Resultado textual: Anonimato\\
{\bf Descrición:} No momento de mostrar o texto resultante dunha acción, os
xogadores non lerán directamente o nome dos personaxes mencionados, senón que
verán o nome co que eles coñezan a tales personaxes.\\
{\bf Importancia:} Opcional



\section{Requisitos de proxecto}
\subsubsection{Requisito PR.01}~\\
{\bf Título:} Data límite do proxecto\\
{\bf Descrición:} O proxecto deberá rematarse antes do 8 de febreiro do 2017,
data establecida no regulamento de traballos fin de grao para GREI da USC.\\
{\bf Importancia:} Esencial

\section{Requisitos de calidade}
\subsubsection{Requisito CA.01}~\\
{\bf Título:} Internacionalización\\
{\bf Descrición:} O software deberá estar deseñado de tal xeito que se poda
traducir facilmente a distintos idiomas sen realizar cambios no código.\\
{\bf Importancia:} Opcional

\subsubsection{Requisito CA.02}~\\
{\bf Título:} Manual de usuario\\
{\bf Descrición:} O software resultante deberá ir acompañado dun manual de
usuario que explique o funcionamento do mesmo, incidindo especialmente no uso
da linguaxe propia.\\
{\bf Importancia:} Esencial

\subsubsection{Requisito CA.03}~\\
{\bf Título:} Protección contra ataques\\
{\bf Descrición:} O software resultante deberá estar protexido contra os
principais ataques, como por exemplo a inxección de código SQL e a introdución
de código HTML e Javascript non intencionado.\\
{\bf Importancia:} Esencial

\section{Requisitos de almacenamento}
\subsubsection{Requisito AL.01}~\\
{\bf Título:} Persistencia da información\\
{\bf Descrición:} A aplicación debe garantir que a información do mundo
(clases, obxectos, escenarios) se conservará cando a aplicación remate e volva a
poñerse en funcionamento.\\
{\bf Importancia:} Esencial
