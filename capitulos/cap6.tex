\chapter{Implementación e probas}
\label{ch:implementacion}
Unha vez exposto todo o funcionamento do sistema, e tras facer unha descrición
exhaustiva da arquitectura e o deseño dos distintos compoñentes do mesmo, o
último capítulo tratará sobre a implementación do sistema. Neste capítulo
exporanse as principais dificultades atopadas durante o desenvolvemento do
programa, e as decisións críticas que foi necesario tomar. Posteriormente haberá
unha sección destinada ás probas, tanto unitarias como de usabilidade.
Finalmente, farase un resumo do proceso de desenvolvemento, comentando os
distintos sprints que se realizaron.

\section{Principais problemas atopados}
O desenvolvemento deste proxecto supuxo numerosos obstáculos e dificultades, e
producíronse atrasos durante o mesmo que impediron que a entrega puidera ser
realizada no prazo previsto inicialmente. A principal dificultade produciuse
debido a unha incorrecta estimación da complexidade do proxecto, que conlevou
que o prazo de entrega inicial non se puidera cumprir. Isto implicou a
necesidade de cambiar esta data de entrega ata o 8 de febreiro, para cumplir
deste modo coas normas do proxecto.
\par
Outra complicación que afectou de xeito constante ao desenvolvemento do
programa foi o uso de tecnoloxías e ferramentas coas que o desenvolvedor ten
pouca experiencia, que obrigou a unha constante formación e busca de
alternativas ao longo do ciclo de vida do proxecto. Isto ademais supuxo unha
perda de tempo en determinados momentos ao descartar o uso de certas tecnoloxías
en beneficio de outras máis axeitadas.
\par
Por último, na fase final do proxecto xurdiu unha dificultade imprevista
inicialmente: a necesidade de buscar e configurar un servidor VPS para montar un
servidor de xogo completo e facer probas con usuarios reais. Isto implicou un
custo tanto temporal como económico imprevisto que retrasou o proxecto.

\section{Probas}

\section{Sprints}
Seguindo a metodoloxía Scrum, o proceso dividiuse en sprints nos que se
agrupaban as tarefas entre dúas sesións. Os sprints finalmente foron os
seguintes:
\subsection{Fase de análise}
\subsubsection{Análise de requisitos}
\paragraph{Data de inicio:} 13/03/2016
\paragraph{Data de fin:} 25/04/2016
\paragraph{Tarefas:} Elaboración do anteproxecto, análise de requisitos


\subsection{Definición da linguaxe}
\subsubsection{Bosquexos da linguaxe}
\paragraph{Data de inicio:} 25/04/2016
\paragraph{Data de fin:} 23/05/2016
\paragraph{Tarefas:} bosquexo dunha rolda de exemplo, bosquexos de varias clases
de exemplo

\subsubsection{Definición formal da linguaxe}
\paragraph{Data de inicio:} 23/05/2016
\paragraph{Data de fin:} 01/06/2016
\paragraph{Tarefas:} Definición formal BNF

\subsubsection{Analizador sintáctico}
\paragraph{Data de inicio:} 01/06/2016
\paragraph{Data de fin:} 09/06/2016
\paragraph{Tarefas:} Implementación en ANTLR - análise sintáctica básica,
Expansión da gramática, Revisión e probas da gramática

\subsubsection{Implementación da linguaxe en Java}
\paragraph{Data de inicio:} 09/06/2016
\paragraph{Data de fin:} 21/06/2016
\paragraph{Tarefas:} Implementación do programa base en Java, Implementación do
analizador sintáctico no programa, Implementación do sistema de clases e
escenarios

\subsubsection{Clases e escenarios}
\paragraph{Data de inicio:} 21/06/2016
\paragraph{Data de fin:} 19/07/2016
\paragraph{Tarefas:} Implementación da funcionalidade de definir clases no
código, Implementación dos escenarios/estancias e toda a súa funcionalidade
correspondente


\subsection{Desenvolvemento de software}
\subsubsection{Comprobación do input}
\paragraph{Data de inicio:} 19:07/2016
\paragraph{Data de fin:} 08/08/2016
\paragraph{Tarefas:} Comprobacións de compatibilidade de tipos á hora de asignar
valores, Implementación da clase object e os seus métodos por defecto, Método
in-code para mostrar texto ao DX (echo), Construtor da clase, Control de erros

\subsubsection{Interacción co usuario}
\paragraph{Data de inicio:} 08/08/2016
\paragraph{Data de fin:} 29/08/2016
\paragraph{Tarefas:} Interface de proba para comunicarse co sistema, base de
datos

\subsubsection{Servizos REST}
\paragraph{Data de inicio:} 29/08/2016
\paragraph{Data de fin:} 19/09/2016
\paragraph{Tarefas:} Substitución de RMI por servizos REST, deseño DemiurgoWeb a
partir de Framework

\subsubsection{Base da memoria}
\paragraph{Data de inicio:} 19/09/2016
\paragraph{Data de fin:} 09/11/2016
\paragraph{Tarefas:} Memoria: introdución e trazos fundamentais no deseño,
memoria: diagrama de clases do universo de xogo

\subsubsection{Deseño inicial da web}
\paragraph{Data de inicio:} 09/11/2016
\paragraph{Data de fin:} 15/12/2016
\paragraph{Tarefas:} Adaptación dos servizos web a JSON, interface web 	para DX,
interface web para usuario, altas de usuarios

\subsubsection{Apartado gráfico da interface}
\paragraph{Data de inicio:} 15/12/2016
\paragraph{Data de fin:} 28/12/2016
\paragraph{Tarefas:} Deseño de CSS, Creación e modificación de clases

\subsubsection{Inventarios e configuración}
\paragraph{Data de inicio:} 28/12/2016
\paragraph{Data de fin:} 08/02/2017
\paragraph{Tarefas:} 
