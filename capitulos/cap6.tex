\chapter{Validación e probas}
\label{ch:implementacion}
Unha vez exposto todo o funcionamento do sistema, e tras facer unha descrición
exhaustiva da arquitectura e o deseño dos distintos compoñentes do mesmo, o
último capítulo tratará sobre a implementación do sistema. Neste capítulo
exporanse as principais dificultades atopadas durante o desenvolvemento do
programa, e as decisións críticas que foi necesario tomar. Posteriormente haberá
unha sección destinada ás probas, tanto unitarias como de usabilidade.
Finalmente, farase un resumo do proceso de desenvolvemento, comentando os
distintos sprints que se realizaron.

\section{Principais problemas atopados}
O desenvolvemento deste proxecto supuxo numerosos obstáculos e dificultades, e
producíronse atrasos durante o mesmo que impediron que a entrega puidera ser
realizada no prazo previsto inicialmente. A principal dificultade produciuse
debido a unha incorrecta estimación da complexidade do proxecto, que conlevou
que o prazo de entrega inicial non se puidera cumprir. Isto implicou a
necesidade de cambiar esta data de entrega ata o 8 de febreiro, para cumplir
deste modo coas normas do proxecto.
\par
Outra complicación que afectou de xeito constante ao desenvolvemento do
programa foi o uso de tecnoloxías e ferramentas coas que o desenvolvedor ten
pouca experiencia, que obrigou a unha constante formación e busca de
alternativas ao longo do ciclo de vida do proxecto. Isto ademais supuxo unha
perda de tempo en determinados momentos ao descartar o uso de certas tecnoloxías
en beneficio de outras máis axeitadas.
\par
Por último, na fase final do proxecto xurdiu unha dificultade imprevista
inicialmente: a necesidade de buscar e configurar un servidor VPS para montar un
servidor de xogo completo e facer probas con usuarios reais. Isto implicou un
custo tanto temporal como económico imprevisto que retrasou o proxecto.

\section{Probas}
Realizouse un plan de probas para cada unha das dúas partes básicas da
arquitectura do proxecto. Por un lado, no motor do xogo deseñouse unha batería
de probas unitarias específica  con jUnit para comprobar o correcto
funcionamento do intérprete do código COE, debido ao feito de ser este o
compoñente principal e o que conleva unha maior posibilidade de erro. En canto
ao servidor web, aproveitouse o feito de que o programa será probado por
usuarios reais antes da data de entrega do proxecto para centrar o plan de
probas no \textit{feedback} dos propios usuarios.
 
\subsection{Motor de xogo}
As probas formais estiveron centradas en probar as distintas funcionalidades da
linguaxe. Nelas enviouse un fragmento de código COE ao sistema, e comprobouse
se o resultado no mundo de xogo foi o esperado. As probas están centradas nas
funcionalidades máis frecuentes e/ou máis complexas do sistema.
 
As probas implementáronse mediante o uso de jUnit \cite{junit}, e poden atoparse
no código fonte do sistema no directorio src/test/java. A ferramenta Maven
fomenta o uso deste tipo de probas, que executa automaticamente no momento de
compilar e empaquetar o programa.

Todas as probas parten dun mundo inicial baleiro que conta cun único escenario,
chamado \textit{/room}.
 
A continuación móstranse as distintas probas realizadas:
 
\subsubsection{Crear un obxecto}
A creación de obxectos é unha das actividades máis habituais do sistema. Debido
a isto, fíxose unha proba para comprobar que funciona correctamente. O código
solicita crear un obxecto de clase \textit{Object}, e compróbase que tal obxecto
foi creado correctamente.

\textbf{Código COE}:
\begin{lstlisting}
new object()
\end{lstlisting}

\subsubsection{Crear unha clase}
Crear clases é unha tarefa crucial para deseñar o mundo de xogo. Nesta proba
introdúcese o código COE preciso para crear unha clase chamada
\textit{newclass}, que contén dous campos e un método. Tras a execución do
código, compróbase que a clase está correctamente creada.

\textbf{Código COE}:
\begin{lstlisting}
newclass {
int integer
str string
method() { }
}
\end{lstlisting}

\subsubsection{Actualizar unha clase}
A actualización ou redeseño de clases é unha tarefa relativamente pouco
frecuente, pero cunha alta probabilidade de fallos debido á súa complexidade.
Cando unha clase é modificada, o sistema debe actualizar os seus campos e
métodos, pero tamén actualizar todos os obxectos que herden desa clase. Na proba
créase unha clase \textit{oldclass} e un obxecto da mesma, e posteriormente
modifícase a clase tal e como o faría o Demiurgo. Unha vez executado todo o
código, compróbase o estado da clase e do obxecto.

\textbf{Código COE}:
\begin{lstlisting}
oldclass {
}

newclass {
int var1
str var2
int out = method(int in) {
}
}
\end{lstlisting}

\subsubsection{Mover un obxecto}
Mover obxectos dunha localización a outra é unha tarefa frecuente e con
posibilidades de erro, xa que debe modificar tanto o obxecto como as dúas
localizacións involucradas. Nesta proba créase un novo escenario e un obxecto
\textit{Object}, para posteriormente realizar o desprazamento. Finalmente,
compróbase que o obxecto está no escenario correcto e os escenarios teñen a súa
lista de obxectos actualizada.

\textbf{Código COE}:
\begin{lstlisting}
#0 >> @/anew/room
\end{lstlisting}

\subsubsection{Chamar a un método}
Os métodos son a forma de executar código relativo ás clases, e por tanto o seu
uso é moi frecuente. Na proba, créase unha clase \textit{newclass} que conta cun
campo e un método; a única operación do método é cambiar o valor do campo. A
proba consiste en comprobar o valor final do campo tras a execución do método.

\textbf{Código COE}:
\begin{lstlisting}
newclass {
int vAr = 5
method() {
var = 7
}
}

newclass oBj = new newclass()
obJ.method()
\end{lstlisting}

\subsubsection{Bucle for}
Os bucles \textit{for} proporcionan a posibilidade de realizar operacións
iterativas sobre os elementos do xogo. Nesta proba execútase un bucle
\textit{for} con 10 iteracións, e compróbase se a execución se realizou as 10
veces.

\textbf{Código COE}:
\begin{lstlisting}
for(a : seq(1,10)) {
var = a }
\end{lstlisting}

\subsubsection{Condición if}
Nesta proba impleméntase unha condición falsa, e compróbase que o código que
contén non se executa.

\textbf{Código COE}:
\begin{lstlisting}
$useR1 -> #0
\end{lstlisting}

\subsubsection{Asignar personaxe}
O código COE permite asignar un obxecto a un usuario, convertíndose o primeiro
no personaxe do segundo. Nesta proba compróbase que a asignación se fai
correctamente.

\textbf{Código COE}:
\begin{lstlisting}
new object()
\end{lstlisting}

\subsection{Servidor web}
O plan de probas do servidor web está centrado nos usuarios. Para as probas
contratouse un plan cunha compañía de hosting para ter acceso a un servidor VPS
no que instalar o sistema completo. Posteriormente, inicializouse o mesmo e
foi creada unha partida no mesmo. A partir dese punto, o papel de director de
xogo déixase en mans dun usuario non relacionado co traballo, e incorpóranse
xogadores reais. Debido a isto, as probas contan cunha gran fiabilidade, xa que
se fundamentan no uso do software por parte de usuarios normais sen coñecementos
de informática.
\subsubsection{Probas de usabilidade}
Para as probas de usabilidade fixéronse dous cuestionarios: un destinado ao
director de xogo, e outro feito para os xogadores. Estes cuestionarios serán
cubertos regularmente polos distintos usuarios, obtendo deste modo información
útil á hora de realizar cambios no Demiurgoweb.

\begin{figure}
\centerline{\includegraphics[width=0.8\textwidth]{figuras/cuestionario_dx.pdf}}
\caption{Cuestionario de usabilidade para o director de xogo.}
\label{fig:cuestion-dx}
\end{figure}

\begin{figure}
\centerline{\includegraphics[width=0.8\textwidth]{figuras/cuestionario_xogador.pdf}}
\caption{Cuestionario de usabilidade para o xogador.}
\label{fig:cuestion-xog}
\end{figure}


\section{Sprints}
Seguindo a metodoloxía Scrum, o proceso dividiuse en sprints nos que se
agrupaban as tarefas entre dúas sesións. Os sprints finalmente foron os
seguintes:
\subsection{Fase de análise}
\subsubsection{Análise de requisitos}
\paragraph{Data de inicio:} 13/03/2016
\paragraph{Data de fin:} 25/04/2016
\paragraph{Tarefas:} Elaboración do anteproxecto, análise de requisitos


\subsection{Definición da linguaxe}
\subsubsection{Bosquexos da linguaxe}
\paragraph{Data de inicio:} 25/04/2016
\paragraph{Data de fin:} 23/05/2016
\paragraph{Tarefas:} bosquexo dunha rolda de exemplo, bosquexos de varias clases
de exemplo

\subsubsection{Definición formal da linguaxe}
\paragraph{Data de inicio:} 23/05/2016
\paragraph{Data de fin:} 01/06/2016
\paragraph{Tarefas:} Definición formal BNF

\subsubsection{Analizador sintáctico}
\paragraph{Data de inicio:} 01/06/2016
\paragraph{Data de fin:} 09/06/2016
\paragraph{Tarefas:} Implementación en ANTLR - análise sintáctica básica,
Expansión da gramática, Revisión e probas da gramática

\subsubsection{Implementación da linguaxe en Java}
\paragraph{Data de inicio:} 09/06/2016
\paragraph{Data de fin:} 21/06/2016
\paragraph{Tarefas:} Implementación do programa base en Java, Implementación do
analizador sintáctico no programa, Implementación do sistema de clases e
escenarios

\subsubsection{Clases e escenarios}
\paragraph{Data de inicio:} 21/06/2016
\paragraph{Data de fin:} 19/07/2016
\paragraph{Tarefas:} Implementación da funcionalidade de definir clases no
código, Implementación dos escenarios/estancias e toda a súa funcionalidade
correspondente


\subsection{Desenvolvemento de software}
\subsubsection{Comprobación do input}
\paragraph{Data de inicio:} 19:07/2016
\paragraph{Data de fin:} 08/08/2016
\paragraph{Tarefas:} Comprobacións de compatibilidade de tipos á hora de asignar
valores, Implementación da clase object e os seus métodos por defecto, Método
in-code para mostrar texto ao DX (echo), Construtor da clase, Control de erros

\subsubsection{Interacción co usuario}
\paragraph{Data de inicio:} 08/08/2016
\paragraph{Data de fin:} 29/08/2016
\paragraph{Tarefas:} Interface de proba para comunicarse co sistema, base de
datos

\subsubsection{Servizos REST}
\paragraph{Data de inicio:} 29/08/2016
\paragraph{Data de fin:} 19/09/2016
\paragraph{Tarefas:} Substitución de RMI por servizos REST, deseño DemiurgoWeb a
partir de Framework

\subsubsection{Base da memoria}
\paragraph{Data de inicio:} 19/09/2016
\paragraph{Data de fin:} 09/11/2016
\paragraph{Tarefas:} Memoria: introdución e trazos fundamentais no deseño,
memoria: diagrama de clases do universo de xogo

\subsubsection{Deseño inicial da web}
\paragraph{Data de inicio:} 09/11/2016
\paragraph{Data de fin:} 15/12/2016
\paragraph{Tarefas:} Adaptación dos servizos web a JSON, interface web 	para DX,
interface web para usuario, altas de usuarios

\subsubsection{Apartado gráfico da interface}
\paragraph{Data de inicio:} 15/12/2016
\paragraph{Data de fin:} 28/12/2016
\paragraph{Tarefas:} Deseño de CSS, Creación e modificación de clases

\subsubsection{Inventarios e configuración}
\paragraph{Data de inicio:} 28/12/2016
\paragraph{Data de fin:} 08/02/2017
\paragraph{Tarefas:} 
