\chapter{Arquitectura e ferramentas}
Neste capítulo describirase a arquitectura do sistema, detallando os distintos
compoñentes que a forman e como interactúan entre eles. Ademais diso, tamén se
realizará unha análise das tecnoloxías empregadas no proxecto e os motivos da
súa escolla entre as ferramentas dispoñibles no mercado.

\section{Arquitectura do sistema}
A arquitectura do proxecto xa foi parcialmente descrita no anteproxecto, e o
seu aspecto máis importante é a separación do motor de xogo e a interface
gráfica en dous elementos diferenciados.
\par
O sistema está orientado a realizar partidas de rol por internet. Tal e como
a interacción co usuario foi formulada previamente, o usuario (neste caso o
director de xogo) interactúa co sistema mediante o uso de scripts, mais non é a
única forma: a interface debe proporcionar facilidades para realizar as súas
tarefas sen deixar todo en mans da linguaxe. Este programa diferénciase dos MUDs
tradicionais entre outras cousas na forma de entender a interacción
persoa-ordenador, evitando interfaces áridas para deste modo poder chegar a unha
maior masa de xogadores que poden ter coñecementos limitados ou nulos de
programación informática.
\par
Debido a isto, é fundamental asumir a existencia de dous elementos de software
executable diferenciados: o motor de xogo, que chamaremos \textbf{Demiurgo}, e a
interface de usuario. Ambos elementos describiranse máis en detalle na sección
\ref{sec:bloques}.
\par
Cabe destacar que se asumiu en todo momento o motor de xogo como a prioridade do
proxecto, e dentro deste, a elaboración da linguaxe de script para o usuario,
polo que foi o compoñente ao que se destinou unha maior carga de traballo.
\begin{figure}
\centerline{\includegraphics[width=15cm]{figuras/diagramas/arquitectura.pdf}}
\caption{Diagrama da arquitectura do sistema.}
\label{fig:arquitectura}
\end{figure}

\section{Bloques da arquitectura}
\label{sec:bloques}
Como xa se mencionou, o sistema está inicialmente dividido en dous grandes
bloques independentes: o motor de xogo e a interface. A maiores, cada un destes
compoñentes ten unha arquitectura interna que será descrita a continuación.

\subsection{Motor de xogo: Demiurgo}
O motor de xogo é o elemento central do sistema. Foi bautizado co nome de
\textbf{Demiurgo}. As súas principais funcións son manter o mundo (ou mundos) de
xogo en memoria, mantendo ademais a súa coherencia e integridade; realizar o
procesamento dos scripts para modificar o seu estado; e por último, xestionar os
usuarios e o acceso á información. Podemos distinguir os seguintes compoñentes
diferenciados dentro do motor de xogo:

\subsubsection{Motor de Mundo}
O motor de Mundo mantén a consistencia dos mundos das distintas partidas
activas. Garda unha instancia activa de cada clase de xogo, obxecto, escenario e
demais elementos propios da partida, e ofrece métodos para poder modificar o seu
estado sen comprometer a súa coherencia interna. As clases empregadas por este
Motor de Mundo correspóndese co \textit{modelo} do sistema dentro dun patrón
modelo-vista-controlador.

\subsubsection{Intérprete da linguaxe}
O intérprete realiza a análise léxica, sintáctica e semántica do código
script enviado polo director de xogo, para deste modo extraer as súas
instrucións e alterar o mundo de xogo segundo lle sexa indicado. O intérprete
por tanto ten acceso ao Motor do Mundo, co que se comunica durante a lectura do
código continuamente. No capítulo \ref{ch:desenho} especifícase con máis detalle
o funcionamento deste intérprete.

\subsubsection{Xestor de servizos web}
Ofrece un \textit{Endpoint} á interface web sobre o que facer peticións HTTP
para estabelecer o diálogo co motor de xogo. Conta con todos os métodos
necesarios para poder enviar e recibir información co Demiurgo, dende o envío de
instrucións e código script ata as distintas peticións de información sobre o
estado do mundo. As peticións devolven por norma xeral un obxecto JSON, e poden
recibir como argumentos campos de texto (cando se emprega o método \textit{GET})
ou un obxecto JSON (cando se emprega o método \textit{POST}).

\subsubsection{Motor de persistencia}
O motor de persistencia comunícase do xestor da base de datos para manter no
disco duro unha copia do modelo. A súa función por tanto é transformar os
obxectos do Motor de Mundo a linguaxe relacional para deste modo almacenar o
estado do mundo cando o sistema se detén, e facer o proceso contrario para
recuperar tal estado cando o sistema volve a poñerse en funcionamento.

\subsection{Interface web: DemiurgoWeb}
Os usuarios do sistema precisan un medio para comunicarse co Motor de Xogo dunha
forma cómoda e fácil de entender. Neste proxecto asumiuse a creación dunha
aplicación web para suplir esta función, á que se denominou DemiurgoWeb, que
permitirá tanto a directores de xogo como a xogadores interactuar co sistema.
Esta interface web non realizará traballos complexos nin levará o control dos
privilexios de usuarios: a súa función é exclusivamente facer de ponte co
Demiurgo enviando peticións HTTP e mostrando resultados por pantalla aos
usuarios.
\par
Demiurgo e DemiurgoWeb deberán desenvolverse como dous programas totalmente
diferentes e desacoplados, de tal xeito que poidan crearse outras vías para
acceder ao motor de xogo, tales como apps de móbil, por exemplo.
\par
O DemiurgoWeb implementarase en base ao patrón modelo-vista-controlador, polo
que se deben analizar estas tres partes por separado. A maiores, conta cun
compoñente adicional chamado \textit{conector Demiurgo}, que se encarga
de realizar todas as peticións ao Demiurgo.

\subsubsection{Modelo de datos}
O modelo de datos representa todos os tipos de datos que o sistema manexa:
clases de xogo, obxectos, escenarios, etcétera. No DemiurgoWeb estas clases
deberán ser directamente transformables en obxectos JSON e viceversa, xa que
obterá todos os seus datos a través do conector Demiurgo neste formato.

\subsubsection{Vistas}
As vistas son elementos deseñados para mostrar unha representación visual da
información aos usuarios. Non modifican os datos, a súa única función é
expresalos de forma clara e facilitar a interacción con eles.

\subsubsection{Controladores}
Os controladores, seguindo co patrón, serán as clases intermedias que obteñan
os obxectos do modelo para proporcionarllos ás vistas a partir das accións do
usuario.

\subsubsection{Conector Demiurgo}
O conector Demiurgo está fóra do patrón modelo-vista-controlador. A súa única
función é realizar as peticións ao motor de xogo Demiurgo (concretamente, ao
seu xestor de servizos web) dunha forma centralizada cando os controladores llo
requiran. A URL do motor de xogo e o porto deberán serlle especificados ao
conector mediante ficheiros de configuración.

\section{Linguaxe: COE}
Un pilar fundamental deste proxecto, tanto pola súa carga de traballo como pola
súa relevancia dentro do sistema, é a elaboración dunha linguaxe de script nova
para manexar o mundo de xogo por parte do DX. Esta foi a tarefa de maior
prioridade, polo que se definiu e implementou antes de comezar co resto de
compoñentes.
\par
Inicialmente valorouse a posibilidade de empregar algunha linguaxe xa existente
para evitar o esforzo de crear unha nova. Isto tería o beneficio de aforrar
moitísimo tempo e traballo, permitindo dedicar máis tempo a outros aspectos do
proxecto como a interface web. A pesar diso, a decisión final foi a de crear
unha linguaxe de cero, xa que o concepto do programa non ten precedentes e
ningunha das linguaxes dispoñibles cubriría correctamente os distintos aspectos
requeridos sen sacrificar facilidade de uso.
\par
A linguaxe recibiu o nome de \textbf{COE}, acrónimo de \textit{Código de
Obxectos e Escenarios}. No apéndice \ref{ch:manual} deste documento pódese
atopar un manual desta linguaxe orientado ao seu uso por parte de directores de
xogo.
\par
COE conta cunha gramática de tipo LL \cite{compiladores}, que é analizada por un
analizador sintáctico descendente de esquerda a dereita.
\par
A sintaxe da linguaxe COE está fortemente inspirada na de linguaxes amplamente
difundidas como C ou Java. Está deseñada tendo en mente a súa finalidade e
buscando a maior simplicidade posible. É importante resaltar que, no programa,
o código sempre se executa nun de dous posibles contextos:
\begin{itemize}
  \item No contexto de definición dunha clase, no que o script se empregará
  exclusivamente en definir os campos e métodos dunha clase determinada.
  \item No contexto dunha acción nun escenario, no que o script actuará sobre os
  obxectos e variables do propio escenario onde se executa.
\end{itemize}

\section{Análise tecnolóxica}
Antes de comezar con este proxecto foi fundamental realizar unha busca
exhaustiva para coñecer as diferentes ferramentas dispoñibles no mercado. Non
obstante, isto non se detivo en ningún momento do desenvolvemento do mesmo, xa
que a medida que o proxecto avanzaba era inevitable que apareceran cambios nos
requisitos, ou simplemente descubrir novas tecnoloxías dispoñibles, o que
obrigaba a tomar decisións sobre a incorporación de novas ferramentas.

\subsection{Linguaxe de programación: Java SE}
Antes de comezar co desenvolvemento, unha das primeiras decisións que era
necesario tomar era a escolla dunha linguaxe de programación concreta para os
distintos compoñentes. Valoráronse distintas posibilidades tanto para o motor de
xogo como para a interface web.
\par
No caso do motor de xogo, era necesaria unha linguaxe potente e que ofrecera un
rendemento aceptable. Unha prioridade era que tivera dispoñibles librarías
opensource (un requisito do proxecto) para poder xerar linguaxes propias, xa que
era imprescindible para a elaboración da linguaxe COE.
\par
Considerouse no seu momento momento a posibilidade de empregar unha linguaxe
funcional como Scala. Esta idea foi finalmente descartada debido a que a falta
de experiencia con este tipo de linguaxes e a menor documentación provocarían
dificultades e retrasos nun proxecto xa de por si complexo.
\par
Para o DemiurgoWeb valorouse empregar PHP como linguaxe, xa que a experiencia do
programador con esta linguaxe é significativa, e resulta máis sinxelo e
económico montar un servidor Apache para un programa en PHP que un servidor
Tomcat para Java.
Descartouse tamén pola dificultade de manter o código dun proxecto grande en PHP
en comparación cun servidor en Java.
\par
Finalmente a linguaxe escollida foi Java. Os principais argumentos a favor
para realizar esta escolla foron os seguintes:
\begin{itemize}
  \item Java é unha linguaxe amplamente difundida, cunha extensa documentación
  na rede. Ademais disto, o programador deste proxecto está relativamente
  familiarizado coa súa sintaxe.
  \item Dispón de numerosas librarías que permiten desenvolver os distintos
  compoñentes do proxecto. Dispón en particular de ferramentas para desenvolver
  analizadores léxico-sintácticos, e de librarías para montar sistemas web.
  \item É compatible coa maioría de entornos de desenvolvemento, nomeadamente
  Eclipse, entorno que se empregará neste proxecto.
\end{itemize}

\subsection{Entorno de desenvolvemento: Eclipse}
Aínda que non forme parte do entregable final, é importante comentar a elección
do entorno do desenvolvemento polo seu impacto crítico no desenvolvemento do
proxecto. A decisión final estaba entre Eclipse e Netbeans, xa que son as dúas
linguaxes máis próximas á linguaxe Java e que ofrecen máis ferramentas útiles.
Finalmente optouse por Eclipse, xa que conta con numerosos plugins para
facilitar o desenvolvemento dos distintos compoñentes do sistema, a linguaxe
COE entre eles.

\subsection{Xerador de parsers: ANTLR}
Como xa se comentou previamente, a definición e implementación da linguaxe COE é
un dos aspectos críticos deste proxecto, requerindo a maior parte da carga de
traballo do mesmo. Debido a isto, foi necesario buscar entre as distintas
ferramentas do mercado para atopar unha que facilitara a interpretación do
código, e evitar deste modo a necesidade de codificar un analizador
léxico-sintáctico manualmente (co esforzo e tempo desmesurados que iso
requeriría).
\par
Escolleuse ANTLR como ferramenta para deseñar a linguaxe e xerar os
correspondentes analizadores léxico e sintáctico. ANTLR trátase dunha ferramenta
que a partir dunha gramática LL(1) elabora un analizador sintáctico descendente;
grazas a isto podemos centrarnos no deseño da gramática para que satisfaga os
requisitos do proxecto, e deixar en mans de ANTLR o proceso de implementación
do \textit{parser} (recoñecedor de linguaxe).
\par
O principal motivo polo que se escolleu ANTLR sobre outras ferramentas foi a súa
capacidade de xerar código en Java de forma directa. Hai dispoñible un manual de
usuario de ANTLR 4 (a versión empregada neste proxecto) \cite{antlr}.

\subsection{Apache Maven}
Tanto no Demiurgo como no DemiurgoWeb, decidiuse empregar Maven para
xestionar os diversos paquetes e dependencias, así como os distintos elementos
software de ambos compoñentes.
As principais vantaxes disto son:
\begin{itemize}
  \item Ofrece unha estrutura de directorios estandarizada que facilita o
  mantemento do código de cara ao futuro do proxecto.
  \item Simplifica a xestión de dependencias, descargando de forma automática os
  paquetes necesarios e aforrando tempo neste aspecto.
  \item Simplifica a execución de tests no código.
\end{itemize}

\subsection{Servizos REST}
Un aspecto importante do sistema era a forma na que a interface web (e outras
posibles interfaces futuras non contempladas no alcance deste proxecto) se
comunicaría co motor de xogo. Unha primeira aproximación suxeriu o uso de
Java RMI para comunicar ambos compoñentes, aproveitando o feito de que ambos
corrían sobre JVM. Non obstante, optouse finalmente por descartar esta
opción e empregar servizos web de tipo REST en base ás seguintes cuestións:
\begin{itemize}
  \item Java RMI só funciona entre programas que corran sobre JVM. A pesar de
  non ser este un problema no alcance actual deste proxecto, limita a evolución
  do sistema no futuro; empregando servizos web ábrese a porta a comunicarse co
  servidor de xogo mediante outras aplicacións desenvolvidas noutras
  tecnoloxías.
  \item Java RMI é unha tecnoloxía menos flexible cós servizos web, de
  mantemento máis complexo e que pode xerar dificultades cando os dous
  servidores se atopan en distintas ubicacións físicas. Pola contra, os servizos
  web só requiren dunha conexión vía HTTP entre ambos servidores, polo que son
  máis fáciles de empregar.
  \item Con Java RMI é necesario compartir código entre ambos servidores a
  través de librarías comúns para poder compartir POJOs (\textit{Plain Old
  Java Objects}). Mediante os servizos web, en cambio, a comunicación pode
  realizarse mediante obxectos JSON, totalmente independentes do entorno.
\end{itemize}
Os servizos REST esencialmente fundaméntanse nos propios métodos de HTTP: GET,
POST, PUT e DELETE. As peticións web son simples peticións por HTTP que conteñen
obxectos JSON no corpo da mensaxe. Para este proxecto, todos os métodos son de
tipo POST ou GET, pero poderían empregarse métodos distintos se se considerase
necesario; a elección do método baséase no seguimento duns estándares, mais non
afecta realmente ao funcionamento.

\subsection{Servidor web do motor de xogo: Grizzly}
O motor de xogo precisa de ofrecer un \textit{endpoint} ao que se lle poidan
facer peticións web, polo que foi necesario buscar unha solución para isto que
permitira manter a funcionalidade do motor intacta. A escolla dos servizos REST
sobre Java RMI realizouse nun momento no que o Demiurgo xa tiña un certo nivel
de profundidade, polo que un requisito para a ferramenta a escoller era que non
requerira modificar severamente o código.
\par
Grizzly é unha ferramenta que permite que un executable Jar normal poida montar
un servidor web sen necesidade de usar contenedores como Tomcat. Nese sentido,
escolleuse por tratarse da opción máis sinxela para resolver o problema. Ademais
disto, outro factor determinante foi a facilidade para empregar Grizzly en
conxunto con Jersey, unha ferramenta para crear servizos REST.

\subsection{Xestor de servizos REST: Jersey}
Jersey e Grizzly son dúas librarías que foron escollidas simultaneamente xa que,
como se comentou na subsección de Grizzly, son librarías que ofrecen gran
facilidade para empregarse xuntas, e hai numerosos manuais en internet para
crear servidores con estas dúas ferramentas. A finalidade de Jersey neste
sentido é a de crear \textit{endpoints} nos que implementar os servizos REST que
precisemos, os cales se encargará Grizzly de publicar nun servidor para recibir
peticións.

\subsection{Spring Framework}
Spring é un Framework de Java que facilita o desenvolvemento de aplicacións
grandes. A principal característica de Spring é a inxección de dependencias, que
simplifica o control do ciclo de vida dos obxectos: crear os distintos
compoñentes, inicializalos e manter as referencias entre eles. A maiores, dispón
de módulos útiles para todo tipo de tarefas, tales como montar servidores web,
empregar o patrón modelo-vista-controlador ou executar tests de JUnit.
\par
Neste proxecto empregaremos Spring no DemiurgoWeb. Debido a isto, será necesario
invertir algo de tempo en familiarizarse co funcionamento de Spring, pero unha
vez superado isto axilizará enormemente a implementación do compoñente web e o
seu mantemento.
\par
Pódese atopar máis información sobre Spring na súa páxina web \cite{spring}.

\subsubsection{Spring Boot}
Spring Boot é un módulo de Spring que permite crear ficheiros Jar executables
sen necesidade de empregar Tomcat ou ferramentas similares; Spring Boot
introduce Tomcat no interior do ficheiro Jar para obter un efecto semellante ao
que conseguimos no Demiurgo con Grizzly. Ademais disto, é posible configuralo
para empregar ficheiros WAR que funcionen como executables e tamén para montar
en contenedores web, ofrecendo maior versatilidade. Deste modo, aforramos tempo
ao evitar os erros que se puideran cometer na configuración de contenedores web.

\subsubsection{Spring Web}
Spring ofrece numerosas ferramentas para o desenvolvemento de aplicacións web.
No canto de precisar servlets para as distintas webs, Spring ofrece un servlet
propio que xestiona as distintas peticións e as redirixe aos controladores e
vistas correspondentes, sen necesidade de ficheiros de configuración adicionais.
Este módulo, e o Framework en conxunto, facilitan moito a división do código en
modelo, vista e controlador.

\subsubsection{Thymeleaf}
Thymeleaf é un motor que nos permite enriquecer o código HTML das vistas de
Spring para dotalo de funcionalidade, e deste modo mostrar os datos do modelo
que o controlador lle proporciona. Inicialmente considerouse empregar JSP para
este fin, pero a opción final foi Thymeleaf por varios motivos:
\begin{itemize}
  \item Thymeleaf é unha linguaxe máis nova e con maior potencia que JSP.
  \item Thymeleaf é plenamente compatible con Spring e a interacción entre ambos
  é moi natural.
  \item Thymeleaf emprega ficheiros HTML e non precisa de etiquetas adicionais,
  a diferencia de JSP, que emprega ficheiros JSP coas súas propias etiquetas;
  isto permite que os ficheiros HTML poidan ser visualizados directamente nun
  navegador con datos de proba sen necesidade de montar o servidor.
\end{itemize}

\subsection{Javascript}
No apartado web empregouse Javascript para ofrecer contidos dinámicos nas
distintas páxinas, e deste modo ofrecer unha interface máis amigable de cara ao
usuario. A día de hoxe a inmensa maioría de usuarios empregan Javascript na súa
navegación habitual.

\subsubsection{JQuery}
JQuery é unha libraría de Javascript que ofrece numerosas facilidades para facer
accións complexas dentro da páxina. A maiores, moitos plugins que engaden
funcionalidades avanzadas empregan JQuery como dependencia.

\paragraph{JQuery UI:}
JQuery UI é un plugin de JQuery que facilita o uso de \textit{drag-and-drop} na
páxina (arrastrar e soltar). Empregouse en determinados puntos da web para
facilitar a elaboración de scripts por parte do DX.

\subsubsection{CodeMirror}
CodeMirror é un plugin baseado en Javascript nativo que ofrece unha área de
texto destinada á escritura de código con funcionalidades que o
\textit{textarea} de HTML non ofrece, como resaltar a sintaxe do código ou
acceder a posicións do texto mediante liña e columna, por exemplo. Moi útil á
hora de redactar código COE.

\subsection{Base de datos: MariaDB}
Un dos requisitos do proxecto era a persistencia dos datos unha vez o sistema
deixa de funcionar. Para iso é necesario o uso de bases de datos, e con elas, de
sistemas xestores de bases de datos.
\par MariaDB é un xestor de bases de datos derivado de MySQL. Optouse polo uso
de MariaDB pola súa plena compatibilidade con MySQL e pola súa alta velocidade.

\subsection{Control de versións: Git}
Git é unha ferramenta para desenvolvedores que xestiona os cambios realizados no
código, mantendo un control de versións e facilitando o desenvolvemento de
software.
\par
Neste proxecto empregaremos repositorios Git para os distintos compoñentes, xa
que é unha ferramenta coa que o programador está familiarizado, e pola
posibilidade de manter repositorios públicos na web GitHub de forma gratuíta.