\chapter{Deseño}

%Deseño: cómo se realiza o Sistema, a división deste en diferentes compoñentes e
%a comunicación entre eles. Así mesmo, determinarase o equipamento hardware e
%software necesario, xustificando a súa elección no caso de que non fora un
%requisito previo. Debe achegarse a un nivel suficiente de detalle que permita
%comprender a totalidade da estrutura do produto desenvolvido, utilizando no 
%posible representacións gráficas.

O sistema deste proxecto está inicialmente dividido en dous compoñentes ben
diferenciados pero comunicados entre si:
\begin{itemize}
  \item O servidor de xogo, co nome interno de {\it Demiurgo},
  compoñente fundamental do sistema encargado de xestionar internamente os
  mundos de xogo.
  \item O servidor web co que interactuarán os usuarios, que terá como cometido
  facer de ponte entre usuarios e servidor de xogo sen aplicar ningunha lóxica
  interna.
\end{itemize}

Estes dous compoñentes comunícanse entre si mediante servizos web de tipo REST:
o servidor web recibe peticións dos usuarios e fai peticións ao servidor de
xogo mediante o envío de obxectos JSON a través de HTTP.

\section{Servidor de xogo: Demiurgo}
O servidor do xogo contén todo o necesario para manter unha partida de rol:
mantemento e xestión do universo de xogo, e xestión de usuarios. A comunicación
con el realízase por medio de servizos web. O nome que se lle deu a este
software é {\it Demiurgo}.
\subsection{Linguaxe COE}
O sistema precisa dunha linguaxe de script que permita ao Director de
Xogo manipular os obxectos que compoñen o mundo. Esta linguaxe desenvolveuse
mediante a ferramenta ANTLR, e déuselle o nome de {\bf COE} (acrónimo de {\it
Código de Obxectos e Escenarios}). Esta é unha linguaxe orientada a obxectos
cunha sintaxe que lembra a Java ou C++, pero especializada na xestión de
partidas de rol.
\par
O funcionamento desta linguaxe defínese con maior detalle no capítulo 5 deste
documento.

\subsection{Universo de xogo}
O servidor é o encargado de manter en memoria todos os obxectos que poboan o
universo do xogo, ademais de almacenar tamén as clases e os usuarios. Todo isto
é almacenado en forma de {\it POJOs} ({\it Plain Old Java Objects}), é dicir,
obxectos Java. A medida que o servidor executa código enviado polo Director de
Xogo, o universo modifícase segundo sexa preciso, creando e modificando
obxectos do xogo.
\par
Todos os obxectos de xogo, como se precisou en capítulos anteriores, están
vinculados a escenarios creados polo propio Director de Xogo. Estes escenarios
tamén son almacenados neste universo de xogo.
\par
Nun mesmo servidor é posible ter varios universos de xogo en funcionamento. Os
distintos universos non teñen ningún tipo de comunicación entre si: funcionan
como compartimentos estancos.
\par
Nos anexos pódense atopar os diagramas de clases que detallan polo miúdo o
universo de xogo.

\subsection{Servizos REST}
A comunicación co servidor de xogo realízase mediante o uso de servizos web
de tipo REST.
O servidor provee de todas as funcións necesarias para o seu correcto manexo, que
serán empregadas tanto polo servidor web como por potenciais terceiras
aplicacións.
\par
Estes servizos realízanse enviando obxectos JSON vía HTTP, que conterán todos os
datos precisos que a función requira. O servidor pola outra banda responde con
JSON que conteñen os datos requeridos, ou unha confirmación como mínimo.
\par
Todas as peticións REST requiren autenticación do usuario. Unha petición
de login recibe o nome do usuario e o contrasinal, devolvendo un token que
identificará ao usuario nas posteriores peticións. Este token cifrado contén o
nome de usuario e os seus permisos, diferenciando deste modo entre xogadores e
directores de xogo. Debido a isto, o servidor non precisa almacenar o estado dos
usuarios conectados.

\subsection{Base de datos}
O servidor de xogo comunícase cunha base de datos SQL para garantir a
persistencia do universo de xogo. Nesta almacénase o estado de todos os obxectos
e escenarios. O estado cárgase da base de datos cando se inicia o programa, e
gárdase cando se finaliza a súa execución.

\section{Servidor web}
O servidor web mostra unha interface amigable de cara aos usuarios para poder
acceder ao sistema. Non ten lóxica interna: a súa función exclusiva é a 
comunicación co servidor de xogo.
\par
Neste proxecto, a maior parte da carga de traballo irá destinada ao servidor de
xogo por ser o compoñente maior, polo que a complexidade do servidor web é
reducida. Isto non quita que poda ser mellorado no futuro.

\subsection{Spring Framework}
O servidor web está desenvolvido mediante o uso de Spring Framework, unha
libraría opensource para Java que ofrece ferramentas para crear un produto
software completo. Grazas a este framework é posible centrarse no deseño da web
e reducir o tempo de traballo en cuestións técnicas.
\par
Spring ofrécenos un {\it servlet} que podemos configurar mediante un ficheiro
XML de configuración. Este servlet conta cun xestor de vistas, co cal podemos
enlazar os ficheiros que conteñen as vistas (ficheiros {\it jsp}) coas clases
Java que se encargan de recibir e mostrar os datos ({\it Controllers}).

\subsubsection{Vistas}
O código HTML que determina o aspecto visual das distintas páxinas está contido
en ficheiros JSP. Ademais de HTML, inclúense etiquetas JSP para introducir os
datos obtidos mediante o controlador asociado. Tamén se emprega a libraría JSTL
para poder empregar certas funcionalidades (como bucles e condicionais) que
faciliten a representación destes datos.

\subsubsection{Controladores}
Os controladores son clases Java que se encargan de procesar os datos
transmitidos entre o usuario e o servidor de xogo. Cada controlador ten unha
vista asociada, que o servlet de Spring localiza polo seu nome e devolve ao
usuario, engadindo previamente o controlador os datos necesarios.
