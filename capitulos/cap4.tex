\chapter{Deseño}


\section{Descrición do sistema}
O proxecto consta dun software entregable formado por varios compoñentes
diferenciados pero interrelacionados entre si:
\begin{itemize}
  \item A linguaxe na que se fundamenta o sistema. O sistema precisa unha
  linguaxe orientada a obxectos, na que se poidan especificar clases cos seus
  atributos e métodos. Unha parte crucial deste traballo é a de describir
  formalmente esta linguaxe de forma que cumpra a súa función de definir o
  mundo virtual do xogo, tal e como se indican nos obxectivos deste
  anteproxecto. Debe realizarse unha análise das solucións abertas dispoñibles,
  e no caso de non atopar unha linguaxe que satisfaga todas as necesidades do
  sistema, será necesario implementar unha de cero co seu correspondente
  intérprete.
  \begin{itemize}
    \item Esta linguaxe deberá conter todo o necesario para un correcto
    tratamento da información:
    variables, operadores aritméticos e lóxicos, condicións lóxicas,
    modificación de atributos, etc.
    Todo isto debe ser debidamente especificado na etapa de deseño.
  \end{itemize}
  \item A interface web que tanto o director como os xogadores empregarán para
  interactuar co mundo.
  \begin{itemize}
    \item Os xogadores poderán en todo momento ver o estado do seu personaxe e
    ler o texto descritivo das accións sucedidas na súa presencia (redactado e
    filtrado segundo o criterio do director de xogo), ademais de escribir en
    linguaxe natural as actuacións que desexan realizar ({\it decisións} na
    terminoloxía do sistema) na seguinte iteración ou quenda. En ningún momento
    terán acceso á linguaxe de xogo propiamente dita, nin a datos do mundo
    virtual alleos ao seu personaxe.
    \item O director de xogo pola súa parte terá acceso a unha consola de comandos,
    que lle permitirá alterar o mundo virtual mediante o uso da linguaxe de
    xogo. Recibirá todas as peticións de actuación dos xogadores e escribirá o
    código en consecuencia que deberá ser executado; en tempo de execución o 
   director de xogo poderá instanciar novos obxectos, chamar aos métodos de
   obxectos xa instanciados ou modificar os seus atributos directamente.
  \end{itemize}
  \item A capa de software que mantén o estado do mundo e xestiona o resto de
  aspectos do sistema, tales como a xestión de usuarios, ademais de servir de
  ponte entre o resto de compoñentes.
  \item Outros compoñentes necesarios para o funcionamento do sistema: servidor
  web e base de datos relacional.
\end{itemize}
O software a entregar por tanto terá dous elementos diferenciados: a interface
web, sen lóxica interna e que se limitará a enviar e recibir mensaxes, e o
motor do xogo, que se conectará coa interface web para comunicarse, pero que
debe deseñarse tendo en mente a posibilidade de incorporar outros modos de
comunicarse con el no futuro, tales como software de escritorio para os
usuarios. Estas vías alternativas e o seu desenvolvemento, en todo caso, non
entran no alcance deste proxecto concreto.

\section{Información adicional de interese}
% métodos, técnicas ou arquitecturas utilizadas, xustificación da súa elección
\subsection{Tecnoloxías empregadas}

\subsubsection{Java SE}
Para o desenvolvemento do software deste Traballo de Fin de Grao escolleuse Java
como linguaxe para os distintos compoñentes.
Os principais argumentos a favor para realizar esta escolla foron os seguintes:
\begin{itemize}
  \item Java é unha linguaxe amplamente difundida, cunha extensa documentación
  na rede. Ademais disto, o autor deste TFG está familiarizado coa súa sintaxe.
  \item Dispón de numerosas librarías que permiten desenvolver os distintos
  compoñentes do proxecto. Dispón en particular de ferramentas para desenvolver
  analizadores léxico-sintácticos, e de librarías para montar sistemas web.
  \item É compatible coa maioría de entornos de desenvolvemento, nomeadamente
  Eclipse, entorno que se empregará neste proxecto.
\end{itemize}

\subsubsection{ANTLR}
Un dos pilares centrais deste Traballo de Fin de Grao é a linguaxe de script
empregada polo Director de Xogo para comunicarse co sistema. Debido á alta
especifidade deste proxecto, optouse por deseñar unha linguaxe de
cero no canto de adaptar unha linguaxe de propósito xeral.
\par
Escolleuse ANTLR como ferramenta para deseñar a linguaxe e xerar os
correspondentes analizadores léxico e sintáctico. ANTLR trátase dunha ferramenta
que a partir dunha gramática elabora un analizador sintáctico descendente;
grazas a isto podemos deseñar unha gramática que satisfaga os requisitos do
proxecto e deixar en mans de ANTLR o proceso de parsing.

\subsubsection{Servizos web: Grizzly + Jersey}
Para a comunicación entre o servidor de xogo e o servidor web valoráronse
distintas posibilidades e métodos. Unha primeira aproximación suxeriu o uso de
Java RMI para comunicar ambos compoñentes, aproveitando o feito de que ambos
corrían sobre JVM. Non obstante, optouse finalmente por descartar esta
opción e empregar servizos web de tipo REST en base ás seguintes cuestións:
\begin{itemize}
  \item Java RMI só funciona entre programas que corran sobre JVM. A pesar de
  non ser este un problema no alcance actual deste proxecto, limita a evolución
  do sistema no futuro; empregando servizos web ábrese a porta a comunicarse co
  servidor de xogo mediante outras aplicacións desenvolvidas noutras
  tecnoloxías.
  \item Java RMI é unha tecnoloxía menos flexible cós servizos
  web, de mantemento máis complexo e que pode xerar dificultades cando os dous
  servidores se atopan en distintas ubicacións físicas. Pola contra, os servizos
  web só requiren dunha conexión vía HTTP entre ambos servidores, polo que son
  máis fáciles de empregar.
  \item Con Java RMI é necesario compartir código entre ambos servidores a
  través de librarías comúns para poder compartir POJOs ({\it Plain Old Java
  Objects}). Mediante os servizos web, en cambio, a comunicación pode realizarse
  mediante obxectos JSON.
\end{itemize}
Para implementar os servizos web optouse por empregar as librarías Grizzly e
Jersey, por tratarse da opción máis sinxela para poñer en funcionamento o
servidor. Por un lado, Grizzly ofrece todo o necesario para poder montar un
servidor web sen necesidade de programas adicionais, empregando simplemente o
executable da nosa aplicación. Polo outro, Jersey encárgase da recepción,
manipulación e resposta das peticións web recibidas, axilizando a
implementación do sistema.

\subsubsection{Apache Maven}
Tanto no servidor de xogo como no servidor web, decidiuse empregar Maven para
xestionar os distintos paquetes e dependencias. As principais vantaxes disto
son:
\begin{itemize}
  \item Ofrece unha estrutura de directorios estandarizada que facilita o
  mantemento do código.
  \item Simplifica a xestión de dependencias, descargando de forma automática os
  paquetes necesarios.
  \item Simplifica a execución de tests no código.
\end{itemize}

\subsubsection{Spring Framework}
Spring é un Framework que facilita o desenvolvemento de aplicacións complexas.
Ten módulos útiles para todo tipo de tarefas, tales como montar un servidor web,
empregar o patrón modelo-vista-controlador ou executar tests de JUnit. A
principal característica de Spring é a inxección de dependencias, que simplifica
o control do ciclo de vida dos obxectos: crear os obxectos, chamar aos seus
métodos de inicialización, e referencialos entre eles.
\par
Neste proxecto empregaremos Spring no compoñente web, é dicir, Spring axudará a
crear un servidor web que se conecte co servidor de xogo. Deste modo, aforrarase
tempo que poderá ser empregado no deseño da web.

\subsubsection{MariaDB}
MariaDB é un xestor de bases de datos derivado de MySQL. Empregaremos bases de
datos MariaDB no noso proxecto para manter a persistencia do sistema de xogo.

\subsubsection{Git}
Git é unha ferramenta para desenvolvedores que xestiona os cambios realizados no
código, mantendo un control de versións e facilitando o desenvolvemento de
software.
\par
Neste proxecto empregaremos repositorios Git para os distintos compoñentes.
Usaremos Github para almacenar estes repositorios nun lugar publicamente
accesible.

%Deseño: cómo se realiza o Sistema, a división deste en diferentes compoñentes e
%a comunicación entre eles. Así mesmo, determinarase o equipamento hardware e
%software necesario, xustificando a súa elección no caso de que non fora un
%requisito previo. Debe achegarse a un nivel suficiente de detalle que permita
%comprender a totalidade da estrutura do produto desenvolvido, utilizando no 
%posible representacións gráficas.

O sistema deste proxecto está inicialmente dividido en dous compoñentes ben
diferenciados pero comunicados entre si:
\begin{itemize}
  \item O servidor de xogo, co nome interno de {\it Demiurgo},
  compoñente fundamental do sistema encargado de xestionar internamente os
  mundos de xogo.
  \item O servidor web co que interactuarán os usuarios, que terá como cometido
  facer de ponte entre usuarios e servidor de xogo sen aplicar ningunha lóxica
  interna.
\end{itemize}

Estes dous compoñentes comunícanse entre si mediante servizos web de tipo REST:
o servidor web recibe peticións dos usuarios e fai peticións ao servidor de
xogo mediante o envío de obxectos JSON a través de HTTP.

\section{Servidor de xogo: Demiurgo}
O servidor do xogo contén todo o necesario para manter unha partida de rol:
mantemento e xestión do universo de xogo, e xestión de usuarios. A comunicación
con el realízase por medio de servizos web. O nome que se lle deu a este
software é {\it Demiurgo}.
\subsection{Linguaxe COE}
O sistema precisa dunha linguaxe de script que permita ao Director de
Xogo manipular os obxectos que compoñen o mundo. Esta linguaxe desenvolveuse
mediante a ferramenta ANTLR, e déuselle o nome de {\bf COE} (acrónimo de {\it
Código de Obxectos e Escenarios}). Esta é unha linguaxe orientada a obxectos
cunha sintaxe que lembra a Java ou C++, pero especializada na xestión de
partidas de rol.
\par
O funcionamento desta linguaxe defínese con maior detalle no capítulo 5 deste
documento.

\subsection{Universo de xogo}
O servidor é o encargado de manter en memoria todos os obxectos que poboan o
universo do xogo, ademais de almacenar tamén as clases e os usuarios. Todo isto
é almacenado en forma de {\it POJOs} ({\it Plain Old Java Objects}), é dicir,
obxectos Java. A medida que o servidor executa código enviado polo Director de
Xogo, o universo modifícase segundo sexa preciso, creando e modificando
obxectos do xogo.
\par
Todos os obxectos de xogo, como se precisou en capítulos anteriores, están
vinculados a escenarios creados polo propio Director de Xogo. Estes escenarios
tamén son almacenados neste universo de xogo.
\par
Nun mesmo servidor é posible ter varios universos de xogo en funcionamento. Os
distintos universos non teñen ningún tipo de comunicación entre si: funcionan
como compartimentos estancos.
\par
Nos anexos pódense atopar os diagramas de clases que detallan polo miúdo o
universo de xogo.

\subsection{Servizos REST}
A comunicación co servidor de xogo realízase mediante o uso de servizos web
de tipo REST.
O servidor provee de todas as funcións necesarias para o seu correcto manexo, que
serán empregadas tanto polo servidor web como por potenciais terceiras
aplicacións.
\par
Estes servizos realízanse enviando obxectos JSON vía HTTP, que conterán todos os
datos precisos que a función requira. O servidor pola outra banda responde con
JSON que conteñen os datos requeridos, ou unha confirmación como mínimo.
\par
Todas as peticións REST requiren autenticación do usuario. Unha petición
de login recibe o nome do usuario e o contrasinal, devolvendo un token que
identificará ao usuario nas posteriores peticións. Este token cifrado contén o
nome de usuario e os seus permisos, diferenciando deste modo entre xogadores e
directores de xogo. Debido a isto, o servidor non precisa almacenar o estado dos
usuarios conectados.

\subsection{Base de datos}
O servidor de xogo comunícase cunha base de datos SQL para garantir a
persistencia do universo de xogo. Nesta almacénase o estado de todos os obxectos
e escenarios. O estado cárgase da base de datos cando se inicia o programa, e
gárdase cando se finaliza a súa execución.

\section{Servidor web}
O servidor web mostra unha interface amigable de cara aos usuarios para poder
acceder ao sistema. Non ten lóxica interna: a súa función exclusiva é a 
comunicación co servidor de xogo.
\par
Neste proxecto, a maior parte da carga de traballo irá destinada ao servidor de
xogo por ser o compoñente maior, polo que a complexidade do servidor web é
reducida. Isto non quita que poda ser mellorado no futuro.

\subsection{Spring Framework}
O servidor web está desenvolvido mediante o uso de Spring Framework, unha
libraría opensource para Java que ofrece ferramentas para crear un produto
software completo. Grazas a este framework é posible centrarse no deseño da web
e reducir o tempo de traballo en cuestións técnicas.
\par
Spring ofrécenos un {\it servlet} que podemos configurar mediante un ficheiro
XML de configuración. Este servlet conta cun xestor de vistas, co cal podemos
enlazar os ficheiros que conteñen as vistas (ficheiros {\it jsp}) coas clases
Java que se encargan de recibir e mostrar os datos ({\it Controllers}).

\subsubsection{Vistas}
O código HTML que determina o aspecto visual das distintas páxinas está contido
en ficheiros JSP. Ademais de HTML, inclúense etiquetas JSP para introducir os
datos obtidos mediante o controlador asociado. Tamén se emprega a libraría JSTL
para poder empregar certas funcionalidades (como bucles e condicionais) que
faciliten a representación destes datos.

\subsubsection{Controladores}
Os controladores son clases Java que se encargan de procesar os datos
transmitidos entre o usuario e o servidor de xogo. Cada controlador ten unha
vista asociada, que o servlet de Spring localiza polo seu nome e devolve ao
usuario, engadindo previamente o controlador os datos necesarios.
