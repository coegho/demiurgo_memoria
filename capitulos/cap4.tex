\chapter{Deseño}

%Deseño: cómo se realiza o Sistema, a división deste en diferentes compoñentes e
%a comunicación entre eles. Así mesmo, determinarase o equipamento hardware e
%software necesario, xustificando a súa elección no caso de que non fora un
%requisito previo. Debe achegarse a un nivel suficiente de detalle que permita
%comprender a totalidade da estrutura do produto desenvolvido, utilizando no 
%posible representacións gráficas.

O sistema deste proxecto está inicialmente dividido en dous compoñentes ben
diferenciados pero comunicados entre si:
\begin{itemize}
  \item O servidor de xogo, co nome interno de {\it Demiurgo},
  compoñente fundamental do sistema encargado de xestionar internamente os
  mundos de xogo.
  \item O servidor web co que interactuarán os usuarios, que terá como cometido
  facer de ponte entre usuarios e servidor de xogo sen aplicar ningunha lóxica
  interna.
\end{itemize}

Estes dous compoñentes comunícanse entre si mediante servizos web de tipo REST:
o servidor web recibe peticións dos usuarios e fai peticións ao servidor de
xogo mediante o envío de obxectos JSON a través de HTTP.

\section{Servidor de xogo}
\subsection{Linguaxe COE}
O sistema precisa dunha linguaxe de script que permita ao Director de
Xogo manipular os obxectos que compoñen o mundo. Esta linguaxe desenvolveuse
mediante a ferramenta ANTLR, e déuselle o nome de {\bf COE} (acrónimo de {\it
Código de Obxectos e Escenarios}). Esta é unha linguaxe orientada a obxectos
cunha sintaxe que lembra a Java ou C++, pero especializada na xestión de
partidas de rol.

\subsection{Universo de xogo}

\subsection{Servizos web}

\subsection{Base de datos}

\section{Servidor web}